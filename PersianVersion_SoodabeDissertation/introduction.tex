\newpage
\vspace*{-1cm}
\section*{پیش‌گفتار}
\addcontentsline{toc}{section}{پیش‌گفتار}
تئوری تراوش\LTRfootnote{Percolation}، به توصیف رفتار خوشه‌های به هم پیوسته در گراف‌های با ساختار تصادفی  می‌پردازد \cite{chris}. تاریخچه این نظریه به سال $1957$ برمی‌گردد، زمانی که برادبنت\LTRfootnote{Broadbent} و هَمرسلی\LTRfootnote{Hammersley} در پی جوابی برای یافتن مسیری برای عبور مایعات از میان مواد متخلخل بودند\cite{broad}.  نظریه تراوش را کم و بیش می‌توانیم در طبیعت ببینیم؛ مطالعه رشد بیماری در جامعه انسان‌ها، عبور اطلاعات از شبکه جهانی وب، آتش‌سوزی جنگل و مثال‌هایی از این نوع مساله تراوش را برایمان روشن می‌سازد \cite{hasen}. در واقع می‌توان گفت تراوش، الگوریتمی است که مسیر رشد اطلاعات را در یک شبکه نشان می‌دهد. این مسیر بزرگترین مسیر به هم پیوسته از اجزای  شبکه است. 

در بررسی مساله تراوش، به حالت تعمیم‌یافته‌ای از آن برمی‌خوریم که تراوش خودراه‌انداز نامیده می‌شود.  تراوش خودراه‌انداز در سال $1979$ با کار‌های چالوپا\LTRfootnote{Challpa}، لِث\LTRfootnote{Heath} و  ریچ\LTRfootnote{Reich} در جریان مطالعاتشان بر روی سیستم‌های بی‌نظم مغناطیسی شناخته شد. بعد از آن این نظریه به خاطر اهمیت در ارتباط‌ با مدل‌های فیزیکی و نیز کاربرد‌های مختلف  مانند فعالیت‌های  نورونی و گذار‌های ناگهانی مورد مطالعه فیزیک‌دانان و سایر علوم قرار گرفته است \cite{go}.  در این مدل، مکان‌ها به شکل تصادفی اشغال می‌شوند. در پی آن، مکان‌های غیر فعال با  شرط داشتن حداقل $m$ همسایه فعال، قادر خواهند بود به حالت فعال درآیند. این روند تا رسیدن شبکه به حالت پایا ادامه خواهد داشت؛ جایی که دیگر راسی نتواند فعال شود. 

از جمله کاربرد‌های تراوش در شبکه‌های نورونی است که با  استفاده از این تئوری انتقال  و رشد اطلاعات بین نورون‌ها بررسی می‌شود. نورون‌ها در یک شبکه عصبی دارای اتصال‌های قوی هستند و اطلاعات از طریق سیناپس‌ها به نورون‌ها منتقل می‌شوند. مغز انسان یکی از جالب توجه‌ترین سامانه‌های پیچیده محسوب می‌شود، اما به دلیل ساختار بسیار پیچیده و عدم توانایی در محاسبه این پیچیدگی، تا به امروز شناخت کاملی از شبکه مغز انجام نشده است\cite{stam}. 
مغز انسان به طور تقریبی از $100$ میلیارد نورون تشکیل شده است.  اتصال بین این نورون‌ها با یکدیگر یکی از ویژگی‌های اساسی برای درک و ساختار و مکانیزم شبکه نورونی است\cite{sorian}. 
در چند دهه اخیر شبکه‌های نورونی به دلیل شباهتشان به شبکه‌های پیچیده و گراف‌های تصادفی مورد توجه فیزیک‌دانان و ریاضی‌دانان قرار گرفته است. تئوری گراف، ساختار‌های پیچیده شبکه‌های واقعی و مصنوعی را از نقطه نظر مفاهیم اساسی شبکه‌های پیچیده به زبانی ساده بیان می‌کند \cite{boccara}. از جمله این مفاهیم می‌توان به خوشگی\LTRfootnote{Clustring} و انواع گذار‌ها در شبکه اشاره کرد. در شبکه‌های با ابعاد بی‌نهایت و تصادفی، یکی از ساده‌ترین روش‌ها برای نشان دادن این گونه مفاهیم تئوری تراوش بر روی گراف‌های تصادفی است\cite{sorian}. با استفاده از این تئوری، فعالیت بین نورون‌ها و نیز گسترش آن را در شبکه بررسی می‌کنیم.

در این پایان‌نامه فعال‌سازی شبکه نورونی بر روی گراف  تصادفی با دو توزیع درجه همگن و گاوسی مورد مطالعه قرار گرفته است و با استفاده از الگوریتم تراوش توانسته‌ایم اندازه بزرگ‌ترین خوشه به هم پیوسته از نورون‌های فعال در شبکه را بررسی کنیم. 

فصل اول رساله مروری است بر مفاهیم ابتدایی شبکه عصبی، شامل تعاریف اولیه نورون و اجرا آن و نیز تشریح فعالیت نورون‌ها در سیستم عصبی. فصل دوم اشاره‌ای به شبکه‌های پیچیده و مدل‌‌های آن دارد. در فصل سوم تراوش در حالت استاندارد روی شبکه مربعی، گراف تصادفی و  نیز نتایج تجربی که  روی شبکه نورونی  انجام شده است را بررسی کردیم و در نهایت در فصل آخر تئوری  تراوش خودراه‌انداز و شبیه‌سازی شبکه نورونی مورد بحث قرار گرفته است.





