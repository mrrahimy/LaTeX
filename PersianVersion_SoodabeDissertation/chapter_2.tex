\chapter{شبکه‌های پیچیده }
\section{مقدمه}
  شبکه‌های بزرگ مقیاس به عنوان یک سامانه پیچیده\LTRfootnote{complex network} رفتارهای پیچیده‌ای از خود نشان می‌دهند \cite{boccara}. شبکه‌های کامپیوتر، جامعه انسان‌ها  و شبکه‌های نورونی نمونه‌هایی از شبکه پیچیده هستند که از تعداد زیادی عناصر متصل به هم ساخته شده‌اند \cite{newman}. شبکه پیچیده، شبکه‌ای است  که رفتار کلی آن با بررسی رفتار تک تک عناصر ممکن نمی‌باشد بلکه باید رفتار دسته‌جمعی شبکه مطالعه شود. برای بررسی این شبکه‌ها از نظریه گراف استفاده می‌شود \cite{costa}. در این نظریه‌ی ریاضی، شبکه مانند گرافی در نظر گرفته می‌شود  که راس‌های\LTRfootnote{vertex} آن عناصر شبکه و اتصال‌ بین راس‌ها، یال‌های\LTRfootnote{edge} آن را تشکیل می‌دهد \cite{boc,albert}. مطالعه شبکه سابقه‌ی طولانی در ریاضیات گسسته، جامعه‌شناسی و نظریه گراف داشته است و اخیرا فیزیک و زیست‌شناسی را نیز تحت تاثیر خود قرار داده است. 

مطالعه شبکه و تئوری گراف به اوایل سده ۱۸ برمی‌گردد، زمانی‌ که اویلر\LTRfootnote{Leonhard\,Paul\,Eule} در پی جوابی برای حل مساله‌ی هفت پل کونیگسبرگ\RTLfootnote{نام امروزیش کالینینگراد است و در شهر لیتوانی در کشور روسیه قرار دارد.} مشغول بود؛ ساکنین شهر به دنبال مسیری بودند که از تمام پل‌ها تنها یک بار عبور کنند. اویلر با استفاده از گراف زیر ثابت می‌کند که این کار غیر ممکن است \cite{network,west}:
\begin{figure}[htbp]
\centering
\includegraphics[width=9cm , height=4cm]{pol.png} 
\caption[گراف مربوط به پل گونیگسبرگ] {\footnotesize گراف مربوط به پل گونیگسبرگ (راست) و طرحی از شهر کونیگسبرگ (چپ) \cite{network}.}
\label{fig:pol}
\end{figure}\\

تا قبل از دهه  $1950$ مطالعات بر روی گراف بیشتر در زمینه گراف‌های منظم\LTRfootnote{regular} بوده است. اما بعد از آن با مطالعه شبکه‌های واقعی، دیدگاه بشر به سمت گراف‌های نامنظم سوق پیدا کرده است. اردوش و رنی\LTRfootnote{Erdös–Rényi}  اولین بار در دهه $1950$ مدل گراف تصادفی\LTRfootnote{random} را مطالعه کردند. در گراف تصادفی، هر دو راس  به طور تصادفی  با احتمالی یکسان با یک یال به یکدیگر متصل می‌شوند.  مدت‌ها بعد از تئوری اردوش و رنی، واتس و استروگتز\LTRfootnote{Watts and Strogatz}مدل خود را در قالب شبکه‌ی جهان کوچک\LTRfootnote{small World} \cite{watts} و   باراباسی و آلبرت\LTRfootnote{Barabási and Albert}  در قالب شبکه‌ی بی‌مقیاس \LTRfootnote{scale free} \cite{bara} رفتار شبکه‌ واقعی را بررسی کردند.

همانطور که گفتیم، شبکه‌ها از دیرباز شاخه‌ای از علم گراف محسوب می‌شدند و همچنین به این دلیل که شبکه‌ها قابل نمایش با گراف هستند، در ابتدا نماد‌ها و تعاریف اولیه در گراف را مورد بررسی قرار می‌دهیم و سپس مدل‌هایی از شبکه‌های پیچیده را معرفی می‌کنیم \cite{new}.
\section{ماتریس مجاورت}
در تعریف گراف، برای نمایش عناصر آن چندین روش وجود دارد\RTLfootnote{از روش‌های دیگر می‌توان به لیست مجاورت و لیست مجاورت معکوس اشاره کرد.}. یکی از این روش‌ها نمایش ماتریس مجاورت\LTRfootnote{adjacency matrix} است. این ماتریس راس‌هایی را که با هم در ارتباط هستند مشخص می‌کند. برای مثال شبکه‌ای را در نظر می‌گیریم که راس‌های آن از $1$ تا $\textsc{n}$ شماره‌گذاری شده است و تعدادی از راس‌ها نیز با‌ هم در ارتباط می‌باشند. ماتریس مجاورت $\textsc{A}$ برای گراف ساده غیرجهتی\LTRfootnote{undirected grapf} با عناصر $\textsc{A}_{\textsc{ij}}$ چنین تعریف می‌شود که اگر بین دو راس یالی وجود داشته باشد $\textsc{A}_{\textsc{ij}} = 1$ و در غیر این‌صورت 
$\textsc{A}_{\textsc{ij}} = 0$ خواهد بود. ماتریس مجاورت برای یک گراف غیرجهتی یک ماتریس متقارن خواهد بود. یعنی اگر از راس $\textsc{i}$ به راس $\textsc{j}$ یالی وجود داشته باشد حتما از $\textsc{j}$ به$\textsc{i}$ نیز وجود خواهد داشت. %همچنین به دلیل نبودِ  حلقه در شبکه‌های غیر جهتی درایه‌های قطر اصلی مقادیر صفر دارند. 
\subsection{شبکه جهتی}
شبکه یا گراف جهتی\LTRfootnote{directed graph} که به اختصار گراف ‌جهت‌دار (digraph) نامیده می‌شود گرافی‌ است که هر یالش با یک مسیر راس‌ها را با یک فلش به یکدیگر متصل می‌کند. مثال‌هایی از این نوع سامانه‌ها شبکه جهانی وب\LTRfootnote{World Wide Web} می‌باشد که در آن لینک‌ها در یک جهت از یک صفحه به صفحه دیگر به اجرا در می‌آیند و یا نمونه دیگر شبکه‌های نورونی هستند که انتقال اطلاعات از یک نورون به نورون دیگر توسط سیناپس‌ها صورت می‌گیرد. ماتریس مجاورت برای گراف جهتی این‌گونه تعریف می‌شود که اگر از $i$ به $j$ مسیری وجود داشته باشد 
$\textsc{A}_{\textsc{ij}} = 1$  و در غیر این‌‌صورت $\textsc{A}_{\textsc{ij}} = 0$ . باید توجه کنیم که تفاوت اساسی میان گراف جهتی و غیر جهتی  آن است که ماتریس مجاورت شبکه غیر جهتی متقارن نیست.  یعنی اگر بین دو راس $\textsc{i}$ و $\textsc{j}$ یالی وجود داشته باشد لزوما معکوس آن برقرار نمی‌باشد. % همچنین درایه‌های روی قطر اصلی در این شبکه‌ها برابر با یک است. 
\begin{figure}[htbp]
\hspace*{0cm}
\centering
%\begin{minipage}[b]{0.4\textwidth}
\includegraphics[width=0.3\linewidth, height=45mm]{directed.png}\centering(الف)
\includegraphics[width=0.3\linewidth, height=45mm]{undirected.png}\centering(ب)
\caption[تقسیم‌بندی گراف جهتی و غیر جهتی] {\footnotesize دو گراف: (الف) گراف غیرجهتی  و (ب) گراف جهتی برای چهار راس \cite{network}.}
\end{figure}
\subsection{شبکه وزن‌دار}
بیشتر شبکه‌هایی که مورد مطالعه قرار می‌گیرند بین رئوسشان دو حالت وجود دارد: یا اتصالی  بین آنها برقرار است  و یا برقرار نیست. اما در برخی موارد  بسته به نوع کاربرد‌هایی که شبکه می‌تواند داشته باشد وزنی را به شبکه نسبت می‌دهیم. شبکه وزن‌دار\LTRfootnote{weighted} یعنی اینکه درجه راس‌ها از توزیع خاصی پیروی می‌کنند. بنابراین اگر بخواهیم به شبکه‌های کامپیوتری وزنی را نسبت دهیم تعداد داده‌هایی که در طول شبکه وجود دارد و پهنای باند شبکه وزن آن محسوب می‌شود. وزن‌ها در شبکه‌های وزنی اصولأ مثبت در نظر گرفته می‌شوند. اما در برخی مواقع می‌توان وزن منفی نیز به شبکه نسبت داد. به عنوان مثال در یک شبکه اجتماعی افرادی که با هم رابطه دوستی دارند مثبت و آنهایی که با هم دشمنی دارند منفی در نظر می‌گیرند. مثالی دیگر از شبکه وزنی، شبکه نورونی است که در آن نورون‌ها نمایشگر گره‌ها و سیناپس‌ها یال‌های شبکه را شکل می‌دهند و وزن شبکه معرف قدرت سیناپسی  شبکه است. ماتریس مجاورت برای  چنین شبکه‌هایی کسری است. 
 
 \subsection{شبکه درختی}
  از نمونه‌های گراف غیرجهتی می‌توان به شبکه‌های درختی\LTRfootnote{tree } اشاره کرد. شبکه درختی، شبکه غیرجهتی متصل به هم هستند که هیچ حلقه‌ای در آن وجود ندارد\normalfootnotes\footnote{در واقع می‌توانیم به یال‌ها در شبکه درختی جهت نیز نسبت دهیم که در این صورت یک گراف جهتی خواهیم داشت. اما همان‌طور که تعریف کرده‌ایم این شبکه، یک شبکه بدون حلقه درنظر گرفته می‌شود و بنابراین از جهتی بودن آن صرف نظر می‌شود.}. اتصال در این شبکه به این معناست که هر راس در شبکه درختی از طریق برخی مسیر‌ها در شبکه به راس‌های دیگر دسترسی پیدا می‌کند. چنین شبکه‌هایی می‌توانند دارای دو یا چند قسمت مجزا باشند. اگر یک بخش منحصر به فرد در شبکه دارای حلقه نباشد ساختار آن درختی خواهد بود. از  جمله ویژگی‌های شبکه درختی این است که برای   $\textsc{n}$ راس در ساختار دقیقا   $\textsc{n - 1}$ یال وجود دارد. عکس این قضیه نیز درست است. یعنی اگر شبکه‌‌ای داشته باشیم که   $\textsc{n}$ راس و $\textsc{n - 1}$ یال داشته باشد شبکه درختی خواهد بود.

\subsection{طول کوتاه‌ترین مسیر و خوشگی}
مسیر\LTRfootnote{path} در گراف زنجیری است که دو راس را به هم وصل می‌کند. طول کوتاه‌ترین مسیر\LTRfootnote{shortest path length}  در گراف، کوتاه‌ترین فاصله میان دو گره است. در گراف فاصله میان دو گره را با متوسط طول کوتا‌ترین مسیر\LTRfootnote{average shortest path length}  پیدا می‌کنند. این کمیت با میانگین گیری بر روی تمام کوتاه‌ترین مسیر بین هر دو جفت گره به دست می‌آید.

 
ضریب خوشگی\LTRfootnote{clustering coefficient} یکی از ویژگی‌های آماری شبکه است. ویژگی خوشه شدن در شبکه به این معناست که راس‌های شبکه تمایل به تشکیل خوشه‌های جدا از هم دارند. مشابه این وضعیت در شبکه اجتماعی همان تشکیل گروه‌های متفاوت در جامعه است که بر اساس آن مشخص می‌شود چقدر احتمال دارد که دوستان یک فرد در شبکه با یکدیگر دوست باشند. ضریب خوشگی یک راس برابر است با تعداد مثلث‌های گراف تقسیم بر تعداد کل مثلث‌هایی که می‌تواند در گراف وجود داشته باشد. منظور از مثلث یعنی اینکه اگر بین دو راس $i$ و $\textsc{j}$ و نیز راس‌های $\textsc{j}$ و $\textsc{k}$ مسیری وجود داشته باشد بین دو راس $\textsc{i}$ و $\textsc{k}$ نیز یالی برقرار باشد.
\subsection{تابع توزیع درجه}
یکی از ویژگی‌های  متمایز کننده شبکه‌ها از یکدیگر توزیع درجه‌\LTRfootnote{degree distribution} راس‌ها می‌باشد که آن را با $P(k)$ نمایش می‌دهند. $P(k)$ نشان دهنده کسری از راس‌های گراف است که درجه آنها $k$ است. تابع توزیع درجه برای شبکه‌های مختلف متفاوت است. برای مثال این توزیع برای شبکه بی‌مقیاس به شکل توانی\LTRfootnote{power low}، برای شبکه‌های تصادفی به شکل پواسونی\LTRfootnote{poisson} و یا گاوسی\LTRfootnote{gaussian}.
لازم به  یادآوری است که در تعریف درجه یک راس تنها تعداد یال‌های خارج شده و یا وارد شده به آن راس شمرده می‌شود. اما اینکه آن یال از کدام راس به آن رسیده است در نظر گرفته نمی‌شود. 
\section{مدل‌های شبکه}  
دستیابی به اطلاعات یک سیستم پیچیده واقعی کاری بس مشکل و چه بسا ناممکن است. چرا که  اطلاعات دقیقی از سیستم واقعی در اختیار نداریم و نیز کامپیوتر‌ها قدرت محاسبه چنین اطلاعات سنگینی را ندارند. یکی از این مشکلات در شبکه‌های عصبی دیده می‌شود که برای بررسی سیستم اطلاعات دقیقی از نحوه کارکرد و نیز تعداد سیناپس‌ها و نورون‌ها در اختیار نداریم. برای رفع این مشکل از مدل‌های شبکه بهره می‌گیرند که به شبکه‌های دنیای واقعی نزدیک است.  ویژگی‌های مشترک در شبکه‌های دنیای واقعی طول مشخصه مسیر و ضریب خوشگی  در شبکه‌هاست. اکثر شبکه‌هایی که در دنیای واقعی دیده می‌شوند دارای طول مسیر کوچک و ضریب خوشه شدن بالا هستند. بر اساس این ویژگی‌های ساختاری، می‌توان شبکه‌ها را دسته‌بندی‌ کنیم. از جمله این شبکه‌ها می‌توان   شبکه منظم\LTRfootnote{regular}، جهان کوچک، شبکه تصادفی و بی‌مقیاس را نام برد که در ادامه به بررسی هر یک از آنها می‌پردازیم.
\subsection{شبکه منظم}
در نظریه گراف، یک گراف منظم شبکه‌ای است که هر راس آن همسایه‌های یکسان داشته باشد. به عبارت دیگر همه راس‌ها دارای توزیع درجه یکسان باشند. گراف منظمی که درجه همه راس‌های آن $k$ باشد، گراف منظم-$k$\LTRfootnote{k-regular graph} نامیده می‌شود. 
مثال‌هایی  گراف منظم رادر زیر آمده است:
\begin{itemize}
\item گراف منظم-$0$ (گرافی که درجه همه راس‌ها صفر است.)
\item گراف منظم-$1$ (گرافی که درجه راس‌ها  $1$ است و هر دو راس تنها با یک یال به هم اتصال دارند. )
\item  گراف منظم-$2$(گرافی که درجه همه راس‌ها $2$ باشد. در این نوع هر $3$ راس یک چرخه جدا از هم را تشکیل می‌دهند.)
\item گراف منظم-$3$(به شکل یک گراف مکعبی در نظر گرفته می‌شود.)
\end{itemize}
\begin{figure} [htbp]
\centering
\includegraphics[width=12cm , height=4.cm]{regular.png}
\caption[نمونه‌هایی از گراف منظم] {\footnotesize نمونه‌هایی از گراف منظم: از چپ به راست(گراف منظم-0، 1، 2 و 3)}\cite{network}
\label{fig:regular}
\end{figure}\\\\
توجه کنیم که در شبکه منظم برای گراف‌هایی که 
$k  \geq 2$ باشد خوشگی داریم.
\subsection{شبکه جهان کوچک}
در اواخر سده $1960$ میلگرام\LTRfootnote{Milgram} طی آزمایشی معروف افرادی را به طور تصادفی انتخاب کرد و برای آنها نامه‌هایی را فرستاد. وی از آن افراد خواست تا نامه‌ها را به یک فرد که از پیش تعیین شده است بفرستند و در ضمن شرط فرستادن نامه نیز این بود که باید آن را به یک دوست صمیمی بفرستند. در نتیجه افراد به دنبال دوستانی می‌گشتند که به فرد مورد نظر نزدیک باشد. این روند تا جایی تکرار می‌شد که نامه به شخص مورد نظر برسد. زمانی که حدود $20$ درصد از نامه‌های میلگرام به مقصد رسید او متوجه شد  مسیری که هر نامه طی کرده است به طور متوسط دارای طول $6$ بود؛ یا به بیان دیگر فقط $6$ نفر بین هر دو نفر قرار داشتند. نتیجه آزمایش این بود هردو شخصی که بر روی زمین زندگی می‌کنند نیز به تعداد این تکرار‌ها با یکدیگر رابطه برقرار می‌کنند \cite{milgram,traver}. براساس آزمایش میلگرام این پدیده جهان کوچک نام گرفت. 
این نظریه  می گوید هر دو انسان ساکن بر روی کره زمین، به طور میانگین در یک رابطه با $6$ واسطه یا کمتر به هم مربوط می‌شوند.
ش %بکه جهان کوچک این نتیجه را می‌دهد که فاصله میانگین بین دو گره در دو شبکه مختلف می‌تواند نزدیک‌تر از فاصله گره‌هایی باشد که همان شبکه را می‌سازند. 

شبکه جهان کوچک دارای دو ویژگی است:
\begin{enumerate}
\item طول مسیر کوتاه
\item ضریب خوشگی بالا
\end{enumerate}
شبکه جهان کوچک از مدل واتس-استروگتز پیروی می‌کند \cite{albert,watts}. مدل پیشنهادی واتس و استروگتز برای جهان کوچک مدلی است که ساختارش بین ساختار گراف منظم و گراف تصادفی است. این مدل بر پایه‌ بازآرایی\LTRfootnote{rewiring} یال‌ها با احتمال $p$ انجام می‌شود. آنها در ابتدا یک گراف منظم که کامل نیست با $N$ راس در نظر می‌گرفتند. سپس هر یال را با احتمال $p$ بازآرایی می‌کردند. یعنی با احتمال $p$ یک یال را از یک راس جدا کرده و به راس دیگر وصل کردند. این عمل برای تک تک راس‌ها تکرار شد \cite{watts,boc,albert}. در این مدل با انتخاب $p = 0$ گراف به یک شبکه منظم و با احتمال $p = 1$ به گراف تصادفی می‌رسیم و برای مقادیر میانی کوچک $p$ شبکه به یک گراف جهان کوچک تبدیل می‌شود.
\begin{figure} [htbp]
\centering
\includegraphics[width=9.5cm , height=3.cm]{small.png}
\caption[شبکه جهان کوچک با مدل پیشنهادی واتس-استروگتز] {\footnotesize شبکه جهان کوچک با مدل پیشنهادی واتس-استروگتز. در این مدل بازآرایی    یال‌ها را مشاهده می‌کنیم \cite{watts}.}
\label{fig:small}
\end{figure}
\subsection{شبکه تصادفی}
اولین مدلی که برای توصیف شبکه‌های واقعی ارائه شد مدل گراف تصادفی  بود که در سال $1959$ توسط اردوش و رنی پایه‌گذاری شد \cite{erdos}. در این مدل تعدادی گره در نظر می‌گیرند و سپس هر یک از گره‌ها را با احتمال ثابتی به گره‌های دیگر وصل می‌کنند. به زبان دیگر، برای ساختن چنین گرافی یک پارامتر احتمال $p$ تعریف می‌کنند. دو راس با احتمال $p$ به هم با یک یال متصل می‌شوند. 
توزیع درجه‌ی $P_{k}$ برای گراف تصادفی  به صورت زیر از توزیع دو جمله‌ای پیروی می‌کند،
\begin{equation}
P_{k} = \binom{N-1}{k}p^{k}(1-p)^{N-1-k}.
\end{equation}
 ‌این توزیع درجه در حد $N$های بزرگ به تابع توزیع پواسونی میل می‌کند،
\begin{equation}
P_{k} = e^{-c}~~\frac{c^{k}}{k!}. 
\end{equation}
که $k$ تعداد راس ها با درجه $k$ و  $c$ میانگین درجه راس های شبکه است \cite{new}.

مدل گراف تصادفی ویژگی گراف جهان کوچک را دارد و اندازه کوتاه‌ترین مسیر بین دو گره با لگاریتم تعداد گره‌های گراف متناسب است. در مقابل این گراف ضریب خوشگی کوچکی دارد.
\subsection{شبکه بی‌مقیاس}
شبکه بی‌مقیاس به شبکه‌ای گفته می‌شود که توزیع درجات راس‌ها از قانون توانی  به صورت زیر پیروی کند \cite{bar}،
\begin{equation}
P_{k} = k^{-\gamma}. 
\end{equation}
این توزیع نشان‌دهنده احتمال اتصال گره جدید به گره $i$ در شبکه مورد نظر است و متناسب با درجه هر راس تعریف می‌شود. 
همانطور که دیدیم مدل‌های قبلی که برای شبکه‌ها در نظر گرفتیم دارای توزیع درجه پواسونی بودند. به همین دلیل در سال $1998$ آلبرت و باراباسی مدل جدیدی را با در نظر گرفتن توزیع توانی شبکه‌های واقعی ارائه دادند\cite{bara,barab}.  در مدل آن‌ها نمای $\gamma$   برابر با $3$  در نظر گرفته شد. مدل باراباسی و آلبرت از دو مرحله رشد\LTRfootnote{growth} و مرحله پیوند ترجیحی\LTRfootnote{preferential attachment} تشکیل شده است. 

طبق این روش، در مرحله رشد شبکه‌ای کاملا به هم پیوسته با $m_{0}$  گره را در نظر گرفتند. سپس در هر مرحله یک گره جدید با $m$ یال$(m < m_{0})$ متصل به گره قدیمی اضافه کردند.
در مرحله پیوند ترجیحی هر یال جدید به یک گره قدیمی با احتمالی متناسب با درجه آن متصل می‌شود. احتمالی که راس جدید به راس قدیم متصل شود به شکل زیر تعریف می‌شود،

\begin{equation}
P(i) = \dfrac{k_{i}}{\sum_{i}{k_{i}}}.
\end{equation}
که در آن  $k_{i}$  درجه راس $i$ام است. با توجه با این رابطه هرچه احتمال برای یک راس بیشتر باشد، احتمال اتصال راس جدید به آن راس بیشتر می‌شود. % از مشخصه‌های مدل آلبرت و باراباسی این است که با افزایش درجه راس‌ها، ضریب خوشگی کاهش می‌یابد. 
مدل بی‌مقیاس نمونه‌ای از "ضرب‌المثل غنی-غنی‌تر میگردد‌\LTRfootnote{rich-get-richer}" است.
\begin{figure} [htbp]
\centering
\includegraphics[width=9.5cm , height=3.cm]{scale.png}
\caption[نمایش مدل پیوند رشد برای یک شبکه بی‌مقیاس] {\footnotesize نمایش مدل رشد برای یک شبکه بی‌مقیاس. در مرحله $t = 0$ سیستم شامل $m_{0} = 3$ راس مجزا است. در هر گام بعدی یک راس جدید(دایره تیره) به سیستم اضافه می‌شود و با $m = 2$ راس قبلی ارتباط برقرار می‌کند  \cite{watts}.}
\label{fig:scale}
\end{figure}\\\\
بعد از آنکه آلبرت و باراباسی مدل خود را ارائه دادند، مدل‌های دیگری برای شبکه‌های واقعی مطرح شد که حالت تعمیم یافته مدل باراباسی بوده است.  با بررسی داده‌های واقعی از بسیاری از شبکه‌های پیچیده یافت شد که نمای $\gamma$ برای این شبکه‌‌ها بین $2$ و $3$ متغیر است. از نمونه‌ شبکه‌های بی‌مقیاس که در شبکه‌های واقعی دیده می‌شود می‌توان  به شبکه جهانی وب، شبکه‌های بیولوژیکی و اجتماعی اشاره کرد.

\subsection{شبکه نورونی}
تا کنون انواع مدل‌های شبکه‌ را مورد بررسی قرار داده‌ایم. در اینجا مثالی از شبکه‌ پیچیده را معرفی می‌کنیم و ویژگی‌های آن را با شبکه‌هایی که تاکنون مطالعه کرده‌ایم بررسی می‌کنیم. مجموعه‌ای از نورون‌ها و اتصال‌های سیناپسی بین ‌آنها شبکه نورونی را به وجود می‌آورند.

شبکه‌های نورونی به دو شکل تعریف و بررسی می‌شوند:
 
نوع اول شبکه ساختاری\LTRfootnote{structural network} است که یال‌های بین گره‌ها (سیناپس‌های بین نورون‌ها) را بر اساس ارتباط فیزیکی در نظر می‌گیرند. این ارتباط می‌تواند در مقیاس کوچک بین تک تک نورون‌ها و یا در مقیاس بزرگ بین مجموعه‌ای از نورون‌ها در قسمت‌های مختلف سیستم باشد. 

نوع دوم شبکه کارکردی\LTRfootnote{functional network} مغز است که برای اتصال بین نورون‌ها باید شبکه نورونی بین دو ناحیه فعالیت هم‌زمان داشته باشد. به عبارت دیگر، مستقل از مکان نورون‌ها در صورتی که بین دو ناحیه که در آن نواحی تعداد نورون‌ها زیاد است  رفتار مشابهی دیده شود می‌توان بین نورون‌های آن نواحی اتصالی در نظر گرفت.

شبکه ساختاری  نورونی دارای ارتباط‌های وسیع و پیچیده‌ای است. با در نظر گرفتن مدل‌های شبکه‌ای، می‌توان به درک درستی از شبکه‌های پیچیده از جمله شبکه مغز پیدا کرد. ساختار شبکه‌های نورونی در مقیاس‌های کوچک و بزرگ متفاوت است و نشان داده شده است که این شبکه‌ها خواص شبکه بی‌مقیاس و جهان کوچک را دارا می‌باشند \cite{soode. از طرف دیگر در این شبکه، طول مسیر کوتاه به چشم می‌خورد. به عبارت دیگر هر نورون به واسطه ارتباط‌های سیناپسی، با نورون‌های همسایه‌اش  با احتمال بیشتری نسبت به نورون‌های دورتر ارتباط برقرار می‌کند\cite{gilani}.
 
نورون‌ها می‌توانند به‌ صورت الکتریکی برانگیخته شوند. همانطور که در فصل اول شرح دادیم، سلول در حال استراحت هیج‌ گونه فعالیتی ندارد. نورون‌‌ها آستانه‌ی مشخصی  برای فعال شدن دارند. زمانی که پتانسیل غشا از این حد آستانه بیشتر شود نورون آتش می‌کند. این عمل به نورون‌‌های بعدی نیز منتقل می‌شود و بنابراین باعث تحریک نورون‌های مجاور  و در نتیجه فعال شدن آنها می‌شود. در فصل آینده خواهیم گفت که فعال شدن دسته جمعی در نورون‌ها چه رویکرد‌هایی را به همراه دارد.

 شبکه نورونی مثالی از یک شبکه واقعی بیولوژیکی  است که ساختار آن  و نیز آستانه‌ نورون‌ها از توزیع درجه تصادفی تبعیت می‌کند.  اتصال بین نورون‌ها یکسان نبوده و هر نورون ورودی و خروجی متفاوتی را دریافت می‌کنند.  اما  در مجموع اتصال  بین نورون‌ها حول مقداری متوسط است. از این رو توزیع درجه‌ مناسب برای این شبکه را گاوسی در نظر گرفتند. از طرفی به دلیل همسان نبودن نورون‌ها برای دریافت آستانه فعالیت برای آتش کردن، توزیع آستانه‌ای  که برای شبکه نورونی در نظر  گرفتند گاوسی است. در فصل بعد فعالیت شبکه نورونی را به طور مفصل مورد بررسی قرار می‌دهیم. 
\newpage 
\textbf{خلاصه‌ی فصل دوم}    
\begin{itemize}
\item  ویژگی اصلی شبکه‌های پیچیده وجود عناصر بیشمار در ساختار آن و نیز برهم‌کنش میان این اجزا می‌باشد. 
\item  برای درک بهتر شبکه‌های پیچیده باید رفتار دسته‌جمعی اجزای آن را مورد بررسی قرار داد. 
\item مطالعه گراف اولین بار توسط اویلر بر روی گراف‌های منظم انجام شده است و در پی آن برای بررسی شبکه‌های واقعی، مدل‌های تصادفی شبکه‌ها نیز مورد توجه قرار‌گرفته است.
\item  از مشخصه‌های شبکه‌های واقعی طول مشخصه و ضریب خوشگی است. طول مشخصه فاصله میانگین میان دو راس و ضریب خوشگی تعداد دور‌های به طول $3$  را در گراف مشخص می‌کند.
\item گراف منظم دارای طول مسیر کوتاه و ضریب خوشگی بالا از مرتبه $N^\frac{1}{d} $ است که $d$ بعد شبکه را نشان می‌دهد.
\item طول مشخصه برای شبکه‌های جهان کوچک و تصادفی از $O(\log N)$ است. در عوض ضریب خوشگی در شبکه جهان کوچک بزرگ‌تر از شبکه تصادفی است.
\item ضریب خوشگی شبکه بی‌مقیاس بزرگ‌تر از شبکه تصادفی و کوچک‌تر از شبکه بی‌مقیاس است و طول شخصه این شبکه نیز از $O(\log(\log N))$ می‌باشد. 
\item شبکه‌های نورونی به علت همسان نبودن در دریافت ورودی و نیز آستانه‌ فعالیت شکل ساختاری و نیز آستانه فعالیتشان به شکل گاوسی در نظر گرفته می‌شود.
\end{itemize}
 





 
 
 



