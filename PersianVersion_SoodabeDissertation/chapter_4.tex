\chapter{تراوش خود‌راه‌انداز در  گراف‌های تصادفی}
شبکه نورونی از جمله شبکه‌های پیچیده‌ای است که برای به دست آوردن بزرگترین خوشه به هم پیوسته از نورون‌های فعال  از تئوری تراوش خود‌راه‌انداز استفاده می‌شود.
در فصل گذشته با بررسی کار‌های تجربیِ  مدل تراوش  روی بافت‌های نورونی مشاهده کردیم که مسدود کردن و یا فعال کردن سیناپس‌ها چه تاثیری بر خوشه‌ی فعال در شبکه می‌گذارد. همچنین تاثیر آستانه‌ی فعالیت را روی شبکه و اندازه خوشه مشاهده کردیم. در این فصل  تراوش خودراه‌انداز را ابتدا روی  گراف ER و سپس  ساختار گاوسی  مربوط به شبکه نورونی بررسی خواهیم کرد. همچنین آستانه یکنواخت و نیز آستانه با توزیع گاوسی را بررسی خواهیم کرد. 

\section{شبکه ER با میانگین درجه رئوس یکنواخت}
شبکه تصادفی ER را با N راس در نظر می‌گیریم. با انتخاب یک  احتمال اولیه p به طور تصادفی بین هر دو جفت راس اتصالی برقرار می‌کنیم. برای هر راس میانگین درجه نیز در نظر می‌گیریم که ساختار شبکه با این توزیع توصیف می‌شود. به خاطر داریم که در حد N‌های بزرگ  برای شبکه‌‌ای با میانگین درجه همگن (پوآسونی) احتمال  اولیه با میانگین درجه اتصال میان راس‌ها ارتباط مستقیم دارد.
با این شرایط ساختمان شبکه را می‌سازیم. برای نمایش اتصال بین راس‌ها نیز از ماتریس مجاورت استفاده می‌کنیم.
 هرچه احتمال اولیه بزرگتر باشد، تعداد راس‌هایی که به هم متصل هستند نیز بیشتر خواهد بود.
بعد از ساخت شبکه تصادفی دینامیک را روی شبکه اثر می‌دهیم. این دینامیک مربوط به فعال و یا غیر فعال بودن راس‌ها در شبکه است. نورون‌ها در شبکه دو حالت را می‌پذیرند. برای نمایش حالت‌ها، بردار حالت Sرا به شکل زیر تعریف می‌کنیم،
\begin{equation}
{S_{i} = 0 , 1}
\end{equation}

زمانی که نورون غیرفعال باشد S_{i} = 0  و زمانی که فعال باشد .S_{i} = 1 

می‌دانیم که هر نورون برای آتش کردن نیازمند ولتاژی  است  تا بتواند آن را به آستانه  آتش کردنش برساند.  آستانه آتش کردن هر نورون بسته به نوع نورون‌ها (به عنوان مثال مهاری و یا تحریکی بودن آن‌ها) متفاوت خواهد بود. اگر ولتاژ تحریکی را با v_{i} و آستانه‌ی فعالیت هر نورون را باf نشان دهیم، می‌توان گفت که برای رسیدن به آستانه و در نهایت آتش کردن نورون باید مقدار ولتاژ تحریکیv_{i} بزرگ‌تر از مقدار آستانه‌ی فعالیتfباشد. i نماینده نورون iام است. همچنین در یک تعریف دیگر،fرا احتمال اولیه فعال شدن نورون‌ها در نظر می‌گیریم که متناظر با کسری از نورون‌های فعال اولیه در شبکه است. با این احتمال تعدادی از مکان‌ها فعال می‌شوند و این تعداد تا زمانی‌ که شبکه به حالت پایا برسد فعال باقی ‌می‌مانند. حالت پایا زمانی‌ است که دیگر هیچ نورونی قابلیت فعال شدن را ندارد  و بعد از آن شبکه فعالیتی نمی‌کند و در همان حالت خود باقی می‌ماند. بعد از آنکه با احتمال اولیه مجموعه‌ای از راس‌ها فعال شدند، کسری از راس‌های غیر فعال در شبکه باقی می‌مانند که می‌خواهیم آنها را طی قاعده‌ای فعال کنیم. برای این راس(نورون)‌ها نیز آستانه‌ی فعالیتی(m) جدا از آستانه‌ی فعالیت نورون‌های اولیه تعریف می‌کنیم. بدین معنی که هر راس(نورون)‌ غیر فعال در اطراف خود باید حداقل به اندازه m راس(نورون) فعال داشته باشد تا بتواند ورودی لازم برای آتش کردن را از همسایه‌های فعال خود دریافت کند و فعال شود. این عمل برای همه مکان‌ها در گام‌های مختلف تکرار می‌شود و در هر گام تعداد مکان فعال تولید شده  را به مکان‌ گام‌های قبل اضافه می‌کنیم و آن‌ها را در بردار حالت S با حالت 1 نشان می‌دهیم.

در یک شکل نمادین می‌توانیم بنویسیم: 
\begin{equation}
\text{if}~~~~~v_{i}\geq f \longrightarrow S_{i} = 1
\end{equation}


و می‌توانیم به این شکل بگوییم که اگر احتمال اولیه بزرگتر از آستانه‌ی فعالیت باشد حالت آن راس را در بردار حالت 1 در نظر می‌گیریم.
بعد از این مرحله نوبت به فعال کردن راس‌ها توسط همسایه‌های فعال است. یک راس غیر فعال را انتخاب می‌کنیم. ابتدا بررسی می‌کنیم که آن راس با چه راس‌های دیگر در ارتباط است. با نظر به آنها و شمارش راس‌هایی که از میان آنها فعال هستند، در صورتی که تعدادشان حداقل به اندازه تعداد آستانه‌ی فعالیت شبکه باشند راس غیر فعال نیز فعال می‌شود. بنابراین می‌توان نشان داد:  
\begin{equation}
\text{if}~~~~~m: \sum_{i=1}^{n} A_{ij}S_{j}\geq m \longrightarrow S_{i} = 1
\end{equation}
در اینجا می‌توانیم رابطه‌ای را میان احتکمال مکان‌های فعال اولیه و احتمال‌ فعال شدن هر مکان به شکل زیر بیان کنیم:
\begin{equation} 
\phi(f) = f + (1-f)\psi_{m}(\phi) \label{eq1}
\end{equation}
در این رابطهfاحتمال مکان‌های اولیه فعال،1-f احتمال وجود نورون‌های غیر فعال شبکه هستند. \psi_{m}(\phi) احتمالی است که نورون‌های غیر فعال با داشتن حداقل m همسایه فعال، فعال می‌شوند و \phi نیز احتمال کل فعال شدن یک نورون را نشان می‌دهد. در واقع این رابطه فعال شدن شبکه ابتدا با یک تحریک خارجی اولیهfو سپس پخش شدن با اثر همسایه‌هایش را برایمان روشن می‌سازد. رابطه (\ref{eq1}) گویای این است که\phi(f) در fهای کوچک  به طور خطی باfرشد می‌کند و مقدارش بین  \phi(0) = 0 و \phi(1) = 1 متغیر است.
 به زبان دیگر، زمانی که احتمال اولیهfصفر است هیچ راسی در شبکه فعال نیست و امکان فعال شدن برای راس‌های دیگر نیز وجود ندارد. بنابراین \phi(0) = 0. در مقابل اگر احتمال اولیه بیشترین مقدار خود را داشته باشد (f = 1)، همه راس‌ها با همان احتمال اولیه فعال می‌شوند. از این رو کل مکان‌های شبکه فعال هستند و \phi(1) = 1. از طرفی، تا زمانی که شبکه به طور جداگانه رفتار کند و ارتباطی بین راس‌ها وجود ندشته باشد، با رشدfهمان تعداد راسی که از ابتدا فعال شدند در شبکه به عنوان  راس‌های فعال نهایی باقی می‌مانند. چرا که نمی‌توانند شرط فعال شدن برای دیگر  راس‌ها را مهیا کنند. بنابراین 
 \phi به طور خطی باfزیاد می‌شود (\phi(f) = f). اما زمانی‌‌که به یک مقدار خاص ازfبرسیم که در آن نقطه گذاری در شبکه حاصل شود برایمان قابل قبول خواهد بود که وابستگی بهfاز بین می‌رود. به عبارت دیگر، در این زمان فعال‌سازی شبکه وابسته به همسایه‌های فعال می‌شود و این یعنی این که اتصال بین راس‌ها وابسته به دینامیک است. این نکته قابل ذکر است نقطه‌ای که در آن  از حالت خطی خارج می‌شویم نقطه گذار شبکه است.

با اعمال  آستانه‌ی فعالیت یکنواخت و گاوسی به شبکه ER احتمال تعداد کل مکان‌های فعال و نیز اندازه بزرگ‌ترین خوشه فعال به هم پیوسته را پیدا می‌کنیم.
\subsection{آستانه‌ی فعالیت یکنواخت}
زمانی که نتایج تجربی را بررسی می‌کردیم به این نکته پی بردیم افزایش تعداد مکان‌های اولیه و  نیز آستانه‌ی فعالیت، اندازه خوشه به هم پیوسته از مکان‌های فعال را کاهش می‌دهد. در اینجا می‌خواهیم ببینیم که آیا نتایج شبیه‌سازی با نتایج تجربی هم‌خوانی دارد؟

این‌گونه می‌توانیم ادامه دهیم؛ شبکه‌ای تصادفی از N راس فعال و غیر فعال داریم که با احتمال p به هم وصل شده‌اند. تعداد راس‌های اولیه فعال که با تحریک خارجیfفعال شده‌اند درصدی از تعداد کل راس‌های شبکه است. راس‌های غیر فعال می‌توانند از طریق دینامیک حاکم بر شبکه فعال شوند. دینامیک فعال‌سازی این راس‌ها به این صورت است که راس‌های غیر فعال از طریق فعالیت همسایه‌های فعال خود در صورتی که بتوانند بر آستانه‌ی فعالیت غلبه کنند فعال می‌شوند. این فرایند تا زمانی که راس‌های غیر فعال قابلیت فعال شدن داشته باشند ادامه خواهد داشت. زمانی سیستم به حالت پایا می‌رسد که دیگر دینامیک بیش‌تر از آن جلو نمی‌رود  و دیگر امکان فعال شدن راسی وجود نخواهد داشت. مراحل  فعال‌سازی  تا اتمام فرایند گام به گام تکرار می‌شود. در انتهای فرایند تعداد کل راس‌های فعال شمرده می‌شود. این تعداد  هم شامل راس‌هایی است که با تحریک اولیه فعال شدند و هم آنهایی که در طول اجرای دینامیک توانستند فعال شوند. نمودار (\ref{fig:sites}) اندازه احتمال کل مکان‌های فعال  برای آستانه‌های مختلف  را برحسب احتمال نقاط فعال اولیه نشان می‌دهد. منجنی‌های زیر برای شبکه‌ای با10000 راس و با میانگین درجه 
\lqngle k \rangle = 5  رسم شده  است.  به عبارت دیگر هر راس به طور میانگین با5 راس دیگر اتصال دارد. میانگین‌گیری روی 100
 نمونه انجام شده است. S_{a} در نمودار، زیر تعداد کل مکان‌های فعال را مشخص می‌کند.

\begin{figure} [htbp]
\centering
\includegraphics[width=11cm , height=8cm]{sites.eps} 
\caption [نمودار مربوط به احتمال کل نقاط فعال بر  حسب  تغییر احتمال اولیه برای شبکه ER]{\footnotesize نمودار مربوط به احتمال کل نقاط فعال بر  حسب  تغییر احتمال اولیه برای شبکه ER با 10000 راس و  میانگین درجه اتصال راس‌ها  \lqngle k \rangle = 5 .}
\label{fig:sites}
\end{figure}
نمودار (\ref{fig:sites}) گویای این است که با افزایش احتمال اولیه f  نقاط فعال در شبکه افزایش پیدا می‌کند. با این حال می‌بینیم با افزایش آستانه‌ی فعالیت m تعداد نقاط فعال کاهش می‌یابد. علت نیز این است که با زیاد شدن تعداد نقاط فعال، آستانه‌ی فعالیت نیز برای نورون‌ها افزایش می‌یابد و شرطی که برای فعال کردن نورون غیر فعال در نظر می‌گیریم سخت‌تر می‌شود. به عبارت دیگر یک راس غیر فعال در اطراف خود امکان پیدا کردن نقاط فعال  به اندازه آستانه‌ی فعالیت را نخواهد داشت. به همین دلیل تعداد کل نقاط فعال  شبکه به طور خطی با تعداد نقاط اولیه فعال رشد می‌کند. در نمودار (\ref{fig:sites})  آستانه  5 این رفتار را نشان می‌دهد. همچنین در این نمودار رشد ناگهانی  در احتمال اولیه مکان‌های  فعال شبکه در fهای کوچک مشاهده می‌کنیم. در واقع برای یک مقدار معین f، قبل از آن، راس فعالی در شبکه وجود ندارد یا تعدادشان به اندازه‌ای نیست که شبکه را فعال کند. اما بعد از آن،  شبکه به یک باره فعال می‌شود. همانطور که می‌بینیم با افزایش مکان‌های فعال، اندازه جهش  رفته رفته کاهش می‌یابد و در fهای بالا از بین می‌رود. به روشنی می‌توان دریافت که احتمال  اولیه f باعث تغییر در نقطه گذار می‌شود. آستانه m نیز نقطه احتمال fرا که در آن گذار اتفاق می‌افتد مشخص می‌کند.

از آنجایی که آستانه‌ی فعالیت شرط فعال شدن مکان‌‌های غیر فعال  را آسان یا سخت می‌کند، چیزی که صریحا با آن قابل تغییر است اندازه خوشه ایست که در شبکه به وجود می‌آید. نمودار زیر وجود و عدم وجود خوشه را نشان می‌دهد. این نمودار برای شبکه ER با 10000 راس و با میانگین درجه \lqngle k \rangle = 5 رسم شده است.  میانگین‌گیری روی 100 نمونه انجام شده است.
\begin{figure} [htbp]
\centering
\includegraphics[width=11cm , height=8cm]{max.eps} 
\caption [اندازه خوشه به هم پیوسته از مکان‌های فعال بر حسب احتمال اولیه مکان‌های فعال با آستانه یکنواخت  شبکه ER]{\footnotesize اندازه خوشه به هم پیوسته از مکان‌های فعال بر حسب احتمال  اولیه مکان‌های فعال با آستانه یکنواخت برای شبکه ER با تعداد راس N = 10000 و میانگین درجه \lqngle k \rangle = 5. افزایش آستانه‌ی فعالیت m و  احتمال اولیهfسبب کاهش اندازه خوشه می‌شوند.}
\label{fig:max}
\end{figure}

با توجه به نمودار (\ref{fig:sites})، می‌توان ‌روند تغییرات فعال‌سازی بر حسب آستانه‌های مختلف را استنتاج کرد. با توجه به مباحث قبل در این مورد هم تاثیر احتمال اولیهfرا مشاهده می‌کنیم. در احتمال کوچک به دلیل کم بودن تعداد مکان‌های فعال شکل گرفتن خوشه ممکن نبوده و به همین دلیل اندازه‌اش صفر است (منحنی روی محور افقی  روی صفر حرکت می‌کند). به عبارت دیگر در $f$های کوچک جزیره‌‌های کوچکی از مکان‌های فعال داریم که به هم متصل نیستند. به همین دلیل اندازه خوشه تا مقادیری ازfصفر است. اما با افزایشfو زیاد شدن تعداد مکان‌های فعال در یک نقطه خاص  منحنی رشد ناگهانی را متحمل می‌شود که نشان از فعال شدن سریع شبکه  است. نقطه‌ای که در آن شبکه یکباره فعال می‌شود گذار تراوش را نشان می‌دهد. زمانی که اندازه مکان‌های فعال از یک مقدار معینfعبور کنند  باعث به وجود آمدن جزیره‌های بزرگ‌تر می‌شوند که  اتصالی بین آن‌ها برقرار خواهد شد . همان‌طور که مشخص است قبل از  آنکه گذار تراوش  اتفاق بیفتد  خوشه‌ی  بزرگ نداریم و بعد از آن خوشه شکل می‌گیرد. شایان ذکر است که آستانه‌ی فعالیت نقش موثرتری در تشکیل شدن و نشدن خوشه به هم پیوسته ایفا می‌کند. در مطالب گذشته به این نکته اشاره شده است که افزایش آستانه فعال شدن شبکه را سخت‌تر می‌کند. چرا که با زیاد شدن مقدار آستانه، مکان‌های غیر فعال دسترسی کمتری به مکان‌های فعال اطراف خود دارند و بنابراین نمی‌توانند فعال شوند و شبکه با افزایش آستانه‌ی فعالیت به شکل جداگانه رفتار می‌کند. این عامل سبب کوچک شدن اندازه خوشه می‌شود و در آستانه‌های بالاتر همانطور که از نمودار مشخص است خوشه از بین می‌رود. رفتار خوشه نیز در آستانه‌های بالا گذار پیوسته‌ای را از خود نشان می‌دهد.

\subsection{آستانه‌ی فعالیت گاوسی}
با نقطه نظر به مباحث پیشین، می‌دانیم که در یک شبکه نورونی ‌‌همه نورون‌ها با یک آستانه‌ی فعالیت آتش نمی‌کنند. بلکه تعدادی از آنها آستانه‌ی فعالیتی پایین‌تر از حد میانگین دارند  و تعدادی نیز با آستانه‌ای بالاتر از حد میانگین فعالیت می‌کنند. بنابراین شبکه حول یک مقدار میانگین رفتار می‌کند. با در نظر گرفتن شواهد تجربی می‌توان نتیجه گرفت که رفتار شبکه از یک توزیع گاوسی برخوردار است و نورون‌ها حول میانگین این توزیع رفتار می‌کنند. با تغییر دو پارامتر تابع گاوسی، میانگین و پهنای توزیع گاوسی، رفتار شبکه را بررسی می‌کنیم. ساختار شبکه در این مورد نیز شبکه ER با میانگین درجه اتصال‌های یکنواخت انتخاب شده است. 
در شبکه‌ای که توزیع گاوسی برای آستانه وجود دارد مکان‌هایی که آستانه آنها از مقدار میانگین کمتر باشد به معنای کوچک بودن آستانه‌ی فعالیت برای آن مکان است؛ و نیز ذکر کردیم اگر آستانه‌ی فعالیت برای راسی کوچک باشد شرط فعال شدن برای راس‌های دیگر راحت‌تر خواهد بود.

 
\begin{figure}[htbp]
\hspace*{0cm}
\centering
%\begin{minipage}[b]{0.4\textwidth}
\includegraphics[width=11cm , height=8cm]{setactive.eps}\centering
%\includegraphics[width=0.4\linewidth, height=55mm]{tap.eps}\centering(a)    
\caption [منحنی مربوط‌ به احتمال  نهایی تعداد کل مکان‌های فعال  برای شبکه ER با آستانه گاوسی]{\footnotesize 
احتمال  نهایی تعداد کل مکان‌های فعال برای شبکه ER با تعداد رئوس  $10000$ بر حسب  تغییرات احتمال مکان‌های اولیهfبرای  پهناهای گاوسی  ( $\sigma$های) متفاوت با میانگین آستانه $\mu = 6$.}
\label{fig:ER}
\end{figure}

نمودار (\ref{fig:ER}) تغییرات احتمال  نهایی تعداد کل مکان‌های فعال  در شبکه بر حسب تغییرات احتمال مکان‌های فعال اولیهfرا نشان می‌دهد. $\sigma$ و $\mu$ به ترتیب پهنای گاوسی و میانگین آستانه را مشخص می‌کنند.  در این نمودار، میانگین آستانه فعالیت  ثابت در نظر گرفته شده و منحنی برای مقادیر مختلف پهنای گاوسی رسم شده است. پهنای گاوسی در واقع تعداد راس‌ها با آستانه‌های مختلف را برایمان مشخص می‌کند. با توجه به این گفته، در توجیه رفتار این شکل می‌توان گفت با افزایش پهنای $\sigma$، تعداد راس‌هایی که دارای آستانه کوچک‌تر هستند بیشتر می‌شود. در نتیجه  آستانه‌ی فعالیت مکان‌ها کاهش یافته و بنابراین شبکه در ‌$f$های کوچک‌تر رشد سریع‌تری را از خود نشان می‌دهد.  اما زمانی که $\sigma$ کوچک را برای توزیع آستانه در نظر می‌گیریم با نزدیک شدن به حالت یکنواخت راس‌های کمتری آستانه کوچک دارند و بنابراین راس‌های غیر فعال کمتری می‌توانند فعال شوند. با این استدلال می‌توان دریافت که در  $\sigma$های کوچک به علت کم بودن تعداد مکان‌های فعال  در اطراف یک راس‌ غیرفعال و برآورده نشدن شرط فعال‌سازی برای آن، خوشه به هم پیوسته از مکان‌های فعال نداریم و برای  $\sigma$های بالاتر و پهن‌تر، شاهد به  وجود آمدن خوشه هستیم. از طرفی با تغییر میانگین آستانه نقطه گذار نیز جابه‌جا می‌شود. از آنجایی که آستانه‌های پایین‌تر مسبب رشد سریع شبکه می‌شود می‌توان پی پرد که در میانگین‌های کوچکتر شبکه زودتر فعال می‌شود و نقطه گذارش نسبت به حالتی که میانگین آستانه بزرگتر است، کوچکتر می‌باشد.همچنین برای دو منحنی با مقدار $\sigma = 4$  و  $\sigma = 5$ مقداری از آستانه گاوسی در قسمت منفی قرار گرفته است. می‌توان اینگونه استنتاج کرد که نورون‌ها حتی در آستانه‌های منفی نیز مقداری غیر صفر دارند؛ و یا می‌توان گفت مقدار $f = 0$  برای نورون‌هایی که دارای آستانه منفی هستند به منزله آستانه فعالیت بالا محسوب شده و تعدادی از آنها در این مقدار نیز فعال می‌شوند. به همین دلیل مشاهده می‌کنیم که حتی در  $f = 0$ نیز احتمال کل مکان‌های فعال و نیز اندازه خوشه به هم پیوسته از مکان‌های فعال مقداری غیر صفر را دارند.
%قابل ذکر است، زمانی که میانگین آستانه با میانگین درجه راس‌ها هم مرتبه می‌شود شبکه تنها با مقدار اولیهfرشد می‌کند. به عبارت دیگر، با افزایش آستانه میانگین، فعال شدن راس‌ها در آستانه‌های بالاتر سخت‌تر می‌شود و به همین دلیل شبکه با همان مقدار اولیه نقاط که فعال شدند رشد می‌کند. 

\begin{figure}[htbp]
\hspace*{0cm}
\centering
%\begin{minipage}[b]{0.4\textwidth}
\includegraphics[width=11cm , height=8cm]{setmax.eps}
%\includegraphics[width=0.4\linewidth, height=55mm]{max1.eps}\centering(a)    
\caption [منحنی مربوط به اندازه برگترین خوشه به هم پیوسته در شبکه ER با آستانه گاوسی]{\footnotesize
 منحنی مربوط به اندازه برگترین خوشه به هم پیوسته در شبکه ER با تعداد راس‌های $N = 10000$ و میانگین درجه $\lqngle k \rangle = 10$ برای میانگین آستانه $\mu = 6$. در اینجا نیز مشاهده می‌کنیم با افزایش پهنا تعداد نقاط فعال زیاد شده و باعث رشد سریع شبکه می‌شود.در $\sigma$های بزرگ اندازه خوشه صفر می‌شود و شبکه بااحتمال   اولیه رشد می‌‌کند. }
\label{fig:ER1}
\end{figure}
نمودار (\ref{fig:ER1}) اندازه خوشه به هم پیوسته را بر حسب تغییرات احتمال اولیه فعال نشان می‌دهد. با توجه به نتایج قبلی که برای اندازه خوشه به دست آمده است، مشاهده می‌کنیم اندازه خوشه در مقادیر کوچکfصفر است؛ و این یعنی اینکه شرط تشکیل خوشه زمانی که تعداد مکان‌های اشغال شده در شبکه کم باشد ارضا نمی‌شود. نکته‌ای که در اینجا می‌توان به آن اشاره کرد این است که به خاطر غیر یکنواخت بودن آستانه‌ی فعالیت مکان‌ها و نیز به این خاطر که آستانه‌ی فعالیت تعدادی از مکان‌ها از حد میانگین بالاتر و تعدادی از آنها نیز آستانه‌شان از حد میانگین کمتر است مشاهده می‌کنیم که با افزایش پهنا ابتدا منحنی‌ها سریع رشد کرده و در نهایت دیرتر به حالت پایا می‌رسند. در این حالت نیز می‌توان تاثیر مقدار میانگین را به وضوح مشاهده کرد که در میانگین کوچک‌تر از آنجایی که آستانه کوچک می‌شود احتمال فعال شدن مکان‌ها بیشتر شده و شبکه سریع‌تر رشد خواهد کرد. بنابراین  نمودار (\ref{fig:ER1}) مصداقی است بر این ادعا که همه نورون‌ها با یک آستانه یکسان آتش نمی‌کنند.


%\begin{figure}[htbp]
%\hspace*{0cm}
%\centering
%\begin{minipage}[b]{0.4\textwidth}
%\includegraphics[width=0.4\linewidth, height=55mm]{.eps}\centering(b)
%\includegraphics[width=0.4\linewidth, height=55mm]{sigma1.eps}\centering(a)    
%\caption{\footnotesize منحنی‌های مربوط به اندازه کل نقاط فعال در شبکه: هر دو نمودار برای یک پهنای مشترک به دست امده است. (a) ماتریس اتصال‌ها با احتمال $0.02$ و (b) ماتریس اتصال‌ها با احتمال $0.03$ درست شده است}
%\label{fig:ER2}
%\end{figure}

\subsection{تغییر اندازه بزرگترین خوشه به هم پیوسته بر حسب تغییرات آستانه‌ی فعالیت}
گفتیم از جمله پارامتر‌هایی که در تغییر رفتار خوشه به هم پیوسته نقش اساسی دارد آستانه‌ی فعالیت است. همان‌طور که در نمودار‌های قبل دیدیم، افزایش آستانه‌ی فعالیت باعث کاهش اندازه خوشه می‌شود. این گفته خود را با استفاده از نمودار (\ref{fig:SM})   تایید می‌کنیم. نمودار (\ref{fig:SM})  برای شبکه ER با تعداد راس $N = 1000$، میانگین درجه $\lqngle k \rangle = 15$ برای مقادیر مختلف کثری از مکان‌های اولیه فعال به دست آمده است. 
\begin{figure}[htbp]
\hspace*{0cm}
\centering
\includegraphics[width=11cm , height=8cm]{S_TH.eps}
%\includegraphics[width=0.4\linewidth, height=55mm]{max1.eps}\centering(a)    
\caption [تغییر اندازه خوشه بر حسب آستانه‌ی فعالیت برای شبکه ER با آستانه‌ی فعالیت یکنواخت]{\footnotesize تغییر اندازه خوشه بر حسب آستانه‌ی فعالیت یکنواخت برای شبکه ER با تعداد راس $N = 1000$ و میانگین درجه $\lqngle k \rangle = 15$  می‌بینیم که با افزایش آستانه‌ی فعالیت اندازه خوشه کوچک می‌شود و در نهایت آستانه‌های بزرگ به صفر می‌رسد. }
\label{fig:SM}
\end{figure}\\
در نمودار (\ref{fig:SM}) رفتار کاهشیِ اندازه خوشه‌ی به هم پیوسته را مشاهده می‌کنیم. می‌بینیم که در آستانه‌های کوچک اندازه خوشه بزرگ است و با افزایش آستانه اندازه آن به صفر می‌رسد.  همچنین در بحث‌های پیشین  نشان دادیم که با افزایش مکان‌های اولیه فعال اندازه خوشه کوچک می‌شود و درfهای بزرگ شبکه تنها با مکان‌های اولیه فعال شده پیش می‌رود. به وضوح می‌بینیم که این گفته با  منحنی‌های زیر هم‌خوانی دارد. هرچهfبزرگ‌تر می‌شود اندازه خوشه تا جایی پیش می‌رود که همان مقدار اولیهfرا داریم.  

اکنون موردی را در نظر می‌گیریم که در آن آستانه‌ی فعالیت گاوسی بر شبکه حاکم است و در این مورد نیز اندازه خوشه به هم پیوسته را بر حسب تغییرات میانگین آستانه‌ی فعالیت نشان می‌دهیم. از نمودار 
(\ref{fig:EM1})  به این نتیجه می‌رسیم که با اعمال آستانه گاوسی به شبکه به علت غیر همسان بودن آستانه‌ی فعالیت برای مکان‌ها، اندازه خوشه به هم پیوسته‌ی فعال برای مقادیر کوچک میانگین آستانه بیشترین مقدار خود را دارد و با افزایش آستانه میانگین اندازه خوشه نیز به کوچک می‌شود و در نهایت در میانگین آستانه‌های بزرگ به صفر می‌رسد.  
\begin{figure}[htbp]
\hspace*{0cm}
\centering
\includegraphics[width=11cm , height=8cm]{EM.eps}
%\includegraphics[width=0.4\linewidth, height=55mm]{max1.eps}\centering(a)    
\caption [تغییر اندازه خوشه بر حسب آستانه‌ی فعالیت برای شبکه ER با آستانه گاوسی]{\footnotesize تغییر اندازه خوشه بر حسب آستانه‌ی فعالیت گاوسی برای شبکه ER با تعداد راس $N = 1000$ و میانگین درجه $\lqngle k \rangle = 15$ . می‌بینیم که با افزایش آستانه‌ی فعالیت اندازه خوشه کوچک می‌شود و در نهایت در میانگین آستانه‌های بزرگ به صفر می‌رسد. }
\label{fig:EM1}
\end{figure}

\section{نمودار فاز شبکه اردوش-رنی}
در پایان نمودار فاز  سه بعدی شبکه ER  را بر حسب سه متغیر میانگین درجه $\lqngle k \rangle$، میانگین آستانه فعالیت  $\mu^*$ و احتمال نهایی تعداد کل مکان‌های فعال ($S_{a}$) رسم کردیم. نمودار زیر احتمال  نهایی تعداد کل مکان‌های فعال در گراف تصادفی ER را با $10000$ راس نشان می‌دهد. محور عمودی توزیع  درجه اتصال‌ راس ها ($\lqngle k \rangle$) و محور افقی میانگین آستانه‌ی فعالیت  ($\mu^*$) را نشان می‌دهد. شبکه با تعداد اولیه $f = 0.15$ فعال شده است و پهنای گاوسی نیز $\sigma = 0.5$ در نظر گرفته شده است. برای مقادیر کوچک $\mu$ تعداد زیادی از مکان‌ها به علت کوچک بودن آستانه می‌توانند فعال شوند و بنابراین در میانگین‌های کوچک فعال سازی خود به خودی به شکل دسته جمعی به طور ناگهانی صورت می‌گیرد. با افزایش $\mu$  تعداد مکان‌های فعال نیز کم می‌شود. از طرفی با دقت به این نمودار فاز در می‌یابیم که هرچه میانگین درجه اتصال راس‌ها نیز کوچک باشد تعداد کمتری از راس‌ها قادر خواهند بود فعال شوند و همچنین تعداد محدودی از راس‌های غیر فعال را نیز فعال کنند. بنابراین در $\lqngle k \rangle$های کوچک نیز تعداد راس‌های فعال کمتری نسبت به $\lqngle k \rangle$های بزرگ مشاهده می‌کنیم. هرچه از ناحیه تیره به سمت ناحیه روشن می‌رویم افزایش مکان‌های فعال را می‌بینیم. 
\begin{figure}[htbp]
\hspace*{0cm}
\centering
\includegraphics[width=11cm , height=8cm]{DP.eps}
%\includegraphics[width=0.4\linewidth, height=55mm]{max1.eps}\centering(a)    
\caption [نمودار فاز مربوط به شبکه ER]{\footnotesize 
نمودار فاز مربوط به شبکه ER. این نمودار احتمال نهایی تعداد کل مکان‌های فعال را برای یک مقدار میانگین آستانه و میانگین درجه نشان می‌دهد. محور افقی میانگین آستانه و محور عمودی میانگین درجه راس‌ها را نشان می‌دهد.}
\label{fig:DP}
\end{figure}

\section{شبکه با ساختار گوسی}
نتایجی که از داده‌های قبل به دست آمد، برای شبکه ER با میانگین درجه اتصال یکنواخت و همگن بوده است. در ادامه کار می‌خواهیم ببینیم با تغییر در ساختمان اتصال‌های شبکه و اعمال دینامیک یکسان با شبکه‌ ER در حصول نتایج چه تفاوت‌هایی دیده می‌شود. در فصل پیش  ذکر کردیم، شبکه‌های نورونی را با گراف تصادفی مدل می‌کنند. در این مدل نورون‌ها، راس‌ها و اتصال‌های سیناپسی، یال‌های گراف را تشکیل می‌دهند. همانطور که می‌دانیم در شبکه‌های با  اندازه  بسیار بزرگ  برای  بررسی رفتار شبکه از توزیع گاوسی استفاده می‌کنند. از آنجایی که  اتصال‌های شبکه نورونی  نیز در شبکه تعدادی بیشمار  است، ساختمان شبکه نورونی را به شکل  گاوسی ساختیم. به این صورت که با تعریف تابع گاوسی، میانگین درجه اتصال‌های شبکه را گاوسی  درنظر گرفتیم. به این صورت که با تعریف یک تابع گاوسی برای راس‌های شبکه، به هر راس  مقدار میانگبن و توریع درجه خاصی را نسبت دادیم و راس‌های شبکه با این دو مقدار داده شده به  یکدیگر متصل می‌شوند. سپس  با اعمال  دینامیک قبل شامل بررسی آستانه یکنواخت و گاوسی، ‌نتایج را به صورت زیر به دست آوردیم. در همه نمودار‌های زیر پهنا و  میانگین  توزیع گاوسی به ترتیب $10$ و $35 $ در نظر گرفته شده است. مقدار $\lqngle k \rangle = 35$  میانگین درجه اتصال‌ راس‌ها با در نطر گرفتن پهنای گاوسی و میانگین داده شده است که در نمودار‌ها آورده شده است.

\subsection{آستانه‌ی فعالیت یکنواخت}
 با نظر به نمودار (\ref{fig:gaussian})، زمانی که شبکه را با  میانگین درجه گاوسی  می‌سازیم، نسبت به شبکه با میانگین درجه همگن،‌  حالتی که توزیع گاوسی داریم شبکه سریع‌تر نسبت به توزیع همگن رشد می‌کند. همچنین نقطه گذار نیز در این ساختار در $f$های بزرگ‌تر اتفاق می‌افتد.  
 نمودار (\ref{fig:gaussian})  این نتیجه را می‌دهد که شبکه با ساختار گاوسی نسبت به شبکه با ساختار پواسونی سریع‌تر به حالت پایای خود می‌رسد. در آستانه‌های کوچک شبکه زود‌تر رشد می‌کند و هرچه اندازه آستانه بزرگ‌تر می‌شود شبکه با احتمال اولیه فعال رشد می‌کند.
 دلیل این رفتار نیز به علت بزرگ بودن میانگین درجه ساختار گاوسی نسبت به شبکه ER با میانگین درجه همگن است. به عبارت دیگر، در ساختار گاوسی به دلیل بزرگ بودن میانگین درجه، یک راس با راس‌های زیادی در شبکه اتصال دارد. نتایج برای شبکه با $N=10000$ راس و با میانگین‌گیری روی $E=100$  نمونه به دست آمده است. همچنین میانگین درجه برای این ساختار را $\lqngle k \rangle = 35$  در نظر گرفته‌ایم. از هر دو نمودار به دست آمده در (\ref{fig:gaussian}) به این نتیجه می‌رسیم در شبکه‌هایی که میانگین درجه آنها زیاد است اندازه بزرگ‌ترین خوشه به هم پیوسته از مکان‌های فعال با احتمال نهایی تعداد کل مکان‌های فعالدر شبکه یکسان است. 
 
\begin{figure}[htbp]
\hspace*{0cm}
  %\begin{minipage}[b]{0.4\textwidth}
\includegraphics[width=11cm , height=8cm]{gaussavtive1.eps}\centering(الف)
\includegraphics[width=11cm , height=8cm]{gaussmax1.eps}\centering(ب)
\caption [احتمال نهایی تعداد کل مکان‌های فعالو بزرگ‌ترین خوشه‌ی فعال به هم پیوسته برای شبکه  با ساختار گاوسی]{\footnotesize منحنی مربوط به
 احتمال نهایی تعداد کل مکان‌های فعال(الف) و بزرگ‌ترین خوشه‌ی فعال به هم پیوسته (ب) برای شبکه  با ساختار گاوسی.  در این منحنی میانگین درجه برابر با $\lqngle k \rangle  = 35$
 برای شبکه با $N = 10000$ راس در نظر گرفته شده است.}
\label{fig:gaussian}
\end{figure}
\newpage 
 \subsection{آستانه‌ی فعالیت گاوسی}
 از آنجایی که همه نورون‌ها یکسان نیستند و هر نورون آستانه فعال شدنش با نورون‌های دیگر متفاوت است بنابر این با در نظر گرفتن دینامیک با آستانه گاوسی رفتار شبکه را بررسی می‌کنیم.
 
 با توجه به نمودار (\ref{fig:gauss}) این واقعیت برای ما روشن می‌شود که نورون‌ها با یک آستانه یکسان آتش نمی‌کنند. بلکه تعدادی از آن‌ها با آستانه‌ای بزرگتر وتعدادی نیز با آستانه‌ای کوچک‌تر از آستانه‌ی فعالیت آتش می‌کنند همچنین مشاهده می‌کنیم با افزایش پهنای آستانه گاوسی ($\sigma$)، شبکه زودتر به حالت پایا می‌رسد و اندازه خوشه به هم پیوسته از مکان‌های فعال  نیز با افزایش پهنای  $\sigma$، بزرگتر می‌شود. $\mu$ آستانه میانگین شبکه  است و میانگین‌گیری نیز روی $E=100$ انجام شده است.
\begin{figure}[htbp] 
\hspace*{0cm}
\centering
\includegraphics[width=11cm , height=8cm]{TGA.eps}\centering(الف)    
\includegraphics[width=11cm , height=8cm]{TGM.eps}\centering(ب)
\caption [منحنی مربوط به توزیع آستانه گاوسی]{\footnotesize
 منحنی مربوط به توزیع آستانه گاوسی برای شبکه با $N=10000$ راس برای دو حالت (الف) احتمال کل مکان‌های فعال و (ب) اندازه بزرگترین خوشه‌ی به هم پیوسته فعال. منحنی‌های مدنظر با میانگین درجه $\lqngle k \rangle = 35$   برای آستانه میانگین  $\mu = 15$ و پهنا‌های گاوسی متفاوت رسم شده است.}
\label{fig:gauss}
\end{figure}

\newpage
\subsection{تغییر اندازه خوشه‌ی به هم پیوسته بر حسب آستانه‌ی فعالیت}
نمودار (\ref{fig:TS})  روند تغییرات اندازه خوشه‌ی به هم پیوسته از مکان‌‌های فعال را بر حسب تغییرات آستانه‌ی فعالیت برای شبکه با ساختار گاوسی با $N=1000$ راس و میانگین درجه $\lqngle k \rangle = 35$  نشان می‌دهد. با توجه به این نمودار می‌بینیم که اندازه خوشه در آستانه‌های کوچک بزرگ است و برای آستانه‌‌های بزرگ‌تر رفتار کاهشی دارد.

\begin{figure}[htbp] 
\hspace*{0cm}
\centering
\includegraphics[width=11cm , height=8cm]{TS.eps}
\caption [منحنی مربوط به تغییرات اندازه خوشه بر حسب تغییرات آستانه‌ی فعالیت مربوط به ساختار گاوسی]{\footnotesize منحنی مربوط به تغییرات اندازه خوشه بر حسب تغییرات آستانه‌ی فعالیت برای شبکه با $N=10000$ راس. }
\label{fig:TS}
\end{figure}
 \newpage
مشابه با شبکه ER در این مورد نیز می‌توانیم تغییرات اندزه خوشه به هم پیوسته از مکان‌های فعال را بر حسب تغییرات میانگین آستانه برای آستانه‌های گاوسی مشاهده کنیم.  نمودار (\ref{fig:MM1})  رفتار تغییر اندازه خوشه را برحسب تغییرات آستانه‌ی فعالیت گاوسی نشان می‌دهد. از مقایسه این نمودار با نمودار (\ref{fig:EM1}) درمی‌یابیم که در شبکه با ساختار گاوسی به علت بزرگ بودن میانگین درجه نسبت به ساختار ER، تغییرات اندازه خوشه بر حسب میانگین آستانه‌ی $\mu$ در آستانه‌های کوچک سریع‌تر به حالت پایا می‌رسد. به همین دلیل اندازه خوشه در ساختار گاوسی برای میانگین آستانه‌های کوچک بزرگ‌تر است. $\sigma$ پهنای آستانه فعالیت گاوسی را نشان می‌دهد.
\begin{figure}[htbp] 
\hspace*{0cm}
\centering
\includegraphics[width=11cm , height=8cm]{MM.eps}
\caption [منحنی مربوط به تغییرات اندازه خوشه بر حسب تغییرات آستانه‌ی فعالیت گاوسی]{\footnotesize منحنی مربوط به تغییرات اندازه خوشه بر حسب تغییرات آستانه‌ی فعالیت گاوسی برای شبکه با $N=10000$ راس. با توچه به}
\label{fig:MM1}
\end{figure}\\\
 
\newpage
\section{نتیجه گیری}
در این رساله به منظور بررسی میزان تاثیر آستانه‌ی فعالیت و دیگر عوامل دخیل در اندازه بزرگ‌ترین خوشه‌ی فعال به هم پیوسته از راس‌ها(نورون‌ها)ی فعال دو ساختار ER با میانگین درجه همگن و ساختار گاوسی با میانگین درجه گاوسی مورد استفاده قرار گرفته است.  
ابتدا  برای درک بهتر تراوش استاندارد را روی شبکه مربعی دو بعدی  و گراف انجام دادیم. مشاهده کردیم که در شبکه مربعی  تنها با بزرگ شدن اندازه شبکه نقطه گذار به مقدار واقعی خود می‌رسد. همچنین گذار پیوسته را در این شبکه دیدیم.  در بررسی نتایج حاصل از تراوش معمولی روی گراف نیز مشاهده کردیم که نقطه گذار در این حالت با معکوس میانگین درجه خود ارتباط مستقیم دارد. همچنین در این مورد نیز گذار پیوسته را در رفتار خوشه به ‌هم پیوسته دیدیم. 


و اما در بررسی نتایجی که از تراوش خود‌راه‌انداز روی شبکه ER و ساختار گاوسی  به دست آوردیم  مشاهده کردیم که با افزایش احتمال  اولیه در شبکه، افزایش ناگهانی در مکان‌های فعال نهایی شبکه صورت می‌گیرد و شبکه زودتر به حالت پایا می‌رسد. با توجه به نمودار (\ref{fig:EG}) می‌بینیم در شبکه‌ای که میانگین درجه راس‌ها بزرگ است و راس‌های بیشتری به هم متصل هستند(ساختار گاوسی)، نسبت به شبکه با میانگین درجه پایین(ساختار ER)، زودتر رشد می‌کند و بااحتمال اولیه کمتری به حالت پایا می‌رسند. در نمودار (\ref{fig:EG}) دو ساختار گاوسی و ER نشان داده می‌شود. ساختار گاوسی با میانگین درجه $\lqngle k \rangle = 35$  و ساختار ER با میانگین درجه $\lqngle k \rangle = 5$ است. دینامیک یکسان با آستانه‌ی فعالیت $m = 5$ نیز روی شبکه اثر می‌گذارد.
\begin{figure}[htbp]
\hspace*{0cm}
\centering
\includegraphics[width=11cm , height=8cm]{EG.eps}
%\includegraphics[width=0.4\linewidth, height=55mm]{max1.eps}\centering(a)    
\caption [مقایسه اندازه خوشه به هم پیوسته برای دو ساختار تصادفی ER و گاوسی]{\footnotesize 
مقایسه اندازه خوشه به هم پیوسته برای دو ساختار تصادفی ER و گاوسی. در این نمودار مشاهده می‌کنیم که با اعمال دینامیک یکسان به دو شبکه با ساختار‌ متفاوت، شبکه‌ی گاوسی به خاطر بزرگ بودن میانگین درجه نسبت به ساختار ER سریع‌تر به حالت پایا می‌رسد. میانگین درجه ساختار تصادفی ER،  $\lqngle k \rangle = 5$ و میانگین درجه شبکه با ساختار گاوسی $\lqngle k \rangle = 35$ است. در هر دو شبکه دینامیک یکسان با آستانه‌ی فعالیت $m = 5$ اعمال شده است.}
\label{fig:EG}
\end{figure}\\\
 همچنین نتایج حاصل از هر دو مدل حاکی از آن است اندازه بزرگ‌ترین خوشه به هم پیوسته از مکان‌های فعال با افزایش آستانه‌ی فعالیت m کاهش می‌یابد و در  نقطه گذار $m = m_{c}$ به صفر می‌رسد. زمانی که آستانه‌ی فعالیت بزرگ‌تر از نقطه گذار باشد شبکه به سمتی پیش می‌رود که دیگر خوشه بزرگی در آن تشکیل نمی‌شود. در این زمان گذار فاز پیوسته‌ای را در رفتار خوشه‌ی فعال به هم پیوسته  دیدیم. این گذار با پارامتر نظم $S_{gc}$ (اندازه برزگترین خوشه‌ی فعال به هم پیوسته) سنجیده می‌شود. به عبارت دیگر می‌توان اینگونه تعبیر کرد؛ اگر اندازه خوشه صفر باشد، آستانه‌ی فعالیت بزرگ‌تر از مقدار  آستانه در نقطه گذار و اگر اندازه خوشه، بزرگ و البته هم‌مرتبه با اندازه شبکه مورد نظر باشد آستانه کوچک‌تر از مقدار  آستانه در نقطه گذار است. در یک تعریف ریاضی‌گونه گفته خود را به شکل زیر می‌گوییم:\\\\
 

\begin{aling}
\begin{center}
\begin{cases}
\[
\text{if}~~~~~S_{gc} = 0 \longrightarrow m = m_{c}.\\
\text{if}~~~~~S_{gc} > 0 \longrightarrow m < m_{c}.           
\end{cases}
\]
\end{center}
\end{aling}
از طرفی، در یک حالت حدی، زمانی که آستانه‌ی فعالیت خیلی بیشتر از  مقدار  آستانه در نقطه گذار باشد $(m >> m_{c})$، شبکه کاملا جداگانه رفتار می‌کند و مکان‌های شبکه(نورون‌ها) فقط با تحریک خارجی اولیه برانگیخته می‌شود. در این صورت برای همه مقادیر احتمال اولیهfتنها یک جواب $\phi = f$  خواهیم داشت که نشان از رفتار خطی شبکه باfاست. همچنین در هر دو مورد رفتار اندازه خوشه به هم پیوسته در نقطه گذار، هم گذار فاز پیوسته و هم گذار فاز ناپیوسته را نشان می‌دهد.

این نکته قابل ذکر است در بررسی مطالعات خود روی شبکه نورونی فرض را بر این گذاشتیم که تمام نورون‌ها تحریکی بوده و نیز نورون‌ها وزن یکسانی برابر با یک دارند. همچنین فرض بر غیرجهتی بودن شبکه نیز شده است.

 به طور خلاصه می‌توان گفت با مقایسه رفتار اندازه بزرگ‌ترین خوشه هم‌بند در دو شبکه ER و شبکه با ساختار گاوسی به این نتیجه رسیدیم که شبکه با ساختار گاوسی به دلیل اتصال‌های بیشتر و میانگین درجه راس‌های بالاتر نسبت به شبکه ER با میانگین درجه یکنواخت، برای فعال شدن ناگهانی شبکه و نیز تشکیل خوشه به هم پیوسته به احتمال اولیهfبزرگ‌تری نیاز دارند. در صورتی که شبکه ER با میانگین درجه اتصال‌های خود در $f$های کوچک به یک‌باره فعال می‌شوند. همچنین آستانه‌ی فعالیتی که برای شبکه با ساختار گاوسی در نظر گرفته می‌شود بزرگ‌تر از آستانه‌ی فعالیت شبکه ER است. 
 
  \section{کار‌های پیش‌رو و پیشنهادها}
   در ادامه کار قصد داریم ابتدا شبکه جهتی را جایگزین شبکه غیر جهتی کنیم و نیز  وزن خاصی را به یال‌های شبکه نسبت می‌دهیم که معرف قدرت سیناپس‌ها باشد. در این صورت میانگین درجه اتصال راس‌ها به گونه‌ای متفاوت خواهد بود و هر راس درجه ورودی و خروجی خود را دارد. همچنین می‌توانیم در مطالعات آتی تاثیر نورون‌های مهاری را نیز روی  انتشار فعال‌سازی نورون‌های شبکه بررسی کنیم. 
   \newpage
 \section{خلاصه فصل چهارم}
 \begin{itemize}
 \item شبکه ER دارای ساختاری با میانگین درجه یکنواخت و میانین درجه کوچک از مرتبه $\lqngle k \rangle = 5$ تا $\lqngle k \rangle = 10 $ است. 
 \item در شبکه ER، برای بررسی رفتار بزرگ‌ترین خوشه به هم پیوسته از مکان‌های فعال آستانه‌های کوچک‌تری نیاز است تا شبکه به یکباره فعال شود. این فعال شدن ناگهانی گذار فازی را در شبکه ایجاد می‌کند که در نقطه گذار  $f$، قبل آن خوشه‌ای نداریم و بعد از آن خوشه فعال به هم پیوسته دیده می‌شود.
 \item شبکه با ساختار گاوسی با میانگین درجه اتصال  بزرگ در $f$های بزرگ‌تر این گذار را از خود نشان می‌دهد. به این خاطر که در 
 $f$های کوچک به دلیل  گسسته بودن و جدا بودن نورون‌ها از هم تشکیل بزرگ‌ترین خوشه  فعال به هم پیوسته در شبکه ممکن نبوده و شبکه با مقدار اولیه نورون‌ها(مکان‌ها) رشد می‌کند.
 \item در هر دو شبکه در نقطه گذار دو نوع گذار فاز پیوسته و گسسته را در رفتار اندازه خوشه و کل مکان‌های فعال  مشاهده کردیم. دیدیم که در آن نقطه رفتارشان پیوسته  است اما همانند گذار فاز گسسته جهش ناگهانی را از خود نشان می‌دهند.
 \item اندازه بزرگ‌ترین خوشه‌ی فعال در شبکه با افزایش آستانه‌ی فعالیت کاهش می‌یابد و در آستانه‌های بالاتر اندازه خوشه به صفر می‌رسد.

 \end{itemize} 






