\noindent 
Standard percolation theory is the study of clusters behavior in a network. One of the important issues in  percolation theory is finding the size and the emergence point of infinitively large cluster of occupied locations. A generalized model of standard percolation is a model called bootstrap percolation. This model has been introduced to describe development of an activation process on the network in which activation emerges collectively through successive activation of location in them. Activation process in neuronal network is an illustrative example of bootstrap percolation. In this thesis we investigate neurons activation process in a neural network. In such a network, when a neuron reaches its activation threshold and fire, affects its neighbouring neuron. In the bootstrap percolation, a neuron will be activated on the condition that at least $m$ neighbouring neuron of that active. Collective firing of neuron, leads to a connected cluster of active neuron which is a noticeable fraction of the whole neuronal network. The site of the active cluster obtained using bootstrap percolation models and simulations. In this study, for simplicity, this model has been implemented on a random network whit  
 uniform distribution of degree, and then generalized to a network whit Gaussian distribution of degree, which is appropriate for neuronal networks. We see that increasing the number of formerly active neurons causes the number of active neuron to process at the end, and in a point grows dramatically. More over the size of giant component cluster of active neurons, will show a phase transition which is a combination first and second order phase transition. These transition are studied for uniform and Gaussian distribution fraction and for various threshold.
 
\latinkeywords{Activation, Bootsrap Percolation, Giant Connected Component, Phase Transition}
