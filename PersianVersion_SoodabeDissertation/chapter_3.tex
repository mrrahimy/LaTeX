\chapter{تراوش}
\section{مقدمه}
 مساله تراوش\LTRfootnote{percolation} (نفوذپذیری) به مطالعه رفتار  خوشه‌ها  در شبکه‌های با ابعاد بی‌نهایت می‌پردازد. این خوشه‌ها زمانی در شبکه ظاهر می‌شوند که مکان‌های اشغال شده به یکدیگر متصل شده و جزیره‌هایی از مکان‌های اشغال شده را به وجود آورند. در این فصل مفاهیم مربوط به تراوش را روشن‌تر بیان می‌کنیم و سپس مدل استاندارد آن را بر روی شبکه دوبعدی و گراف تصادفی بررسی خوهیم کرد. در انتها نیز تراوش خودراه‌انداز را به شکل ساده برای درک بیشتر آن مختصری چند خواهیم گفت.
\section{تعاریف}
تراوش یکی از مفاهیم مهم در فیزیک به شمار می‌آید و در بسیاری از پدیده‌های طبیعی می‌توان آن را مشاهده کرد. 
در طول پنج دهه گذشته نظریه‌ی تراوش مطالعات گسترده‌ای را برای درک بهتر در زمینه‌های مختلف فیزیک، علم مواد، شبکه‌های پیچیده و زمینه‌های دیگر برای دنیای بشریت  فراهم آورده است.  در فیزیک آماری و ریاضی این مدل رفتار خوشه‌های به هم ‌پیوسته  از مکان‌ّای فعال را در یک گراف توصیف می‌کند. در علم زمین شناسی نیز این مساله به شارش آب از میان خاک و سنگ‌‌های نفوذپذیر اشاره می‌کند. در ادامه بحث، ابتدا خواص و انواع تراوش را برشمرده و سپس این مدل استاندارد آن را برای ساختار مربعی و گراف  مورد بررسی قرار می‌دهیم.
\subsection{ویژگی‌های تراوش}
در بررسی مساله تراوش برخی از کمیت‌ها، ویژگی‌های شبکه را برای ما روشن می‌سازد. از جمله این کمیت‌ها اندازه یا تعداد مکان‌های اشغال شده و نیز تعداد خوشه‌ها در شبکه با ابعاد بزرگ است.   
چگالی خوشه‌ها با اندازه $\textsc{s}$، ($\textsc{n}_{\textsc{s}}(\textsc{f})$) و احتمالی که یک مکان پر متعلق به خوشه‌ی بی‌‌نهایت باشد ($\textsc{P}_{\infty}(\textsc{f})$)، به شکل زیر تعریف می‌شوند، 
\begin{align}
\textsc{n}_{\textsc{s}}(\textsc{f}) &= \dfrac{\text{تعداد خوشه‌ها با ابعاد $\textsc{s}$ }}{\text{تعداد کل مکان‌ها در شبکه }},\\ \textsc{P}_{\infty}(\textsc{f})  &= \dfrac{\text{تعداد مکان‌های اشغال شده در خوشه بی‌نهایت}}{\text{تعداد کل مکان‌های اشغال شده در شبکه }}.
\end{align}
$\textsc{f}$ احتمال اولیه اشغال شدن مکان‌های شبکه است. در یک شبکه دو بعدی $\textsc{P}_{\infty}$ در نقطه گذار $\textsc{f}_{\textsc{c}}$  دو مقدار صفر و یک دارد. می‌توان گفت:
\begin{equation}
\[
\begin{cases}
  \text{if } & \textsc{f} < \textsc{f}_{\textsc{c}}~~~~~~~~\ \textsc{P}_{\infty} = 0 \\
   \text{if} & \textsc{f} > \textsc{f}_{\textsc{c}}~~~~~~~~\textsc{P}_{\infty} = 1             
\end{cases}
\]
\end{equation}
\begin{figure}[htbp]
\hspace*{0cm}
\centering
%\begin{minipage}[b]{0.4\textwidth}
\includegraphics[width=0.4\linewidth, height=50mm]{per.png}\centering(الف)    
\includegraphics[width=0.4\linewidth, height=50mm]{per1.png}\centering(ب)
\caption[مثالی از تراوش بر روی شبکه مربعی] {\footnotesize مثالی از تراوش بر روی شبکه مربعی با اندازه $16\times 16$:  (الف) برای $\textsc{f} = 0.2$ 
خوشه‌هایی با چگالی 
$\textsc{n}_{\textsc{s}}(1) = 20$ 
و 
$\textsc{n}_{\textsc{s}}(2) = 4$
 و 
 $\textsc{n}_{\textsc{s}}(3) = 5$
  و
 $\textsc{n}_{\textsc{s}}(7) = 1$
     دیده می‌شود که به ترتیب تعداد خوشه‌ها با اندازه $1$  و $2$ و $3$ و $7$  را نشان می‌دهد. (ب) 
 $\textsc{f} = 0.59$
     . برای این پیکربندی 
  $\textsc{P}_{\infty}(\textsc{f} = 0.56) = 140/154$ \cite{jan}.}
\label{fig:jan}
\end{figure}\\
 شکل (\ref{fig:jan}) مثالی از روشن شدن مفاهیم 
 $\textsc{P}_{\infty}$
  و  
 $\textsc{n}_{\textsc{s}}(\textsc{p}})$
  است. قسمت (الف) برای مقدار
 $\textsc{f} = 0.2$
    تعداد خوشه‌ها با اندازه $1$، $2$، $3$ و  $7$ را نشان می‌دهد. چگالی این خوشه‌ها به ترتیب 
 $\textsc{n}_{\textsc{s}}(1) = 20$ 
 و 
 $\textsc{n}_{\textsc{s}}(2) = 4$
  و 
 $\textsc{n}_{\textsc{s}}(3) = 5$ 
  و
 $\textsc{n}_{\textsc{s}}(7) = 1$
    است. قسمت ‌(ب) برای 
 $\textsc{f} = 0.59$ 
    مقدار 
 $\textsc{P}_{\infty}(\textsc{f} = 0.56) = 140/154$
      را نشان می‌دهد. 
 



%\begin{align}
%P_{\infty}& = \dfrac{\text{تعداد مکان‌های اشغال شده در خوشه بی‌نهایت}}{\text{تعداد کل مکان‌های اشغال شده در شبکه }}.\\
%&  ~~~~~n_{s}(p) = \dfrac{\text{تعداد خوشه‌ها با ابعاد $s$}}{\text{تعداد کل مکان‌هادر شبکه }}.
%\end{align}


\section{انواع تراوش}
در تعریف تراوش به دو نوع تراوش جایگاهی\LTRfootnote{site-percolation} و پیوندی\LTRfootnote{bond-percolation} می‌توان اشاره کرد\cite{percolate}. 
\subsubsection{تراوش جایگاهی}
فرض می‌کنیم در جعبه‌ای به طور تصادفی گلوله‌های فلزی و شیشه‌ای ریخته‌ایم. می‌خواهیم بدانیم که آیا جعبه به شکل یک رسانا رفتار می‌کند یا عایق. به زبان دیگر، به دنبال مسیری هستیم که از به هم پیوستن گلوله‌های فلزی، بالا و پایین و یا چپ و راست شبکه را به هم متصل کند. تصوّر می‌کنیم یک شبکه دو بعدی داریم که همه مکان‌هایش تهی است. در یک حالت تصادفی، گلوله‌های فلزی  با احتمال 
$\textsc{f}$
 پر می‌شوند و یا با احتمال
  $1 - \textsc{f}$ 
  گلوله‌های شیشه‌ای قرار می‌گیرند. یال ها نیز در این شبکه اتصال گلوله‌هایی است که با یکدیگر در تماس هستند. با این ساختار به دنبال نقطه گذاری هستیم تا رفتار شبکه را بررسی کنیم. می‌بینیم که در در شکل (\ref{fig:jan})  تصویر اول (راست) مسیر پرکولیت وجود ندارد، اما  در تصویر دوم (چپ) مسیر به وضوح پیدا است. 
\begin{figure}[htbp]
\hspace*{0cm}
\centering
%\begin{minipage}[b]{0.4\textwidth}
\includegraphics[width=0.3\linewidth, height=45mm]{path1.png}\centering(الف)    
\includegraphics[width=0.3\linewidth, height=45mm]{path.png}\centering(ب)
\caption[شمایی از تراوش جایگاهی] {\footnotesize شمایی از تراوش جایگاهی با ابعاد $35\times 35$: (الف) در احتمال 
$\textsc{f} = 0.25$ 
تراوش رخ نمی‌دهد و (ب) در احتمال 
$\textsc{f} = 0.65$ 
تراوش و مسیر خوشه مشخص است \cite{bela}.}
\label{fig:site}
\end{figure}
به این ترتیب نتیجه می‌گیریم مادامی که در گراف با راس‌ها سروکار داشته باشیم تراوش از نوع جایگاهی داریم. در این نوع تراوش مکان ها مستقل از هم با احتمال $\textsc{f}$ اشغال می‌شوند. در تراوش جایگاهی به دنبال خوشه به هم پیوسته از مکان‌های اشغال شده هستیم \cite{bela}.
\subsubsection{تراوش پیوندی}
برای درک بهتر از مدل تراوش  پیوندی  می‌توان مثال‌های زیادی را مورد بررسی قرار داد. یکی از مثال‌ها مربوط به شارش آب از میان یک محیط متخلخل است \cite{sahini}.

مساله از این قرار است که فرض می‌کنیم یک محیط متخلخل مانند اسفنج داریم و در این محیط از بالا آب می‌ریزیم. در اینجا سوالی  که برای ما مطرح می‌شود  این است که آیا مسیری وجود خواهد داشت تا مایع خود را از بالا به پایین برساند؟ رفتار این مساله از نوع تراوش پیوندی است. 

محیط متخلخل است. اما این تخلخل چقدر باشد تا آب را از خود نفوذ دهد؟ نکته اصلی در این است که حتما باید مسیری وجود داشته باشد که آب از بالا به پایین برسد.  فرض کنیم هر زمان  که آب به یکی از این مسیر‌ها وارد شود، می‌تواند به یکی از همسایه‌هایش نفوذ کند، به شرط آنکه آن مسیر‌ِ همسایه نیز دارای حفره‌ای باشد که جا برای آب وجود داشته باشد. به همین منوال هر مسیرِ 
همسایه نیز بتواند آب را به حفره‌ی مجاور خود برساند. 

اگر فرض کنیم احتمال اینکه هرکدام از این مسیرها که آب را عبور می‌دهد $\textsc{f}$ باشد (یعنی با احتمال $\textsc{f}$ یکی از مسیرها را روشن می‌کنیم)، و با احتمال
 $1 - \textsc{f} = \textsc{q}$ 
 آب نفوذ نکند؛ با این شرایط می‌توانیم مجموعه‌ای را به صورت آماری و تصادفی بسازیم. به این صورت که  برای هر مسیر شبکه یک عدد تصادفی بین صفر و یک انتخاب می‌کنیم. اگر آن عدد تصادفی از $\textsc{f}$ بزرگ‌تر بود پیوند  بین دو مکان را روشن می‌کنیم. اگر عدد $\textsc{f}$ خیلی کوچک باشد تعداد خیلی کمی از پیوندها روشن خواهد شد و احتمال اینکه تراوش اتفاق بیفتد خیلی کم است.  اما اگر 
 $\textsc{f} = 1$
  باشد، یعنی همه مکان‌ها روشن هستند و حتما تراوش اتفاق افتاده است. بنابراین ترواش پیوندی به مطالعه و پیدا کردن خوشه‌های به هم ویوسته   می‌پردازد که در آن خوشه به جای مکان‌ها، یال‌ها به هم متصل هستند \cite{book}.
 شکل (\ref{fig:bond})  این موقعیت را برایمان مشخص می‌کند.
\begin{figure}[htbp]
\hspace*{0cm}
\centering
%\begin{minipage}[b]{0.4\textwidth}
\includegraphics[width=0.3\linewidth, height=45mm]{pathbond.png}\centering(الف)    
\includegraphics[width=0.3\linewidth, height=45mm]{pathbond1.png}\centering(ب)
\caption[شمایی از تراوش پیوندی] {\footnotesize شمایی از تراوش پیوندی با ابعاد $40\times 40$: (الف) تراوش رخ می‌دهد  (ب) تراوش رخ داده است \cite{bela}.}
\label{fig:bond}
\end{figure}\\
\ 

 

 
%\subsection{بررسی تراوش استاندارد در یک بعد}

%در فضای یک بعدی مساله خیلی بدیهی است و به شکل تحلیلی قابل حل است. در این فرایند $f_{c}$ برابر یک است. برای اثبات این ادعا، شبکه یک بعدی با مجموعه نامحدودی از مکان‌ها را مطابق با شکل (\ref{fig:1D})  در نظر می‌گیریم که در فاصله برابر در طول یک خط چیده شده‌اند. همه مکان‌ها دو حالت دارند: یا با احتمال $\textsc{f}$ اشغال می‌شوند و یا با احتمال $1 - f$ تهی باقی می‌مانند.

%\begin{figure} [htbp]
%\centering
%\includegraphics[width=8cm , height=1cm]{1D.png} 
%\caption{\footnotesize تراوش در یک بعد. سایت‌ها با احتمال $\textsc{f}$ اشغال می‌شوند. علامت ضرب‌در نقاط تهی و دایره‌های مشکی نقاط اشغال شده‌اند. در این شکل خوشه‌هایی با اندازه پنج ، دو و یک را مشاهده می‌کنیم\cite{percolation}.}
%\label{fig:1D}
%\end{figure}

%چیزی که برای ما اهمیت دارد پیدا کردن نقطه گذار برای تشکیل خوشه در هر بعدی برای این نظریه است. در مساله یک بعدی، خوشه پرکولیت شده از $-\infty$ تا $\infty$ را شامل می‌شود و بدیهی است این زمانی امکان‌پذیر است که همه مکان‌ها اشغال شده باشند. زمانی که همه مکان‌ها اشغال شوند به این معنی است که نقطه گذار $f_{c}$ برابر یک است و اگر تنها یک مکان تهی در شبکه وجود داشته باشد مانع از تشکیل شدن خوشه تراوا می‌شود. اما در دو و سه بعد پیدا کردن مقدار $f_{c}$ به نوع شبکه بستگی دارد. برای بعضی از شبکه‌ها برای محاسبه حد بحرانی حل دقیق داریم. اما برای برخی دیگر حل دقیق پاسخ‌گوی نیاز ما نیست و باید از شبیه‌سازی برای تعیین مقدار بحرانی بهره بگیریم.
\section{مدل تراوش استاندارد در دو بعد}


{برای بررسی مساله تراوش استاندارد در شبکه‌های دوبعدی، یک شبکه \textsc{N } $\times$ \textsc{N} از مکان‌ها را در نظر می‌گیریم.
 $\textsc{N}$ 
 تعداد کل مکان‌ها در شبکه است. همان‌طور که در مبحث‌های پیشین گفتیم، این مکان‌ها با احتمال $\textsc{f}$ اشغال می‌شوند. با در نظر گرفتن عدد تصادفی برای هر مکان و  مقایسه احتمال $\textsc{f}$ با عدد تصادفی ، مکان‌ها را اشغال می‌کنیم. مکان‌هایی که اشغال شده‌اند  نیز تا پایان در همین حالت باقی می‌مانند. مکان‌های شبکه با احتمال داده شده پُر می‌شوند و خوشه‌هایی در این شبکه شکل می‌گیرند. هدف  ما تعیین تعداد نقاط اشغال شده و به دست آوردن بزرگ‌ترین خوشه به هم پیوسته از مکان‌های اشغال شده از میان چندین خوشه موجود در شبکه است. خوشه‌ای که شکل می‌گیرد، کسری از نقاط اشغال شده در شبکه است. در تراوش استاندارد  در شبکه‌ مربعی، رفتار خوشه نسبت به پارامتر $\textsc{f}$  پیوسته است و  نقطه گذار با اندازه شبکه جابجا می‌شود.  همچنین زمانی که اندازه شبکه را زیاد می‌کنیم به نقطه گذار واقعی نزدیک‌ می‌شویم.
 
 برای به دست آوردن بزرگ‌ترین خوشه به هم پیوسته  در شبکه مربعی از الگوریتمی که هوشن\LTRfootnote{Hoshen} و کوپلمن\LTRfootnote{Kopelman} در سال $1976$ پیشنهاد دادند  استفاده  کرده‌ایم. در این الگوریتم آرایه‌ها سطر به سطر از چپ به راست و از بالا به پایین پویش می‌شوند و مکان‌هایی که پُر و متصل به مکان‌های قبلی هستند یافت می‌شوند. با تکرار این روند و پیمایش کامل مکان‌ها، بزرگ‌ترین خوشه تعیین می‌شود \cite{baba}.  }

%مزایای این الگوریتم نسبت به باقی روش‌ها، ساده و سریع‌ بودن الگوریتم، استفاده از حافظه کم برای اجرای الگوریتم و ... است.
 الگوریتم هوشن-کوپلمن به این صورت است که همه مکان‌های پر از بالا به پایین و از چپ به راست پوییده می‌شوند و هر مکان پر شده با یک برچسب به عنوان اندازه خوشه معین می‌شود. الگوریتم اینگونه پیش می‌رود که اگر مکان اشغال شده‌ای همسایه پر نداشته باشد با برچسب جدید مشخص می‌شود. به عبارت دیگر مکان‌هایی که همسایه پر ندارند به عنوان یک خوشه جدید و مجزا شناخته می‌شوند. اگر مکانی یک همسایه پر داشته باشد با برچسب همسایه‌اش هم علامت می‌شود و به اندازه خوشه همسایه اضافه می‌شود. اما اگر مکان اشغال شده دو یا بیشتر از دو همسایه پر داشته باشد،  به همسایه‌ای متصل می‌شود که  برچسب آن کوچک‌تر است. در پیمایش بعدی همه مکان‌هایی که اشغال و به هم متصل هستند هم‌برچسب می‌شوند و به این ترتیب بزرگ‌ترین خوشه به هم پیوسته از مکان‌های اشغال را پیدا می‌کنیم \cite{hosh}. . 
 
 شکل (\ref{fig:s1}) اندازه بزرگ‌ترین خوشه به هم پیوسته از مکان‌های اشغال شده را برحسب تعداد مکان‌های اشغال شده اولیه برای شبکه مربعی با  اندازه‌های متفاوت نشان می‌دهد. در این شکل
  $\textsc{S}_{\textsc{gc}}$ 
  اندازه  بزرگ‌ترین خوشه به هم‌ پیوسته از مکان‌های اشغال شده و 
  $\textsc{f}$
   احتمال اولیه اشغال شدن مکان‌ها است. میانگین‌گیری روی $100$ نمونه انجام شده است.
 
\begin{figure} [htbp]
\centering
\includegraphics[width=11cm , height=8cm]{max3.eps} 
\caption[اندازه بزرگترین خوشه به هم پیوسته در یک شبکه دو بعدی] {\vspace{-0.01}\footnotesize اندازه بزرگترین خوشه به هم پیوسته  بر حسب تعداد مکان‌های اوبیه اشغال شده در یک شبکه دو بعدی مربوط به تراوش استاندارد.}
\label{fig:s1}
\end{figure}
 نمودار (\ref{fig:s1}) این گفته را برای ما یادآوری می‌کند که تغییر در اندازه شبکه، نقطه گذار را تغییر داده و با افزایش اندازه به نقطه گذار واقعی  نزدیک می‌شویم. علاوه بر این، در این نمودار‌ها گذار پیوسته را برای تراوش استاندارد مشاهده می‌کنیم. 
 
\section{تراوش استاندارد بر روی گراف تصادفی}
   برای بررسی تراوش استاندارد بر روی شبکه‌های تصادفی از مدل گراف تصادفی اردوش و رنی (\textsc{ER}) استفاده می‌کنیم. در این گراف با 
   $\textsc{N}$ 
   راس، در ابتدا با یک احتمال اولیه 
   $\textsc{p}$ 
   تعدادی از راس‌ها را به طور تصادفی به هم وصل می‌کنیم. میانگین درجه راس‌ها در حد 
   $\textsc{N}$های بزرگ با رابطه زیر به احتمال
    $\textsc{p}$
     مربوط است \cite{newman}، 
   \begin{equation}
   {\left\langle \textsc{k} \right\rangle = \textsc{p}\times \textsc{N}}.
   \end{equation}
   همچنین تعدادی از مکان‌ها را با احتمال اولیه $\textsc{f}$ فعال می‌کنیم. سپس بزرگترین خوشه از مکان‌های فعال را در شبکه به دست می‌آوریم. 
    قبل از آنکه به بررسی نتایج بپردازیم به این نکته نیز اشاره می‌کنیم که برای به دست آوردن اندازه بزرگ‌ترین خوشه به هم پیوسته از مکان‌های فعال از الگوریتم جستجوی سطح اول\LTRfootnote{Breadth First Search} (‌BFS) بهره گرفته‌ایم. این الگوریتم برای پیدا کردن خوشه‌ها در ساختار‌های درختی و گراف استفاده می‌شود. برای پیاده‌سازی این الگوریتم از صف استفاده می‌شود. یک صف برای نگه‌داشتن راس‌های همسایه استفاده می‌شود. 
   
   روش کار مبنتی بر این است که ابتدا یکی از راس‌ها را به طور تصادفی برای شروع الگوریتم انتخاب می‌کنیم و آن را در صف به عنوان ریشه قرار می‌دهیم. سپس همه راس‌هایی که با آن همسایه هستند و نیز تا به حال ملاقات نشده‌اند را پیدا می‌کنیم و آن را در  صف قرار می‌دهیم. در این زمان راسی که به عنوان ریشه انتخاب شده بود را از صف حذف می‌کنیم. در مرحله بعد از اولین راسی که در صف قرار دارد شروع به بازدید راس‌های همسایه و ملاقات نشده می‌کنیم و تمام راس‌هایی که  در این مرحله بازدید و  مشخص شده‌اند را در انتهای صف قرار می‌دهیم. به همین روال از دومین  راس در صف شروع کرده و راس قبلی را از صف حذف و راس‌های جدید را به آن اضافه می‌کنیم. به این ترتیب همه راس‌های همسایه سطح به سطح در گراف پویش می‌شوند. در خاتمه، همه  راس‌هایی که در صف قرار می‌گرفتند را شمارش کرده و تعدادشان را به عنوان اندازه خوشه ذخیره می‌کنیم. فرایند تا جایی پیش می‌رود که همه راس‌ها در گراف پیموده شوند. از میان خوشه‌هایی که به دست می‌آید، خوشه‌ای که اندازه‌اش بزرگ‌تر از باقی خوشه‌ها باشد به عنوان بزرگ‌ترین خوشه فعال به هم پیوسته گزارش شده است  \cite{zu,newman}.
   
   
   در گراف
    $\textsc{ER}$
     نتایجی که از تراوش استاندارد بدست آمده است در شکل (\ref{fig:SP}) نشان داده می‌شود. در این نمودار با تغییر احتمال 
$\textsc{p}$، مکان گذار جابه‌جا می‌شود. از محاسبات تحلیلی نشان داده شده است در گراف 
 $\textsc{ER}$ 
 مکان گذار با معکوس میانگین درجه مشخص می‌شود. به عبارت دیگر گذار دقیقا بر روی معکوس میانگین درجه‌اش اتفاق می‌افتد \cite{newman}. نتایج شبیه‌سازی دو منحنی که در شکل  (\ref{fig:SP}) به دست آمده است تایید کننده نتایج تحلیلی است. به روشنی مشخص است برای شبکه‌ای که میانگین درجه‌اش
  $\left\langle \textsc{k} \right\rangle  = 5$
   است (منحنی $1$)، نقطه گذار در $0.2$ و برای شبکه‌ای که میانگین درجه‌اش
    $\left\langle \textsc{k}\right\rangle  = 2$
     است (منحنی $2$)، نقطه گذار بر روی $0.5$ قرار دارد. منحنی‌های به دست آمده برای شبکه‌ای با تعداد راس‌های 
     $\textsc{N} = 10000$
      رسم شده است. 
      $\textsc{E}$
       در اینجا تعداد میانگین‌گیری است که  بر روی  $100$  نمونه انجام گرفته است. 
   \begin{figure} [htbp]
   \centering
   \includegraphics[width=11cm , height=7cm]{standard.eps} 
   \caption[نمودار مربوط به اندازه بزرگ‌ترین خوشه در گراف  
   $\textsc{ER}$
   ] {\vspace{-0.01}\footnotesize نمودار مربوط به اندازه بزرگ‌ترین خوشه در گراف 
    $\textsc{ER}$ 
    با 
    $\textsc{N}=10000$
     راس مربوط به تراوش استاندارد. منحنی‌ $1$ برای گراف با میانگین درجه
      $\left\langle \textsc{k} \right\rangle = 5$ 
      و منحنی $2$ برای گراف با میانگین درجه 
      $\left\langle \textsc{k} \right\rangle = 2$ 
      رسم شده است. همانطور که می‌بینیم نقطه گذار برای هر نمودار با معکوس میانگین‌ درجه‌اش متناسب است.}
   \label{fig:SP}
   \end{figure}
 
 
 
 \newpage
 \section{معرفی تراوش خود‌راه‌انداز}
 تراوش خودراه‌انداز به عنوان مدلی برای انتشار  فعال‌سازی مکان‌های یک شبکه مطرح شده است \cite{cohen}. از جمله کاربرد‌های این مدل می‌توان به شبکه‌های نورونی \cite{ec}، شبکه‌های مغناطیسی \cite{baxter}، اجتماعی \cite{camp} و غیره اشاره کرد.  
مشابه با  مدل تراوش استاندارد در تئوری تراوش خود‌راه‌انداز در حالت اولیه، مکان‌ها با یک احتمال اولیه $\textsc{f}$ با یک پیکربندی اولیه فعال می‌شوند. مکان‌هایی که فعال شدند تا پایان فرایند در همان حالت باقی می‌مانند. در مدل خود‌راه‌انداز با استفاده از یک قاعده فعال‌سازی نقاطی که غیر فعال باقی ماندند را فعال می‌کنیم؛ به این شکل که هر مکان غیرفعال در صورتی که ‌حداقل
 $\textsc{m}$ 
 تا از نزدیک‌ترین همسایه‌هایش فعال باشند، فعال می‌‌شوند. 
 $\textsc{m}$
   آستانه فعالیت برای اشغال شدن مکان‌های تهی است. این فرایند تا جایی پیش می‌رود که هیچ مکانی قابلیت فعال شدن را نداشته باشد. اگر مکان‌های فعال شبکه هم‌مرتبه با کل نقاط شبکه شوند می‌گوییم که تراوش در شبکه اتفاق افتاده‌ است (شکل \ref{fig:boot}). نتایج این مدل  تابع بعد شبکه 
   $(\textsc{d})$، احتمال اولیه 
   $(\textsc{f})$ 
   و پارامتر آستانه فعالیت 
   $(\textsc{m})$
    است \cite{grav}. 
 
 \begin{figure}[htbp]
 \hspace*{0cm}
 \centering
 %\begin{minipage}[b]{0.4\textwidth}
 \includegraphics[width=0.3\linewidth, height=35mm]{boot1.png}\centering(الف)    
 \includegraphics[width=0.3\linewidth, height=35mm]{boot2.png}\centering(ب)
 \includegraphics[width=0.3\linewidth, height=35mm]{boot3.png}\centering(ج)
 \caption[نمایش تراوش خودراه‌انداز مربوط به شبکه مربعی] {\footnotesize نمایش تراوش خودراه‌انداز  مربوط به شبکه مربعی با آستانه فعالیت 
 $\textsc{m} = 2$
 : (الف) موقعیت اولیه شبکه، مکان‌های فعال اولیه با با مربع‌های سیاه‌رنگ مشخص شده اند; (ب) مکان‌هایی که می‌توانند با $2$ همسایه فعال شوند با علامت ضربدر مشخص شده‌اند; (ج) مکان‌های مشخص شده نیز فعال شده‌اند \cite{kozma}.}
 \label{fig:boot}
 \end{figure}
 
 در یک شبکه 
 $\textsc{d}$
  بعدی، یک نقطه گذار در  
  $\textsc{f}_{\textsc{c}}$
   وجود دارد که تابع بعد شبکه و آستانه فعالیت است 
   $\textsc{f}_{\textsc{c}} = \textsc{f}(\textsc{d,m})$
   . در این مقدار، زمانی که
     $\textsc{f} > \textsc{f}_\textsc{{c}}$
      تراوش اتفاق می‌افتد و برای 
      $\textsc{f} < \textsc{f}_{\textsc{c}}$
       تراوش اتفاق نمی‌افتد \cite{kozma}. نقطه 
       $\textsc{f}_{\textsc{c}}$
       ، نقطه‌ای است که در آن خوشه شروع به شکل گرفتن می‌کند. کمتر از نقطه گذار 
       $\textsc{f}_{\textsc{c}}$
        جزیره‌های کوچکی از مکان‌های فعال وجود دارد که به هم متصل نیستند و به همین دلیل خوشه به هم پیوسته در شبکه دیده نمی‌شود. اما بعد از آن جزیره‌ها به هم وصل شده و تشکیل خوشه بی‌نهایت را می‌دهند.
 
   در شبکه مربعی درجه هر مکان چهار است. یعنی هر مکان به چهار همسایه چپ و راست، بالا و پایین خود متصل است. هر مکان دو حالت می‌پذیرد: فعال و یا غیر فعال. ابتدا تعدادی از مکان‌ها را با احتمال اولیه 
   $\textsc{f}$ فعال می‌کنیم\RTLfootnote{در شبکه مربعی می‌توانیم به جای کلمه فعال کردن از اشغال شدن مکان‌ها نیز استفاده کنیم.}. همانطور که در فصل قبل نیز گفته شد، در این حالت با نسبت دادن هر مکان به یک عدد تصادفی و مقایسه آن عدد با احتمال اولیه مکان‌ها فعال می‌شوند. بنابراین کسری از مکان‌های فعال اولیه ($\textsc{f}$)در شبکه خواهیم داشت. در مرحله بعد با استفاده از قوانین تراوش خودراه‌انداز دینامیک را روی شبکه اثر می‌دهیم. یک مکان غیر فعال را گزینش می‌کنیم. با در نظر گرفتن یک آستانه‌ی فعالیت 
   $\textsc{m}$ 
   برای مکان‌ها، در صورتی که مکان غیر فعال در نزدیک‌ترین همسایه‌های خود به تعداد حداقل 
   $\textsc{m}$
    مکان فعال داشته باشد می‌تواند فعال شود. فرایند تا جایی پیش می‌رود که مکانی شرط فعال شدن برایش وجود نداشته باشد. بعد از اتمام فرایند، دو کار را بررسی می‌کنیم. اول اینکه اندازه  تعداد کل نقاط فعال در شبکه را  بر حسب تغییرات نقاط فعال اولیه  به دست می‌آوریم و سپس از میان کل نقاط فعال، آن نقاطی  که با یکدیگر همسایه هستند و از به هم پیوستن آن‌ها خوشه بزرگ در شبکه تشکیل می‌شود را پیدا می‌کنیم.  
 
 نمودار (\ref{fig:lattice})  منحنی‌ مربوط به اندازه خوشه به هم پیوسته را برای یک شبکه با اندازه‌های متفاوت نشان می‌دهد. پارامتر 
  $\textsc{S}_{\textsc{gc}}$
   و
    $\textsc{E}$  به ترتیب بزرگ‌ترین خوشه به هم پیوسته و تعداد آنسامبل‌ها را مشخص می‌کند. نمودار تعیین شده برای آستانه‌ی فعالیت 
  $\textsc{m} = 3$
   با تعداد آنسامبل $\textsc{E}=100$ 
     رسم شده است. با توجه به این نمودار می‌بینیم که در این مورد نیز مانند حالت تراوش استاندارد با افزایش اندازه شبکه نقطه گذار به نقطه گذار واقعی نزدیک می‌شود و نیز گذار پیوسته را نیز در رفتار خوشه مشاهده می‌کنیم. 
 
 \begin{figure}[htbp]
 \hspace*{0cm}
 \centering
 %\begin{minipage}[b]{0.4\textwidth}
 %\includegraphics[width=0.4\linewidth,height=55mm]{lattice2.eps}\centering(الف)  
 \includegraphics[width=11cm , height=8cm]{lattice3.eps}
 \caption[نمودار بزرگترین خوشه به هم پیوسته مربوط به تراوش  خودراه‌انداز برای شبکه مربعی] {\footnotesize  نمودار بزرگترین خوشه به هم پیوسته مربوط به تراوش  خودراه‌انداز برای شبکه مربعی با اندازه‌های متفات برای آستانه‌ی فعالیت  
 $\textsc{m}=3$.}
 \label{fig:lattice}
 \end{figure}
 


\section{تراوش خودراه‌انداز بر روی شبکه نورونی}
 بعد از بررسی تراوش استاندارد در شبکه تصادفی در این بخش به معرفی و عملکرد تراوش خودراه انداز در شبکه نورونی می‌پردازیم. 
 در شبکه نورونی، با اعمال تحریک خارجی اولیه به شبکه، مجموعه‌ کوچکی از نورون‌ها فعال می‌شوند که این تعداد می‌تواند منجر به فعال شدن تعداد زیادی از نورون‌ها در شبکه شوند. مدل تراوشی که در بررسی این شبکه‌ها مورد بررسی قرار می‌گیرد از نوع خودره‌انداز است. در این مدل همان‌طور که برای شبکه مربعی در نظر گرفته‌ایم فعال شدن نورون‌ها وابسته به همسایه‌های فعالش است. با این تفاوت که در شبکه مربعی یک راس تنها با چهار همسایه خود در ارتباط است و همسایه‌ها نیز مکانشان مشخص است، اما در شبکه‌های نورونی تعداد همسایه‌های فعال نامعلوم و تصادفی جایگزیده‌اند.
 

 شبکه نورونی، شبکه بزرگی شامل اتصالات بسیار زیاد از نورون‌هاست. در این مجموعه بزرگ گروه‌های کوچکی وجود دارند که در آن نورون‌ها به یکدیگر متصل هستند و فعالیت می‌کنند. فعالیت این نورون‌ها نه تنها تاثیرپذیر از همان مجموعه کوچک است بلکه ممکن است ناشی از نورون‌هایی باشد که در خارج از این جمعیت هستند و نیز به طور مستقیم به جمعیت کوچک نورون‌ها متصل نیستند. به موجب این امر می‌توان با بررسی ارتباط‌هایی که بین نورون‌ها وجود دارد کمیت‌هایی از شبکه از جمله میانگین اتصال هر نورون ($\left\langle \textsc{k} \right\rangle$)، توزیع درجه (
 $\textsc{P}_{\textsc{k}}$
 ) و بزرگترین خوشه فعال (
 $\textsc{g}$
 ) در شبکه را بدست آورد \cite{sori}. 
 تراوش خودراه‌انداز در شبکه‌ نورونی زمانی اتفاق می‌افتد که برای آتش کردن یک نورون حتما باید در اطرافش نورون‌های فعال به اندازه‌ای وجود داشته باشد تا بتوانند آن نورون را فعال کنند. در این مدل با یک ولتاژ خارجی تعدادی از نورون‌ها فعال می‌شوند. این تعداد تا زمانی که شبکه به حالت پایای خود برسد فعال باقی می‌مانند. نورون‌های فعال در حالت اولیه به وسیله مکانیزم فعال‌سازی قادر به فعال کردن همسایه‌های غیر فعال هستند. در این حالت نیز وجود نقطه گذار که رفتار شبکه را در آن نقطه مشخص می‌کند، وجود و عدم وجود خوشه فعال به هم پیوسته را نشان می‌دهد.
 دو عامل در شبکه وجود این گذار را برای ما مشخص می‌کنند. عامل اول به اتصال بین نورون‌ها مربوط است و پارامتر کنترل تعریف می‌شود. پارامتر کنترل در واقع تعداد ورودی‌های لازم برای برای فعال کردن یک نورون غیر فعال در نظر گرفته می‌شود. در ادامه از آستانه فعالیت به جای پارامتر کنترل استفاده می‌کنیم. با این پارامتر ارتباط بین اجزا در شبکه کاهش  یا افزایش می‌یابد. زمانی‌ که آستانه فعالیت افزایش می‌یابد، به دلیل آنکه نورون غیر فعال نمی‌تواند  تعداد 
 $\textsc{m}$ 
 نورون فعال در اطراف خود پیدا کند خاموش  می‌ماند. به همین منوال این شرط برای نورون‌های دیگر نیز ارضا نمی‌شود.  با ادامه این روند از آنجایی که نورون‌ها غیر فعال باقی می‌مانند  اتصالی بین آن‌ها برقرار نمی‌شود و ارتباط بین آن‌ها کا‌هش می‌یابد. در حالی که برای 
 $\textsc{m}$های پایین به دلیل آنکه شرط فعال شدن برای نورون‌های غیر فعال راحت‌تر می‌شود، نورون‌های بیشتری فعال می‌شوند و بنابراین ارتباط افزایش می‌یابد. عامل دوم تعداد نورون‌هایی است که به تحریک پاسخ می‌دهند تا بتوانند فعال شوند و آن را به عنوان پارامتر نظم تعریف می‌کنیم. از آنجایی که پارامتر نظم، دو محیط با فاز‌های مختلف را نشان می‌دهد، حضور یک جهش ناگهانی  در این پارامتر به معنای وجود بزرگترین خوشه فعال به هم پیوسته خواهد بود.  
 
 
\subsection{بررسی نتایج تجربی}
چیدمان آزمایشگاهی برای بدست آوردن نتایج مورد نظر از بافت‌های عصبی اولین بار توسط پاپا\LTRfootnote{Papa} در آزمایشگاه راه‌اندازی شد\cite{papa}. برای انجام آزمایش، قسمتی از بافت‌های جوان از نورون‌های هیپوکمپ\LTRfootnote{hippocampal} موش گرفته شده و آن را به مدت $3 - 2$ هفته در یک محفظه شیشه‌ای پرورش می‌دهند تا ارتباط میان نورون‌ها در بافت شکل گیرد. دو الکترود  به دو طرف بافت متصل است و پالس‌های دو قطبی به اندازه 
$20 \textsc{msec}$
 به بافت‌ها از طریق الکترود‌ها وارد می‌شود و به سبب آن نورون‌ها تحریک می‌شوند\RTLfootnote{پالس دو قطبی یک پالس منفی و مثبت است که قسمت منفی باعث مهار و قسمت مثبت باعث تحریک ولتاژ می‌شود. به عبارت دیگر با اعمال پالس مثبت و منفی نورون‌ها به آستانه فعالیت نزدیک و یا از آن دور می‌شوند.} \cite{sorian}.
با اعمال هر پالس، جریان کنترل و ولتاژ به تدریج بین هر دو پالس افزایش می‌یابد. فعالیت نورون‌ها توسط اسیلوسکوپ فلئورسنس اندازه گیری می‌شود. مزیت استفاده از این اسیلوسکوپ دقت بسیار زیاد، بی تاثیر بودن اختلال بر روی آن و آسیب نرساندن به بافت‌های نورونی‌ است. اسیلوسکوپ فلئورسنس با شناساگر کلسیم،‌ شارش جریان و تغییرات ولتاژ در قسمت‌های مختلف بافت را به وضوح نمایش می‌دهد (شکل \ref{fig:setop}). 
\begin{figure}[htbp]
\hspace*{0cm}
\centering
%\begin{minipage}[b]{0.4\textwidth}
\includegraphics[width=0.3\linewidth, height=45mm]{setap.png}
\includegraphics[width=0.3\linewidth, height=45mm]{felo.png}    
\caption[تصویرسازی فلئورسنس از یک بافت نورونی و شمایی از چیدمان آزمایشگاهی] {\footnotesize (a) تصویرسازی فلئورسنس از یک بافت نورونی. قسمت‌های روشن جسم سلولی، اتصال‌های نورونی و شاخه‌های دندریتی را نشان می‌دهد. (b) شمایی از چیدمان آزمایشگاهی \cite{sorian}.}
\label{fig:setop}
\end{figure}

شبکه نورونی متشکل از نورون‌های تحریکی و مهاری است. نورون‌های تحریکی باعث تحریک ولتاژ و نورون‌های مهاری باعث کاهش ولتاژ نورون‌های دیگر می‌شوند. نورون پس‌سیناپسی دارای گیرنده‌هایی است که اطلاعات را از نورون‌های پیش‌سیناپسی دریافت می‌کنند. این گیرنده‌ها همانطور که در گذشته ذکر شده است دارای دو نوع تحریکی و مهاری است.
 $\textsc{NMDA}$
 \LTRfootnote{N-methyl-D-aspartete} و 
 $\textsc{AMPA}$
 \LTRfootnote{alpha-amino-3-hydroxy-5-methyl-4-isoxazolepropionic acid} دو نوع گیرنده تحریکی و 
 $\textsc{GABA}$
 \LTRfootnote{Gamma-aminobutyric acid} گیرنده نوع مهاری شناخته شده است. در طی انجام آزمایش محققان برای تقویت و تضعیف قدرت سیناپسی نورون‌ها از موادی برای مسدود کردن و یا فعال کردن گیرنده‌ها استفاده می‌کنند. از جمله موادی که برای تضعیف و مسدود کردن گیرنده‌های تحریکی 
 $\textsc{AMPA}$
  مورد استفاده قرار می‌گیرد اعمال   $\textsc{CNQX}$ \LTRfootnote{6-cyano-7-nitroquinoxaline-2,3-dione} به بافت‌های نورونی است. از طرفی گیرنده‌های مهاری 
  $\textsc{GABA}$
   نیز توسط $\mu M$40
بایکوکولین\LTRfootnote{biguguline} (که در واقع غیر فعال کننده گیرنده
 $\textsc{GABA}$
  است) مسدود می‌شوند. اعمال   $\textsc{CNQX}$  به شبکه موجب تغییراتی در اتصال و نیز آستانه فعالیت نورون‌ها می‌شود. واکنش شبکه به غلظت   $\textsc{CNQX}$  داده شده به آن به عنوان کسری از نورون‌های $\phi$ تعریف می‌شود که این نورون‌ها با تحریک خارجی فعال می‌شوند. با توجه به شکل (\ref{fig:CNQX}) مشاهده می‌کنیم در حالتی که شبکه کاملا به هم متصل است متناسب با
   $\textsc{CNQX} = 100\textsc{nM}$
    و زمانی که نورون‌ها کاملا جداگانه‌ از هم رفتار می‌کنند
     $\textsc{CNQX} = 700\textsc{nM}$ 
     می‌باشد \cite{sori}.
\begin{figure} [htbp]
\centering
\includegraphics[width=9cm , height=3cm]{CNQX.png} 
\caption[تاثیر   $\textsc{CNQX}$  بر روی بافت‌های نورونی] {\vspace{-0.01}\footnotesize تاثیر   $\textsc{CNQX}$  بر روی بافت‌های نورونی. در غلظت‌های پایین اندازه خوشه هم‌مرتبه با کل شبکه است و به تدریج در غلظت‌های بالا اندازه خوشه کم می‌شود \cite{sori}.}
\label{fig:CNQX}
\end{figure}

 زمانی که غلظت   $\textsc{CNQX}$  پایین است، همه نورون‌ها به هم متصل هستند و بنابراین کمترین ولتاژ تحریکی به شبکه باعث فعال شدن مجموعه‌ی بزرگی از نورون‌ها می‌شود. در این زمان رشد ناگهانی‌ در  تعداد نورون‌های فعال شبکه دیده می‌شود و شبکه را از حالتی که در آن هیچ نورونی فعال نیست و یا تعداد بسیار کمی از آن‌ها فعالند به سمت حالتی با تعداد نورون‌های فعال هم مرتبه با خود شبکه سوق می‌دهد. این پرش بزرگ نمایشگر اندازه بزرگترین خوشه به هم پیوسته از نورون‌های فعال در شبکه است. در مقابل زمانی‌ که غلظت   $\textsc{CNQX}$  به تدریج افزایش می‌یابد به علت تضعیف قدرت سیناپسی بین نورون‌ها، از تعداد نورون‌های متصل به هم کاسته شده و نورون‌ها به شکل جداگانه از هم رفتار می‌کنند. از آنجایی که این نورون‌ها نمی‌توانند از نورون‌های فعال اطراف خود ورودی دریافت کنند، بنابراین برای فعال شدن آن‌ها احتیاج به ولتاژ بالا می‌باشد، به اندازه‌ای که ولتاژ هر نورون به ولتاژ آستانه فعالیت خود برای آتش کردن برسد. به همین خاطر تعدادی از نورون‌ها ممکن است با ولتاژ ورودی کم و تعدادی نیز با ولتاژ ورودی زیاد به آن مقدار آستانه برسند. اما در مجموع همه نورون‌ها در یک مقدار خاص میانی به آستانه می‌رسند. با در نظر گرفتن این موضوع و نتایجی که از داده‌های تجربی بدست آمد (شکل \ref{fig:s(a)})، مشاهده می‌کنیم که در غلظت‌های بالا می‌توان توزیع آستانه فعالیت برای نورون‌ها را گاوسی در نظر گرفت. محققان طی انجام آزمایشات متعدد نشان داده‌اند که داده‌های به دست آمده با تابع خطا\LTRfootnote{error function} به شکل
  $\phi(v) = 0.5 + 0.5erf(\dfrac{v-v_{0}}{\surd2\times\sigma})$ 
  قابل برازش است. این تابع آستانه آتش کردن نورون‌ها را به صورت تابعی از یک توزیع گاوسی با میانگین
   $\textsc{v}_{0}$
    و پهنای
     $2\sigma$ 
     نشان می‌دهد.
      $v_{0}$ 
       آستانه فعالیت میانگین نورون‌ها است. بنابراین توافق شده است که توزیع گاوسی برای آستانه فعالیت نورون های شبکه نورونی توزیعی قابل قبول و مناسبی است.  
       $\phi_{v}$ 
       نیز تعداد نورون‌هایی ا\phiست که به تحریک خارجی 
       $v$ 
       برای فعال شدن پاسخ می‌دهند \cite{sorian}.

\begin{figure}[htbp]
\hspace*{0cm}
\centering
%\begin{minipage}[b]{0.4\textwidth}
\includegraphics[width=0.45\linewidth, height=55mm]{data2.png}\centering(الف)   
\includegraphics[width=0.4\linewidth, height=50mm]{giant.png}\centering(ب)
\caption[واکنش شبکه به اعمال   $\textsc{CNQX}$ ] {\footnotesize
 (الف) واکنش شبکه به اعمال   $\textsc{CNQX}$ . در غلظت‌های پایین تعداد نورون‌هایی که به تحریک پاسخ می‌دهند تا فعال شوند زیاد است و بنابراین شبکه به یکباره فعال می‌شود. اما با افزایش غلظت   $\textsc{CNQX}$  چون نورون‌ها جدا از هم رفتار می‌کنند و اتصال بین آن‌ها ضعیف می‌شود شبکه دیرتر فعال می‌شود. اندازه پرش در  منحنی‌ها اندازه بزرگ‌ترین خوشه به هم پیوسته از نورون‌ها را نشان می‌دهد. (ب) نمودار مربوط به اندازه بزرگترین خوشه به هم پیوسته بر حسب غلظت   $\textsc{CNQX}$ . در غلظت‌های پایین اندازه خوشه برابر با یک است و به تدریج با افزایش آن اندازه خوشه کوچک شده و در نهایت در غلظت‌های بالا به صفر می‌رسد \cite{sori}.}
\label{fig:s(a)}
\end{figure}

آزمایشاتی که بر روی بافت‌های نورونی صورت می‌گیرد در واقع برای مشاهده و اندازه‌گیری بزرگترین خوشه به هم پیوسته از نورون‌های فعال در شبکه است. این خوشه بزرگترین کسر از نورون‌های فعال در شبکه است که با یکدیگر نسبت به تحریک وارد شده به آن‌ها فعال می‌شوند و به یکبارهتمام شبکه را پوشش می‌دهند. با توجه به قسمت (الف) شکل (\ref{fig:s(a)}) تاثیر   $\textsc{CNQX}$  را بر روی بافت‌های نورونی مشاهده می‌کنیم. تزریق   $\textsc{CNQX}$  به بافت‌ها اتصال بین آن‌ها را تضعیف کرده و نورون‌ها به تدریج از هم جدا می‌شوند. کاسته شدن اتصال بین نورون‌ها به معنای کاهش اندازه خوشه به هم پیوسته در شبکه تلقی می‌شود و هنگامی که غلظت   $\textsc{CNQX}$  خیلی زیاد شود نورون‌ها کاملا از هم جدا هستند و بنابراین خوشه‌ای در شبکه وجود نخواهد داشت. در مقابل برای غلظت پایین و نزدیک به صفر   $\textsc{CNQX}$   که شبکه کاملا به هم متصل است، اندازه خوشه بزرگ‌ترین مقدار خود را دارد. در این زمان گذار گسسته‌ای را در رفنار خوشه می‌بینیم. در این هنگام زمانی که نورون‌ها هیچ‌گونه فعالیتی ندارند سایز خوشه هم‌مرتبه با کل شبکه در نظر گرفته می‌شود. در قسمت  (ب) شکل (\ref{fig:s(a)})  مشاهده می‌کنیم که حضور خوشه در شبکه ناشی از غلظت پایین   $\textsc{CNQX}$  است. در این مقدار اندازه خوشه بزرگ‌ترین مقدار خود را دارد. کوچک شدن جهش در منحنی واکنش به منزله کاهش اندازه خوشه می‌باشد. بنابراین می‌توان گفت زمانی که اندازه خوشه به صفر رسیده است به علت اعمال   $\textsc{CNQX}$  در غلظت‌های بالا، اتصال‌ بین نورون‌ها از بین رفته و به موجب آن خوشه‌ای در شبکه دیده نمی‌شود. این امر به وضوح از نمودار‌های فوق قابل مشاهده است.


\subsection{تاثیر \textsc{CNQX} بر آستانه فعالیت}
آستانه فعالیت، تعداد نورون‌های فعال مورد نیاز برای آتش کردن یک نورون غیر فعال می‌باشد. طبق گفته‌های پیشین افزایش   $\textsc{CNQX}$  و نیز کاهش قدرت سیناپسی بین نورون‌ها سبب جدا شدن نورون‌ها از یکدیگر می‌شود. به موجب این امر، تعداد نورون‌های فعال در شبکه به تدریج کم می‌شود. در این صورت از تعداد نورون‌های فعال در همسایگی نورون غیر فعال نیز کم می‌شود. بنابراین از آنجایی که نورون‌ها به سختی قادر خواهند بود تا با شرط آستانه بالا خود را فعال کنند، رفته رفته از تعداد نورون‌هایی که قادر به فعال شدن هستند کم و اندازه خوشه‌ای که تولید می‌شود به تدریج کوچک می‌شود. بنابراین هنگامی که آستانه فعالیت برای نورون‌ها را بالا می‌بریم شبکه سخت‌تر فعال شده و به نسبت آن اندازه خوشه نیز کاهش می‌یابد.
 \begin{figure} [htbp]
\centering
\includegraphics[width=12cm , height=5cm]{ss.png} 
\caption[تاثیر آستانه فعالیت بر اندازه خوشه] {\vspace{-0.01}\footnotesize تاثیر آستانه فعالیت بر اندازه خوشه. دایره‌های روشن نورون‌های فعال و تیره نورون‌های غیرفعال و فلش‌ها ارتباط سیناپسی بین نورون‌ها را نشان می‌دهند. آستانه فعالیت برای این شبکه
 $\textsc{m} = 2$
  در نظر گرفته شده است.  با افزایش
   $\textsc{m}$
    می‌بینیم که تعداد نورون‌های فعال به تدریج کم شده و اندازه خوشه نیز به سمت صفر پیش می‌رود. برای 
    $\textsc{m} = 8$
     می‌بینیم که هیچ نورون فعال و خوشه‌ای در شبکه وجود ندارد \cite{cohen}.}
\label{fig:ss}
\end{figure}


همانطور که  از شکل (\ref{fig:ss}) مشاهده می‌کنیم زمانی که آستانه فعالیت کوچک است و نیز ساختار شبکه به طور کامل شکل گرفته است، به خاطر حضور تعداد نورون‌های فعال فراوان، یک نورون غیر فعال به راحتی می‌تواند آتش کند. آتش کردن یک نورون می‌تواند منجر به فعال شدن کل شبکه شود. در این زمان بزرگترین خوشه فعال در شبکه را از نورون‌های فعال خواهیم دید. اما زمانی که ارتباط سیناپسی بین نورون‌ها کاهش یابد (به خاطر حضور   $\textsc{CNQX}$ )، آستانه فعالیت 
$\textsc{m}$
 افزایش پیدا می‌کند. تا زمانی که آستانه فعالیت به مقدار بحرانی خود نرسد، می‌توانیم خوشه بزرگ را در شبکه اندازه‌گیری کنیم. اما بعد از آن به خاطر افزایش مقدار   $\textsc{CNQX}$   نورون‌ها از یکدیگر جدا می‌شوند و خوشه‌های مجزا با اندازه‌ای کوچک‌تر در شبکه به وجود می‌آید؛ و در نهایت در
 $\textsc{m}$های بزرگ (تزریق مقدار بسیار زیاد   $\textsc{CNQX}$ ) ارتباط سیناپسی بین نورون‌ها کاملا قطع شده و خوشه‌ای در شبکه شکل نمی‌گیرد. 
 
در فصل بعد مدل سازی شبکه نورونی و نیز نتایجی را که از شبیه‌سازی بدست آورده‌ایم بررسی خواهیم کرد.

  
 
 
 

 \newpage 
\textbf{خلاصه‌ی فصل سوم}    \begin{itemize}
\item از خواص اصلی تراوش پیدا کردن تعداد کل نقاط فعال و نیز پیدا کردن بزرگترین خوشه به هم پیوسته از مکان‌‌های فعال در شبکه است.
\item تراوش پیوندی و جایگاهی دو نوع از متعارف‌ترین و معمول‌ترین تعریف تراوش  در حالت استاندارد و در دو بعد به حساب می‌آیند.
\item در هر دو نوع تراوش پیوندی و جایگاهی، شبکه با یک احتمال اولیه $\textsc{f}$ ساخته می‌شود و هدف ساختن چنین ترکیبی پیدا کردن مسیری است که مکان‌های فعال در شبکه را به هم مرتبط کند. $\textsc{f}$ احتمال مکان‌های فعال اولیه در شبکه است.
\item در تراوش پیوندی، یال‌ها و در تراوش جایگاهی، مکان‌ها با یکدیگر در ارتباط هستند. 
\item تراوش  خودراه‌انداز نوع دیگری از تراوش است که در آن مکان‌ها طی قاعده‌ای فعال می‌شوند. در این نوع، مکان‌های غیرفعال با نگاه‌کردن به همسایه‌های فعال خود و با در نظرگرفتن آستانه فعالیت، قادر به فعال شدن هستند.
\item شبکه نورونی از جمله شبکه‌هایی است که تئوری تراوش خودراه‌انداز در آن کاربرد دارد. با اعمال این  مدل بر روی شبکه بزرگترین خوشه فعال از نورون‌ها که با یکدیگر در ارتباط هستند مشخص می‌شوند. 
\item در کار تجربی که تا کنون انجام شده است، با اعمال   $\textsc{CNQX}$  سیناپس‌ها را تضعیف می‌کنند.
این ماده، روی ارتباط‌های نورونی‌ و نیز روی آستانه فعالیت نورون‌ها تاثیرگذار است. 
\end{itemize}








 


 









 
