
\textbf{•} % -------------------- DON'T EDIT ---------------------------------------
\documentclass[a4paper,12pt,oneside,openany]{xepersian-thesis-iasbs-ftex}
\usepackage[top=3.5cm,right=3cm,bottom=4cm,left=3cm]{geometry}  
\usepackage{amsmath}
 \usepackage[marginal,stable,bottom]{footmisc}     % for footnotes: marginal --> the same margins as text, 
                                                                                  %                       stable--> ?
                                                                                  %                       bottom --> starting the footnotes at a fixed place at the bottom of the page.
\usepackage{perpage}                                             % for footnotes: starting from 1 perpage
\MakePerPage{footnote}
\usepackage{cite}                                                     % for collecting citations: [1,2,3,4] --> [1-4]
\usepackage{setspace}                                            % for switching between double/single space in document
\allowdisplaybreaks                                                    % breaking the lines&pages when needed for the style.
\usepackage{parskip}                                               % ? 

%\relpenalty=9999                                            % show the neccessity of breaks for lines, changing the number to 10000 cause no break in lines.
%\binoppenalty=9999

\usepackage{xecolour}
\usepackage{amsmath}
\usepackage{makeidx}
\usepackage{verbatim}
\usepackage[colorlinks,linkcolor=blue,citecolor=blue]{hyperref}
\usepackage{graphicx}
\usepackage{ifthen}

 \parindent0pt


\makeatletter
\pdfstringdefDisableCommands{%
\let\lr\@firstofone
}
\makeatother
% ----------------------------------------------------------------------------
\usepackage{caption}
\usepackage{subcaption}

\usepackage[Lenny]{fncychap}

\usepackage{xepersian}

\settextfont[Scale=1.01]{XB Niloofar}
\setlatintextfont[Scale=0.89]{Times New Roman} 
\setdigitfont[Scale=1.05]{Yas} 
% قلم برای اعداد به صورت فارسی با صفر توخالی، در صورتی که بخواهیم اعداد انگلیسی نوشته شوند این خط را غیر فعال می‌کنیم.

\defpersianfont\nastaliq[Scale=2.0]{IranNastaliq}
% قلم برای نوشتن تقدیم
\defpersianfont\nastaliqone[Scale=1.0]{IranNastaliq}
% قلم نستعلیق با سایز متناسب با متن در صورت نیاز
\defpersianfont\anotherfont[Scale=1.2]{XP Ziba}
% قلم برای نوشتن تشکر (قلم فانتزی)
\newenvironment{fantezi}
{\anotherfont }


\makeindex


\def\beginto{
\newpage
\begin{RTL}
\nastaliq

\begin{center}
\vspace*{2.cm}
تقدیم به عاشقانه‌های زندگی‌ام \\\ 
}

\def\endto{
~
\end{center}
\end{RTL}
}

\def\beginthanks{
\newpage

{\centering\Huge{\nastaliqone{
سپاس... ~\\
~\\}}}
}

\def\endthanks{
~
}

\def\thanks{
\beginthanks
شکر شایان نثار ایزد منان که توفيق را رفیق راهم ساخت تا اين پايان نامه را به پايان برسانم.\\\
این پایان‌نامه را ضمن تشکر و سپاس بیکران و در کمال افتخار تقدیم می‌دارم به:\\
استاد عزیز و گران‌‌قدرم، سرکار خانم دکتر ناهید عظیمی\\
استاد فرهیخته‌ام جناب آقای دکتر علیرضا ولیزاده\\
پشتوانه‌های زندگی‌ام: پدر دلسوز، مادر فداکار، خواهران عزیزتر از جان و همراه صبور، همسرم،\\
و همراهان همیشگی‌‌ام در این مدت تحصیل: بهار عزیزم، اکرم دوست‌داشتنی و محمد انصاری‌آرای عزیز،\\
به پاس را‌هنمایی‌ها و حمایت‌های بی‌دریغشان در به تحقق رسیدن اهداف و اتمام این پروژه،\\
به پاس حضور گرم و محبت‌های بی‌چشم‌داشتشان که هرگز فراموش نمی‌شود،\\
به پاس قلب‌های بزرگ‌شان که همواره امید‌بخش من در این دوران سخت بوده‌اند،\\
 و به پاس صبر و حوصله بی‌حدشان که همیشه و در همه جا یاری‌گر من بوده‌اند.\\
\begin{center}
{\large  بوسه بر دستان پرمهرتان.}
\end{center}





\endthanks
}

\def\startpage{
\newpage
\vspace*{5cm}
\begin{center}
\includegraphics[width=15cm]{besm.jpg}
\end{center}
}
\makeatletter
\def\@makechapterhead#1{%
  \vspace*{50\p@}%
  {\parindent \z@ \centering\normalfont
    \ifnum \c@secnumdepth >\m@ne
      \if@mainmatter
        \huge\bfseries \@chapapp\space \tartibi{chapter} 
        \par\nobreak
        \vskip 20\p@
      \fi
    \fi
    \interlinepenalty\@M
    \Huge \bfseries #1\par\nobreak
    \vskip 40\p@
  }}
\def\@makeschapterhead#1{%
  \vspace*{50\p@}%
  {\parindent \z@ \centering
    \normalfont
    \interlinepenalty\@M
    \Huge \bfseries  #1\par\nobreak
    \vskip 40\p@
  }}
\makeatother

\def\by{توسط}
\def\writtenfor{‌پایان‌نامه‌ی‌‌‌}
\def\undersupervision{اساتید راهنما:}
\def\underadvising{استاد مشاور:}
\def\departmentof{دانشکدهٔ}
\def\universityof{دانشگاه}
\def\latinby{by}


{\includegraphics[width=7cm]{logo-fa.jpg}}
{\includegraphics[width=7cm]{logo-en.jpg}}




\renewcommand\bibname{مراجع}
\def\contentsname{فهرست}

% some extra diffinitions %%%%%%%%%%%%%%%%%
%%%%%%%%%%%%%%%%%%%%%%%%%%%%%

\def\bea{\begin{eqnarray}}
\def\eea{\end{eqnarray}}

\def\ba{\begin{array}}
\def\ea{\end{array}}

\def\ni{\noindent}
\def\nn{\nonumber}

\def\bc{\begin{center}}
\def\ec{\end{center}}

% ------------------- You Can EDIT ----------------------------------------
\graphicspath{{chap1_images/}{chap2_images/}{chap3_images/}{chap4_images/}}                       % The paths to the images.

% -------------------- DON'T EDIT ----------------------------------------

\begin{document}

% -------------------- PLEASE EDIT ---------------------------------------

% --------------------------------------   INFORMATION IN PERSIAN  --------------------

-------------------

\title{تراوش خودراه‌انداز در شبکه‌های‌ نورونی}
\author{سودابه اژدر}
%--------------------------------------------
% از بین دو حالت زیر یکی را فعال کنید:
\degree{کارشناسی ارشد}
%\degree{دکتری}
%---------------------------------------------
\supervisor{\begin{center}
دکتر ناهید عظیمی \\ دکتر علیرضا ولی‌زاده 
\end{center}}
% در صورتی که استاد مشاور داشته‌اید نام وی را در خط زیر بنویسید در غیر این صورت خط زیر باید غیرفعال باشد.
%\advisor{دکتر جعفر مصطفوی امجد} \advisorexisttrue
\department{فیزیک}
\university{دانشگاه تحصیلات تکمیلی علوم پایه زنجان}
\city{زنجان}
\thesisdate{مهر 1395}
\makepersiantitle
% -------------------------------------------------------------------------------------------------------------------
\startpage
% -------------------------------------------------------------------------------------------------------------------

\beginto
%%در اینجا پایان‌نامه خود را به هر کس که دوست دارید تقدیم کنید. در غیر این صورت این قسمت را غیر فعال کنید.
\begin{center}
%تقدیم به عاشقانه‌های زندگی‌ام:\\
که ناتوان شدند تا من به توانایی برسم،\\\

موهایشان سفید شد تا رو سفید شوم،\\\

و عاشقانه سوختند تا گرمابخش وجود من و روشنگر راه من باشند:\\\

										  پدر و مادر عزیزم\\
\end{center}
\endto
% -------------------------------------------------------------------------------------------------------------------
%اگر در فایل thanks از دوستان و همکاران خود تشکر کرده‌اید خط زیر را فعال کنید، در غیر این صورت این خط باید غیرفعال باشد.
\thanks
% -------------------------------------------------------------------------------------------------------------------
\newpage
\begin{abstract}
\addcontentsline{toc}{section}{چکیده} % To add the abstract in the index.
تئوری تراوش استاندارد،  به مطالعه رفتار  اندازه خوشه‌ها در  شبکه  می‌پردازد. یکی از مسائل مهم در تئوری تراوش یافتن اندازه و نقطه ظهورِ خوشه‌ی بی‌نهایت از مکان‌های اشغال شده است. یک مدل تعمیم‌یافته از تراوش استاندارد، مدلی است که به تراوش خودراه‌انداز معروف است. این مدل برای توصیف گسترش  فرایند فعال‌سازی روی شبکه‌ها معرفی شده است که در آن فعال‌سازی به شکل دسته‌جمعی و با فعال شدن متوالی مکان‌ها در شبکه ظاهر می‌شود.  فرایند فعال‌سازی در شبکه‌های نورونی مثالی بارز از تراوش خودراه‌انداز است. در این پایان‌نامه، فرایند فعال‌سازی نورون‌ها در شبکه نورونی را مورد بررسی قرار می‌دهیم. در یک شبکه عصبی زمانی که نورون به آستانه فعالیت می‌رسد و آتش می‌کند،‌ نورون‌های همسایه را تحت تاثیر خود قرار می‌دهد.  در مدل تراوش خودراه‌انداز، یک نورون در صورتی فعال خواهد شد که حداقل $m$ تا از نورون‌های همسایه آن فعال باشند. با آتش کردن نورون‌ها به طور دسته‌جمعی خوشه‌ای به هم پیوسته از نورون‌های فعال به دست می‌آید که کسر قابل توجهی از کل شبکه نورونی را به خود اختصاص می‌دهد. اندازه بزرگ‌ترین خوشه به هم پیوسته از مکان‌های فعال را با بهره‌گیری از روش‌های تراوش خودراه‌انداز و با استفاده از شبیه‌سازی به دست می‌آوریم. در این پایان‌نامه برای سادگی ابتدا مدل را بر روی شبکه تصادفی با توزیع درجه یکنواخت پیاده می‌کنیم و سپس آن را به یک شبکه با تابع توزیع درجه گاوسی که مناسب برای شبکه‌ی نورونی است تعمیم خواهیم داد. مشاهده خواهیم کرد که با افزایش تعداد نورون‌های فعال اولیه، تعداد نورون‌های فعال نهایی شبکه افزایش می‌یابد و در یک نقطه به طور ناگهانی رشد می‌کند. همچنین اندازه خوشه به هم پیوسته از نورون‌های فعال، گذار فازی را نشان خواهد داد که ترکیبی از گذار فازهای مرتبه اول و دوم است. این گذار فازها به ازای توابع توزیع یکنواخت و گاوسی برای آستانه‌های متفاوت بررسی می‌شود.

\keywords{ فعال‌سازی ، تراوش خودراه‌انداز ، بزرگ‌ترین خوشه فعال به‌ هم‌پیوسته ، گذار فاز}
\end{abstract}
% -------------------------------------------------------------------------------------------------------------------

\tableofcontents
\listoffigures
%\listoftables
% -------------------------------------------------------------------------------------------------------------------

% اگر قبل از شروع فصل‌های پایان‌نامه مقدمه‌ای دارید، خط زیر را فعال کرده و مقدمه خود را در فایل introduction بنویسید. در غیر این صورت این خط باید غیر فعال باشد.
\newpage
\pagenumbering{arabic}\setcounter{page}{1}
\newpage
\vspace*{-1cm}
\section*{پیش‌گفتار}
\addcontentsline{toc}{section}{پیش‌گفتار}
تئوری تراوش\LTRfootnote{Percolation}، به توصیف رفتار خوشه‌های به هم پیوسته در گراف‌های با ساختار تصادفی  می‌پردازد \cite{chris}. تاریخچه این نظریه به سال $1957$ برمی‌گردد، زمانی که برادبنت\LTRfootnote{Broadbent} و هَمرسلی\LTRfootnote{Hammersley} در پی جوابی برای یافتن مسیری برای عبور مایعات از میان مواد متخلخل بودند\cite{broad}.  نظریه تراوش را کم و بیش می‌توانیم در طبیعت ببینیم؛ مطالعه رشد بیماری در جامعه انسان‌ها، عبور اطلاعات از شبکه جهانی وب، آتش‌سوزی جنگل و مثال‌هایی از این نوع مساله تراوش را برایمان روشن می‌سازد \cite{hasen}. در واقع می‌توان گفت تراوش، الگوریتمی است که مسیر رشد اطلاعات را در یک شبکه نشان می‌دهد. این مسیر بزرگترین مسیر به هم پیوسته از اجزای  شبکه است. 

در بررسی مساله تراوش، به حالت تعمیم‌یافته‌ای از آن برمی‌خوریم که تراوش خودراه‌انداز نامیده می‌شود.  تراوش خودراه‌انداز در سال $1979$ با کار‌های چالوپا\LTRfootnote{Challpa}، لِث\LTRfootnote{Heath} و  ریچ\LTRfootnote{Reich} در جریان مطالعاتشان بر روی سیستم‌های بی‌نظم مغناطیسی شناخته شد. بعد از آن این نظریه به خاطر اهمیت در ارتباط‌ با مدل‌های فیزیکی و نیز کاربرد‌های مختلف  مانند فعالیت‌های  نورونی و گذار‌های ناگهانی مورد مطالعه فیزیک‌دانان و سایر علوم قرار گرفته است \cite{go}.  در این مدل، مکان‌ها به شکل تصادفی اشغال می‌شوند. در پی آن، مکان‌های غیر فعال با  شرط داشتن حداقل $m$ همسایه فعال، قادر خواهند بود به حالت فعال درآیند. این روند تا رسیدن شبکه به حالت پایا ادامه خواهد داشت؛ جایی که دیگر راسی نتواند فعال شود. 

از جمله کاربرد‌های تراوش در شبکه‌های نورونی است که با  استفاده از این تئوری انتقال  و رشد اطلاعات بین نورون‌ها بررسی می‌شود. نورون‌ها در یک شبکه عصبی دارای اتصال‌های قوی هستند و اطلاعات از طریق سیناپس‌ها به نورون‌ها منتقل می‌شوند. مغز انسان یکی از جالب توجه‌ترین سامانه‌های پیچیده محسوب می‌شود، اما به دلیل ساختار بسیار پیچیده و عدم توانایی در محاسبه این پیچیدگی، تا به امروز شناخت کاملی از شبکه مغز انجام نشده است\cite{stam}. 
مغز انسان به طور تقریبی از $100$ میلیارد نورون تشکیل شده است.  اتصال بین این نورون‌ها با یکدیگر یکی از ویژگی‌های اساسی برای درک و ساختار و مکانیزم شبکه نورونی است\cite{sorian}. 
در چند دهه اخیر شبکه‌های نورونی به دلیل شباهتشان به شبکه‌های پیچیده و گراف‌های تصادفی مورد توجه فیزیک‌دانان و ریاضی‌دانان قرار گرفته است. تئوری گراف، ساختار‌های پیچیده شبکه‌های واقعی و مصنوعی را از نقطه نظر مفاهیم اساسی شبکه‌های پیچیده به زبانی ساده بیان می‌کند \cite{boccara}. از جمله این مفاهیم می‌توان به خوشگی\LTRfootnote{Clustring} و انواع گذار‌ها در شبکه اشاره کرد. در شبکه‌های با ابعاد بی‌نهایت و تصادفی، یکی از ساده‌ترین روش‌ها برای نشان دادن این گونه مفاهیم تئوری تراوش بر روی گراف‌های تصادفی است\cite{sorian}. با استفاده از این تئوری، فعالیت بین نورون‌ها و نیز گسترش آن را در شبکه بررسی می‌کنیم.

در این پایان‌نامه فعال‌سازی شبکه نورونی بر روی گراف  تصادفی با دو توزیع درجه همگن و گاوسی مورد مطالعه قرار گرفته است و با استفاده از الگوریتم تراوش توانسته‌ایم اندازه بزرگ‌ترین خوشه به هم پیوسته از نورون‌های فعال در شبکه را بررسی کنیم. 

فصل اول رساله مروری است بر مفاهیم ابتدایی شبکه عصبی، شامل تعاریف اولیه نورون و اجرا آن و نیز تشریح فعالیت نورون‌ها در سیستم عصبی. فصل دوم اشاره‌ای به شبکه‌های پیچیده و مدل‌‌های آن دارد. در فصل سوم تراوش در حالت استاندارد روی شبکه مربعی، گراف تصادفی و  نیز نتایج تجربی که  روی شبکه نورونی  انجام شده است را بررسی کردیم و در نهایت در فصل آخر تئوری  تراوش خودراه‌انداز و شبیه‌سازی شبکه نورونی مورد بحث قرار گرفته است.







% ----------------------------------------   Chapters of the Thesis  ---------------------------------------------
\chapter{دستگاه عصبی مرکزی}
\section{مقدمه}
هماهنگی بین اعمال و اندام‌های بدن توسط دستگاه‌های ارتباطی که در بدن موجودات سلولی وجود دارد انجام می‌شود. سیستم عصبی با ساز و کار ویژه خود وظیفه این هماهنگی را بر عهده دارد. سلول‌های عصبی از مهم‌ترین و پیچیده‌ترین واحد پردازنده سیستم عصبی مرکزی هستند. اجزا و سازوکار این سلول از موضوعات اساسی مطالعه دستگاه عصبی به شمار می‌آید.
در این فصل دستگاه عصبی مرکزی را مورد بررسی قرار می‌دهیم.
\section{انواع سلول‌ها در بافت‌های عصبی}
دستگاه عصبی مرکزی به دو دسته تقسیم می‌شود:
\begin{itemize}\item سلول عصبی به نام نورون\LTRfootnote{neuron}  که  انتقال دهنده‌ پیام‌‌های عصبی به شمار می‌آید. این سلول‌ها به عنوان سلول‌های تحریک‌پذیر شناخته می‌شوند. نورون‌ها پیام‌های عصبی را به بافت‌ها، اندام‌ها و دیگر نورون‌ها می‌فرستند و از این طریق با آن‌ها ارتباط برقرار می‌کنند.
\item سلول‌های غیرعصبی به نام نوروگلیا\LTRfootnote{neuroglia}
 یا گلوسیت\LTRfootnote{gliocyte} که سلول‌های پشتیبان محسوب می‌شوند و وظیفه محافظت از نورون‌ها را بر عهده دارند. این سلول‌ها در انتقال پیام عصبی نقشی ندارند. سلول گلیا به صورت الکتریکی تحریک نمی‌شوند. نوروگلیا را سلول‌های تحریک‌ناپذیر نیز می‌گویند. 
\end{itemize}تعداد نورون‌ها در مغز انسان در حدود $100$
 میلیارد است و هر نورون به طور متوسط می‌تواند با $10$ هزار نورون دیگر ارتباط برقرار کند. در مقابل تعداد نوروگلیا چندین برابر نورون‌هاست ($5$ یا $10$ برابر تعداد نورون‌ها).
\section{ساختمان اصلی نورون}
هر نورون شامل سه بخش اصلی است: جسم سلولی\LTRfootnote{cell body}، دندریت\LTRfootnote{dendrite}، آکسون\LTRfootnote{axon} . شکل (\ref{fig:neuron}) این سه بخش را به طور واضح نشان می‌دهد.
\begin{figure} [htbp]
\centering
\includegraphics[width=9cm , height=5cm]{neuron.png} 
\caption[نورون و بخش‌های مختلف آن] {\footnotesize نورون و بخش‌های مختلف آن \cite{bear}.}
\label{fig:neuron}
\end{figure}
جسم سلولی یا سوما\LTRfootnote{soma} مرکز اصلی سلول عصبی می‌باشد. دندریت‌ها انشعابات درخت‌گونه هستند و در واقع قسمت اصلی دریافت اطلاعات و سیگنال‌هایی هستند که از دیگر نورون‌ها به نورون نوعی می‌رسد. آکسون نیز سیگنال‌های دریافتی را به سلول‌های دیگر می‌فرستد. طول آکسون در برخی موارد به دو متر نیز می‌رسد \cite{kandel}.
‌ %\subsection{تقسیم بندی نورون}
%نورون‌ها از نظر طرز خارج شدن تارهای عصبی از جسم سلولی به سه گروه تک قطبی\LTRfootnote{Unipolar}، دوقطبی\LTRfootnote{Bipolar} و چندقطبی\LTRfootnote{Moltipolar} تقسیم می‌شوند. این تقسیم بندی اولین بار توسط رامون کاخال\LTRfootnote{Ramon Cajal} انجام شده است \cite{kandel}.
%نورون‌‌های تک قطبی ساده‌ترین نوع نورون‌ها محسوب می‌شوند که از یک جسم سلولی و شاخه‌های متعدد با اندازه‌های یکسان تشکیل شده است. یکی از شاخه‌ها به عنوان آکسون و باقی آن‌ها دندریت نورون به شمار می‌آیند. چنین نورون‌هایی در ساختار موجودات بی‌مهره وجود دارند. در ساختار مهره‌داران در سیستم عصبی خودکار دیده می‌شوند. دستگاه عصبی خودکار، دسته‌ای از نورون‌های حرکتی هستند که فعالیت ماهیچه‌های صاف، تراوش غدد، تپش قلب و به طور کلی اندام‌های درونی را کنترل می‌‌کنند.

%نورون‌های دوقطبی دارای جسم سلولی بیضی‌گون هستند که دو شاخه از آن خارج می‌شود. یکی از آن‌ها مربوط به دندریت است که سیگنال‌ها را دریافت می‌کند و دیگری در نقش آکسون و حاوی اطلاعاتی است که آن‌ها را به طرف سیستم عصبی مرکزی می‌فرستد. بیشتر سلول‌های حساس از جمله سلول‌های چشم و بویایی در این دسته جای می‌گیرند. 

%نورون‌های چندقطبی در سیستم عصبی مهره‌داران دیده می‌شوند. آن‌ها نوعا دارای تک آکسون و دندریت‌های فراوان در اطراف جسم سلولی هستند. نورون‌های بدن انسان در این گروه قرار دارند.

%در بعضی قسمت‌های دستگاه عصبی، نورون‌هایی که فاقد آکسون هستند شناسایی شده‌اند که این نورون‌ها فقط قادرند تحریک عصبی را به نورون‌های مجاور خود منتقل کنند.

%نورون‌ها را بر مبنای عملکردشان می‌توان به سه دسته‌ی نورون‌های حسی\LTRfootnote{Sensory Neuron }، نورون‌های حرکتی\LTRfootnote{Motor Neuron} و نورون‌های رابط\LTRfootnote{Interneuron} تقسیم کرد.نورون‌های حسی پیام‌های عصبی را به طرف دستگاه عصبی هدایت می‌کنند. نورون‌های حرکتی پیام‌ها را از دستگاه عصبی مرکزی دریافت می‌کنند و آن‌ها را به سمت عضلات صاف و اسکلتی و قلبی می‌فرستند و در نهایت نورون های رابط یا نورون‌های واسطه، نورون‌هایی هستند که رابط بین نورون‌ها با یکدیگر در سیستم عصبی است. از جمله این نورون‌ها می‌توان به نورون‌های واسطه در قشر مغز اشاره کرد \cite{kandel}. 

%نمونه‌ نورون‌هایی را که در بالا ذکر شده است در شکل مشاهده می‌کنیم. \begin{figure} [htbp]	\centering	\includegraphics[width=9cm , height=6cm]{types.png} 	\caption{\footnotsize طبقه‌بندی نورون‌ها از نظر شکل و کارکرد \cite{kandel}.} \label{fig:types} \end{figure}

\section{فعالیت نورون}
سلول‌های زنده اختلاف پتانسیلی در دو طرف غشا\LTRfootnote{membrane} دارند.  قشر داخلی و خارجی  غشا توسط لایه چربی از هم جدا می‌شوند.  این لایه نسبت به عبور یون‌های موثری که در تولید پتانسیل عمل نقش دارند انتخابی\LTRfootnote{selective} است. در طول غشا کانال‌هایی وجود دارد که سبب شارش یون‌ها و در نتیجه باعث ایجاد اختلاف پتانسیل در دو سوی سلول می‌شوند. این پتانسیل، پتانسیل غشا نامیده می‌شود و از رابطه زیر بدست می‌آید \cite{ermen}:
\begin{equation}
V_{\textsc{M}} = V_{\textsc{in}} - V_{\textsc{out}}
\end{equation}
در این رابطه  $V_{\textsc{M}}$  اختلاف پتانسیل غشا، $V_{\textsc{in}}$ و 
$V_{\textsc{out}}$ به ترتیب پتانسیل داخل و خارج غشا هستند (شکل \ref{fig:potential}).
\begin{figure} [htbp]
\centering
\includegraphics[width=10cm , height=7cm]{potential.png} 
\caption[پتانسیل غشای سلول] {\footnotesize پتانسیل غشای سلول ناشی از جدایی یون‌های مثبت و منفی در دو طرف غشا \cite{ermen}.}
\label{fig:potential}
\end{figure} 
شارش یون‌ها در دو طرف غشا مسبب ایجاد اختلاف پتانسیل است. هنگامی که نورون هیچ فعالیتی نداشته باشد به اصطلاح در حال استراحت است. اختلاف پتانسیل ناشی از این حالت پتانسیل استراحت\LTRfootnote{resting potential} نامیده می‌شود. در این زمان پتانسیل داخل سلول نسبت به خارج آن منفی‌تر است. علت آن است که غلظت یون پتاسیم مثبت در داخل بیشتر از خارج و غلظت یون سدیم مثبت در خارج بیشتر از داخل است. بنابراین سدیم بر اساس شیب غلظت تمایل به ورود و پتاسیم تمایل به خروج دارد. اما از آنجایی که هیدراته\RTLfootnote{چون عدد اتمی سدیم $(11)$ کوچکتر از پتاسیم $(19)$ است تعداد مولکول‌های آبی که یون سدیم جذب می‌کند بیشتر از یون پتاسیم می‌شود. بنابراین هیدراته سدیم بزرگتر از پتاسیم است.} یون سدیم نسبت به پتاسیم بزرگ‌تر است، تعداد یون‌های پتاسیمی که از سلول خارج می‌شوند بیشتر از تعداد سدیمی است که وارد می‌شوند. به همین دلیل یون‌های منفی داخل سلول خود را بیشتر نشان می‌دهند.

% \begin{figure} [htbp]
%\centering
%\includegraphics[width=7cm , height=5cm]{hydrolize.png} 
%\caption{\footnotesize شمایی از هیدراته یون سدیم. این تصویر مفهوم هیدراته را برای یون سدیم نشان می‌دهد. \cite{bear}.}
%\label{fig:sude}
%\end{figure}

\section{ پتانسیل عمل }
پتانسیل عمل تغییر ناگهانی اختلاف پتانسیل در دو طرف غشای سلول است. این اختلاف پتانسیل از تفاوت غلظت یون‌های مثبت و منفی در دو طرف غشای سلول  حاصل می‌شود. پتانسیل عمل یک پالس الکتریکی به مدت $1$ تا $2$ میلی‌ثانیه و  با دامنه‌ای در حدود صد‌ میلی‌ولت می‌باشد \cite{ermen}. 

با توجه به شکل (\ref{fig:action}) مراحل ایجاد پتانسیل عمل به این صورت است که با اعمال اندک تحریک به سلول، ابتدا کانال‌های سدیم باز شده و یون‌ها وارد سلول می‌شوند. شارش یون‌های سدیم به داخل، محیط آن ‌را مثبت‌تر از بیرون می‌کند و سبب افزایش پتانسیل غشا می‌شود. سپس کانال‌های سدیمی بسته و کانال‌های پتاسیم باز می‌شود. شارش پتاسیم به خارج، محیط بیرون را منفی‌تر از داخل می‌کند و در نتیجه پتانسیل کاهش می‌یابد\RTLfootnote{افزایش پتانسیل ناشی از یون‌های سدیم واقطبیدگی و کاهش پتانسیل ناشی از یون‌های پتاسیم بازقطبیدگی (\LRTfootnote{Repolarization}) نامیده می‌شود.}. در مرحله آخر نیز با بسته شدن کانال پتاسیم پتانسیل غشا به حالت استراحت بر می‌گردد. با تکرار این فرایند پتانسیل عمل در یک غشای نورونی تولید می‌شود. هر پتانسیل عمل، بار‌های مثبت سدیم را وارد غشا می‌کند. یون‌های سدیم به ناحیه مجاور سلول که در حال استراحت است جابجا می‌شوند و آن ناحیه را واقطبیده\LTRfootnote{depolarization} می‌کنند. در واقع جابجایی سدیم مثبت به ناحیه مجاور موجب تغییر ولتاژ و باز شدن کانال‌های سدیمی شده و در نهایت پتانسیل عمل تولید می‌شود. بنابراین پتانسیل عمل به عنوان محرکی برای تولید پتانسیل عمل در ناحیه دیگری می‌شود.
\begin{figure} [htbp]
\centering
\includegraphics[width=10cm , height=7cm]{action.png} 
\caption[مراحل ایجاد پتانسیل عمل] {\footnotesize مراحل ایجاد پتانسیل عمل \cite{ermen}.}
\label{fig:action}
\end{figure}
\section{ سیناپس }
محل اتصال دو نورون به یکدیگر سیناپس\LTRfootnote{synapse} نامیده می‌شود که در آن تبادل اطلاعات از آکسون یک نورون به دندریت نورون دیگر صورت می‌گیرد. نورون  تولید کننده پیام عصبی را نورون پیش‌سیناپسی\LTRfootnote{presynaptic} و نورون  دریافت کننده پیام را نورون پس‌سیناپسی\LTRfootnote{postsynaptic }  می‌گویند.  نورون‌های پیش‌سیناپسی پیام‌ها را از طریق آکسون به پایانه آکسونی\LTRfootnote{axon terminal} می‌فرستند. این اطلاعات به فضای سیناپسی\LTRfootnote{synaptic Cleft} فرستاده و از طریق نورون‌های پس‌سیناپسی دریافت می‌شوند. ارتباط بین نورون‌های پیش‌ و پس‌سیناپسی از طریق مواد شیمیایی با نام انتقال دهنده‌های عصبی\LTRfootnote{neurotransmiter} صورت می‌گیرد. پیام‌های عصبی به دو طریق الکتریکی و شیمیایی منتقل می‌شودن. به همین منظور سیناپس‌ها را در دو دسته  الکتریکی\LTRfootnote{electrical synapse} و شیمیایی\LTRfootnote{chemical synapse} قرار می‌دهند \cite{kandel} .
 
در سیناپس الکتریکی دو نورون توسط کانال‌هایی به نام شکاف پیوندگاه\LTRfootnote{‫gap‬‬ ‫junction‬‬} از هم جدا شده‌اند.  این شکاف آنقدر باریک است که تنها اجازه عبور یون ها ی  بسیار کوچک را  میدهد. در سیناپس الکتریکی یون‌ها از یک نورون به طور مستقیم  و بدون واسطه از کانال شکاف پیوندگاه وارد نورون بعدی می‌شوند و با قرار گرفتن بر  روی غشای نورون باعث واقطبیدگی آن می‌شوند. اگر واقطبیدگی سلول از میزان آستانه فعالیت نورون تجاوز کند کانال‌های یونی وابسته به ولتاژِ\LTRfootnote{voltage-gated ion chanel} نورون پس‌سیناپسی باز می‌شوند و یک پتانسیل عمل تولید می‌شود.  این کانال‌ها دقیقا روبروی هم قرار دارند.  به همین خاطر یون‌ها و دیگر مولکول‌ها به راحتی از یک نورون به نورون دیگر وارد می‌شوند. 
یکی از  ویژگی سیناپس الکتریکی این است که به دلیل سرعت بالای انتقال پالس‌های الکتریکی به طور ناگهانی تعداد بسیار زیادی از نورون‌ها با همدیگر به طور هم‌زمان فعال می‌شوند و این یعنی هم‌زمانی فعالیت الکتریکی نورون‌ها.

در سیناپس شیمیایی، نورون‌های پیش و پس‌سیناپسی به طور کامل توسط  شکاف سیناپسی از هم جدا شده‌اند. 
گفتیم که تاخیر زمانی در سیناپس الکتریکی بسیار ناچیز است. اما این تاخیر  اندک در مورد سیناپس شیمیایی امکان‌پذیر نمی‌باشد. چرا که انتقال پیام از طریق سیناپس شیمیایی نیازمند عبور از چند مرحله است: 
\begin{itemize}
\item آزاد شدن پیام‌رسان‌های عصبی از نورون پیش‌سیناپسی
\item پخش مولکول‌های پیام‌رسان در فضای سیناپسی
\item چسبیدن پیام‌رسان‌ها به گیرنده‌های نورون پس‌سیناپسی
\item باز شدن کانال‌های یونی برای شارش یون‌ها و تولید پتانسیل عمل در نورون پس‌سیناپسی
\end{itemize}

با انجام مرحله به مرحله این چهار پروسه پیام عصبی به شکل شیمیایی منتقل می‌شود (شکل \ref{fig:synapse}).
 \begin{figure} [htbp]
\centering
\includegraphics[width=10cm , height=5cm]{synapse.png} 
\caption[عملکرد سیناپس الکتریکی و شیمیایی] {\footnotesize عملکرد سیناپس الکتریکی (سمت چپ) و سیناپس شیمیایی (سمت راست) \cite{kandel}.}
\label{fig:synapse}
\end{figure}

و اما سیناپس شیمیایی بسته به نوع نوروترنسمیتر‌ها به دو نوع تحریکی\LTRfootnote{excitatory} و مهاری\LTRfootnote{inhibitory} تقسیم‌بندی می‌شوند \cite{lodish}. نوروترنسمیتر‌ها با مکانیزم‌های قابل توجهی می‌توانند بر عملکرد یک نورون تاثیر بگذارند. از جمله تاثیر‌های مستقیم نوروترنسمیتر‌ها بر روی سلول عصبی تحریک‌پذیری الکتریکی نورون است. نوروترنسمیتر‌ها بر جریان یونی غشا نورون پیش‌سیناپسی تأثیر می‌گذارند و با تحریک یا مهار سلول، امکان تولید پتانسیل عمل را در نورونی که با آن در تماس است (نورون پس‌سیناپسی) به وجود می‌آورد \cite{lodish}.
 \begin{figure} [htbp]
\centering
\includegraphics[width=8cm , height=8.5cm]{excite.png} 
\caption[واکنش سیناپس تحریکی] {\footnotesize واکنش سیناپس تحریکی ($a$) و مهاری ($b$) \cite{lodish}.}
\label{fig:excite}
\end{figure}

\subsection{ جریان سیناپسی}
یون‌هایی که از غشا عبور می‌کنند جریان سیناپسی به وجود می‌آورند که ناشی از جریان کل یون‌های گذرنده از کانال‌های غشا است.
در یک رابطه‌ی ریاضی برای هر سیناپس که بین دو نورون وجود دارد جریان سیناپسی متناظر با آن را طبق رابطه زیر خواهیم داشت:
\begin{equation}
\ I_{\textsc{syn}} = g_{\textsc{pre}}S(t)(V_{\textsc{post}} - V_{\textsc{rev}})
\end{equation}
در این رابطه $g$  و $V_{\textsc{rev}}$ به ترتیب قدرت سیناپسی و پتانسیل بازگشتی را نشان می‌دهند. برای هر یون پتانسیل بازگشتی یا تعادلی، پتانسیل غشا است که در آن شارش جریان خالص از طریق کانال‌های باز صفر است. به عبارت دیگر، در پتانسیل بازگشتی نیرو‌های الکتریکی و شیمیایی با یکدیگر در تعادل هستند. این پتانسیل را می‌توان از طریق معادله نرنست\LTRfootnote{Nernst equation} محاسبه نمود. $S(t)$ نیز ضریب وابسته به زمان برای کانال‌‌‌هایی است که باز هستند. مقدار $S(t)$ همیشه غیر منفی و برای نورون‌هایی که شلیک\LTRfootnote{fire} نکرده‌اند برابر صفر است. برای سیناپس تحریکی، پتانسیل بازگشتی بزرگ‌تر از پتانسیل استراحت است، به طوری که یک جریان به سمت داخل را ایجاد می‌کند. در مقابل پتانسیل بازگشتی برای سیناپس مهاری نزدیک به پتانسیل بازگشتی  یون پتاسیم\RTLfootnote{پتانسیل بازگشتی برای یون پتاسیم $-88.7 \textsc{mV} $ می‌باشد.} می‌باشد \cite{ermen}.

با تعاریف اولیه و آشنا شدن با کارکرد دستگاه عصبی مرکزی، اکنون می‌توانیم توضیحی از شبکه نورونی را بیان داریم. شبکه نورونی از قرار گرفتن تعداد بسیار نورون در کنار هم و اتصال بین نورون‌ها با یکدیگر به وجود می‌آید که  مطالعه رفتار و فعالیت هر قسمت از شبکه منجر به درک درستی از تمام شبکه خواهد شد. شبکه نورونی با اتصال‌های بسیار زیاد خود تداعی‌گر شبکه‌های پیچیده‌ هستند.  در این شبکه، نورونی که به آستانه فعالیت خود برسد و آتش کند نورون‌های دیگر را نیز  با تحت تاثیر قراردادن  خود فعال می‌کند و این عمل به نورون‌های بعد که قابلیت دریافت پالس عصبی  را داشته باشند  سرایت می‌کند. به همین ترتیب فعالیت دسته جمعی در شبکه‌ نورونی به وجود می‌آید. بررسی  رفتار این گروه از نورون‌ها و شناخت و درک از نحوه فعالیت در شبکه نورونی مستلزم مطالعه مختصری از خصوصیات و مفاهیم پایه‌ای از شبکه پیچیده  است که در فصل آتی بیانگر آن‌ها خواهیم بود.
\newpage 
\textbf{خلاصه‌ی فصل اول}    \begin{itemize}
\item نورون واحد پردازنده اطلاعات در سیستم عصبی مرکزی و دارای سه قسمت اصلی دندریت، جسم سلولی و آکسون است.
\item برقراری ارتباط بین نورون‌ها از طریق سیناپس شکل می‌گیرد و بسته به نوع پیام‌رسانی‌شان به دو دسته الکتریکی و شیمیایی تقسیم می‌شوند. 
\item در سیناپس الکتریکی پیام‌رسانی بدون واسطه و از طریق کانال‌های شکاف پیوندگاه انجام می‌شود. در صورتی‌که در سیناپس شیمیایی با آزادشدن پیام‌رسان‌های شیمیایی بین شکاف سیناپسی این عمل صورت می‌گیرد.
\item سیناپس شیمیایی با توجه به نوع پیام‌رسان‌هایش به سیناپس تحریکی و مهاری دسته‌بندی می‌شود. 
\item در سیناپس تحریکی، پتانسیل عمل نورون‌ پیش‌سیناپسی امکان وقوع پتانسیل عمل را در نورون پس‌سیناپسی افزایش می‌دهد. اما در سیناپس مهاری، احتمال رخ‌داد پتانسیل عمل در نورون پس‌سیناپسی توسط پتانسیل عمل نورون پیش‌سیناپسی بسیار اندک است.
\item جریان سیناپسی برای هر نورون متناظر با کل جریان‌های گذرنده از کانال‌های نورون است. این جریان به عواملی مانند میزان رسانندگی سیناپس‌ها، اختلاف پتانسیل غشای نورون پس‌سیناپسی و پتانسیل بازگشتی و نیز کسری از کانال‌‌های باز وابسته است. 
\end{itemize}
 % if any un-wanted empty page is produced between chapters, you can use \input instead of %\include : \chapter{دستگاه عصبی مرکزی}
\section{مقدمه}
هماهنگی بین اعمال و اندام‌های بدن توسط دستگاه‌های ارتباطی که در بدن موجودات سلولی وجود دارد انجام می‌شود. سیستم عصبی با ساز و کار ویژه خود وظیفه این هماهنگی را بر عهده دارد. سلول‌های عصبی از مهم‌ترین و پیچیده‌ترین واحد پردازنده سیستم عصبی مرکزی هستند. اجزا و سازوکار این سلول از موضوعات اساسی مطالعه دستگاه عصبی به شمار می‌آید.
در این فصل دستگاه عصبی مرکزی را مورد بررسی قرار می‌دهیم.
\section{انواع سلول‌ها در بافت‌های عصبی}
دستگاه عصبی مرکزی به دو دسته تقسیم می‌شود:
\begin{itemize}\item سلول عصبی به نام نورون\LTRfootnote{neuron}  که  انتقال دهنده‌ پیام‌‌های عصبی به شمار می‌آید. این سلول‌ها به عنوان سلول‌های تحریک‌پذیر شناخته می‌شوند. نورون‌ها پیام‌های عصبی را به بافت‌ها، اندام‌ها و دیگر نورون‌ها می‌فرستند و از این طریق با آن‌ها ارتباط برقرار می‌کنند.
\item سلول‌های غیرعصبی به نام نوروگلیا\LTRfootnote{neuroglia}
 یا گلوسیت\LTRfootnote{gliocyte} که سلول‌های پشتیبان محسوب می‌شوند و وظیفه محافظت از نورون‌ها را بر عهده دارند. این سلول‌ها در انتقال پیام عصبی نقشی ندارند. سلول گلیا به صورت الکتریکی تحریک نمی‌شوند. نوروگلیا را سلول‌های تحریک‌ناپذیر نیز می‌گویند. 
\end{itemize}تعداد نورون‌ها در مغز انسان در حدود $100$
 میلیارد است و هر نورون به طور متوسط می‌تواند با $10$ هزار نورون دیگر ارتباط برقرار کند. در مقابل تعداد نوروگلیا چندین برابر نورون‌هاست ($5$ یا $10$ برابر تعداد نورون‌ها).
\section{ساختمان اصلی نورون}
هر نورون شامل سه بخش اصلی است: جسم سلولی\LTRfootnote{cell body}، دندریت\LTRfootnote{dendrite}، آکسون\LTRfootnote{axon} . شکل (\ref{fig:neuron}) این سه بخش را به طور واضح نشان می‌دهد.
\begin{figure} [htbp]
\centering
\includegraphics[width=9cm , height=5cm]{neuron.png} 
\caption[نورون و بخش‌های مختلف آن] {\footnotesize نورون و بخش‌های مختلف آن \cite{bear}.}
\label{fig:neuron}
\end{figure}
جسم سلولی یا سوما\LTRfootnote{soma} مرکز اصلی سلول عصبی می‌باشد. دندریت‌ها انشعابات درخت‌گونه هستند و در واقع قسمت اصلی دریافت اطلاعات و سیگنال‌هایی هستند که از دیگر نورون‌ها به نورون نوعی می‌رسد. آکسون نیز سیگنال‌های دریافتی را به سلول‌های دیگر می‌فرستد. طول آکسون در برخی موارد به دو متر نیز می‌رسد \cite{kandel}.
‌ %\subsection{تقسیم بندی نورون}
%نورون‌ها از نظر طرز خارج شدن تارهای عصبی از جسم سلولی به سه گروه تک قطبی\LTRfootnote{Unipolar}، دوقطبی\LTRfootnote{Bipolar} و چندقطبی\LTRfootnote{Moltipolar} تقسیم می‌شوند. این تقسیم بندی اولین بار توسط رامون کاخال\LTRfootnote{Ramon Cajal} انجام شده است \cite{kandel}.
%نورون‌‌های تک قطبی ساده‌ترین نوع نورون‌ها محسوب می‌شوند که از یک جسم سلولی و شاخه‌های متعدد با اندازه‌های یکسان تشکیل شده است. یکی از شاخه‌ها به عنوان آکسون و باقی آن‌ها دندریت نورون به شمار می‌آیند. چنین نورون‌هایی در ساختار موجودات بی‌مهره وجود دارند. در ساختار مهره‌داران در سیستم عصبی خودکار دیده می‌شوند. دستگاه عصبی خودکار، دسته‌ای از نورون‌های حرکتی هستند که فعالیت ماهیچه‌های صاف، تراوش غدد، تپش قلب و به طور کلی اندام‌های درونی را کنترل می‌‌کنند.

%نورون‌های دوقطبی دارای جسم سلولی بیضی‌گون هستند که دو شاخه از آن خارج می‌شود. یکی از آن‌ها مربوط به دندریت است که سیگنال‌ها را دریافت می‌کند و دیگری در نقش آکسون و حاوی اطلاعاتی است که آن‌ها را به طرف سیستم عصبی مرکزی می‌فرستد. بیشتر سلول‌های حساس از جمله سلول‌های چشم و بویایی در این دسته جای می‌گیرند. 

%نورون‌های چندقطبی در سیستم عصبی مهره‌داران دیده می‌شوند. آن‌ها نوعا دارای تک آکسون و دندریت‌های فراوان در اطراف جسم سلولی هستند. نورون‌های بدن انسان در این گروه قرار دارند.

%در بعضی قسمت‌های دستگاه عصبی، نورون‌هایی که فاقد آکسون هستند شناسایی شده‌اند که این نورون‌ها فقط قادرند تحریک عصبی را به نورون‌های مجاور خود منتقل کنند.

%نورون‌ها را بر مبنای عملکردشان می‌توان به سه دسته‌ی نورون‌های حسی\LTRfootnote{Sensory Neuron }، نورون‌های حرکتی\LTRfootnote{Motor Neuron} و نورون‌های رابط\LTRfootnote{Interneuron} تقسیم کرد.نورون‌های حسی پیام‌های عصبی را به طرف دستگاه عصبی هدایت می‌کنند. نورون‌های حرکتی پیام‌ها را از دستگاه عصبی مرکزی دریافت می‌کنند و آن‌ها را به سمت عضلات صاف و اسکلتی و قلبی می‌فرستند و در نهایت نورون های رابط یا نورون‌های واسطه، نورون‌هایی هستند که رابط بین نورون‌ها با یکدیگر در سیستم عصبی است. از جمله این نورون‌ها می‌توان به نورون‌های واسطه در قشر مغز اشاره کرد \cite{kandel}. 

%نمونه‌ نورون‌هایی را که در بالا ذکر شده است در شکل مشاهده می‌کنیم. \begin{figure} [htbp]	\centering	\includegraphics[width=9cm , height=6cm]{types.png} 	\caption{\footnotsize طبقه‌بندی نورون‌ها از نظر شکل و کارکرد \cite{kandel}.} \label{fig:types} \end{figure}

\section{فعالیت نورون}
سلول‌های زنده اختلاف پتانسیلی در دو طرف غشا\LTRfootnote{membrane} دارند.  قشر داخلی و خارجی  غشا توسط لایه چربی از هم جدا می‌شوند.  این لایه نسبت به عبور یون‌های موثری که در تولید پتانسیل عمل نقش دارند انتخابی\LTRfootnote{selective} است. در طول غشا کانال‌هایی وجود دارد که سبب شارش یون‌ها و در نتیجه باعث ایجاد اختلاف پتانسیل در دو سوی سلول می‌شوند. این پتانسیل، پتانسیل غشا نامیده می‌شود و از رابطه زیر بدست می‌آید \cite{ermen}:
\begin{equation}
V_{\textsc{M}} = V_{\textsc{in}} - V_{\textsc{out}}
\end{equation}
در این رابطه  $V_{\textsc{M}}$  اختلاف پتانسیل غشا، $V_{\textsc{in}}$ و 
$V_{\textsc{out}}$ به ترتیب پتانسیل داخل و خارج غشا هستند (شکل \ref{fig:potential}).
\begin{figure} [htbp]
\centering
\includegraphics[width=10cm , height=7cm]{potential.png} 
\caption[پتانسیل غشای سلول] {\footnotesize پتانسیل غشای سلول ناشی از جدایی یون‌های مثبت و منفی در دو طرف غشا \cite{ermen}.}
\label{fig:potential}
\end{figure} 
شارش یون‌ها در دو طرف غشا مسبب ایجاد اختلاف پتانسیل است. هنگامی که نورون هیچ فعالیتی نداشته باشد به اصطلاح در حال استراحت است. اختلاف پتانسیل ناشی از این حالت پتانسیل استراحت\LTRfootnote{resting potential} نامیده می‌شود. در این زمان پتانسیل داخل سلول نسبت به خارج آن منفی‌تر است. علت آن است که غلظت یون پتاسیم مثبت در داخل بیشتر از خارج و غلظت یون سدیم مثبت در خارج بیشتر از داخل است. بنابراین سدیم بر اساس شیب غلظت تمایل به ورود و پتاسیم تمایل به خروج دارد. اما از آنجایی که هیدراته\RTLfootnote{چون عدد اتمی سدیم $(11)$ کوچکتر از پتاسیم $(19)$ است تعداد مولکول‌های آبی که یون سدیم جذب می‌کند بیشتر از یون پتاسیم می‌شود. بنابراین هیدراته سدیم بزرگتر از پتاسیم است.} یون سدیم نسبت به پتاسیم بزرگ‌تر است، تعداد یون‌های پتاسیمی که از سلول خارج می‌شوند بیشتر از تعداد سدیمی است که وارد می‌شوند. به همین دلیل یون‌های منفی داخل سلول خود را بیشتر نشان می‌دهند.

% \begin{figure} [htbp]
%\centering
%\includegraphics[width=7cm , height=5cm]{hydrolize.png} 
%\caption{\footnotesize شمایی از هیدراته یون سدیم. این تصویر مفهوم هیدراته را برای یون سدیم نشان می‌دهد. \cite{bear}.}
%\label{fig:sude}
%\end{figure}

\section{ پتانسیل عمل }
پتانسیل عمل تغییر ناگهانی اختلاف پتانسیل در دو طرف غشای سلول است. این اختلاف پتانسیل از تفاوت غلظت یون‌های مثبت و منفی در دو طرف غشای سلول  حاصل می‌شود. پتانسیل عمل یک پالس الکتریکی به مدت $1$ تا $2$ میلی‌ثانیه و  با دامنه‌ای در حدود صد‌ میلی‌ولت می‌باشد \cite{ermen}. 

با توجه به شکل (\ref{fig:action}) مراحل ایجاد پتانسیل عمل به این صورت است که با اعمال اندک تحریک به سلول، ابتدا کانال‌های سدیم باز شده و یون‌ها وارد سلول می‌شوند. شارش یون‌های سدیم به داخل، محیط آن ‌را مثبت‌تر از بیرون می‌کند و سبب افزایش پتانسیل غشا می‌شود. سپس کانال‌های سدیمی بسته و کانال‌های پتاسیم باز می‌شود. شارش پتاسیم به خارج، محیط بیرون را منفی‌تر از داخل می‌کند و در نتیجه پتانسیل کاهش می‌یابد\RTLfootnote{افزایش پتانسیل ناشی از یون‌های سدیم واقطبیدگی و کاهش پتانسیل ناشی از یون‌های پتاسیم بازقطبیدگی (\LRTfootnote{Repolarization}) نامیده می‌شود.}. در مرحله آخر نیز با بسته شدن کانال پتاسیم پتانسیل غشا به حالت استراحت بر می‌گردد. با تکرار این فرایند پتانسیل عمل در یک غشای نورونی تولید می‌شود. هر پتانسیل عمل، بار‌های مثبت سدیم را وارد غشا می‌کند. یون‌های سدیم به ناحیه مجاور سلول که در حال استراحت است جابجا می‌شوند و آن ناحیه را واقطبیده\LTRfootnote{depolarization} می‌کنند. در واقع جابجایی سدیم مثبت به ناحیه مجاور موجب تغییر ولتاژ و باز شدن کانال‌های سدیمی شده و در نهایت پتانسیل عمل تولید می‌شود. بنابراین پتانسیل عمل به عنوان محرکی برای تولید پتانسیل عمل در ناحیه دیگری می‌شود.
\begin{figure} [htbp]
\centering
\includegraphics[width=10cm , height=7cm]{action.png} 
\caption[مراحل ایجاد پتانسیل عمل] {\footnotesize مراحل ایجاد پتانسیل عمل \cite{ermen}.}
\label{fig:action}
\end{figure}
\section{ سیناپس }
محل اتصال دو نورون به یکدیگر سیناپس\LTRfootnote{synapse} نامیده می‌شود که در آن تبادل اطلاعات از آکسون یک نورون به دندریت نورون دیگر صورت می‌گیرد. نورون  تولید کننده پیام عصبی را نورون پیش‌سیناپسی\LTRfootnote{presynaptic} و نورون  دریافت کننده پیام را نورون پس‌سیناپسی\LTRfootnote{postsynaptic }  می‌گویند.  نورون‌های پیش‌سیناپسی پیام‌ها را از طریق آکسون به پایانه آکسونی\LTRfootnote{axon terminal} می‌فرستند. این اطلاعات به فضای سیناپسی\LTRfootnote{synaptic Cleft} فرستاده و از طریق نورون‌های پس‌سیناپسی دریافت می‌شوند. ارتباط بین نورون‌های پیش‌ و پس‌سیناپسی از طریق مواد شیمیایی با نام انتقال دهنده‌های عصبی\LTRfootnote{neurotransmiter} صورت می‌گیرد. پیام‌های عصبی به دو طریق الکتریکی و شیمیایی منتقل می‌شودن. به همین منظور سیناپس‌ها را در دو دسته  الکتریکی\LTRfootnote{electrical synapse} و شیمیایی\LTRfootnote{chemical synapse} قرار می‌دهند \cite{kandel} .
 
در سیناپس الکتریکی دو نورون توسط کانال‌هایی به نام شکاف پیوندگاه\LTRfootnote{‫gap‬‬ ‫junction‬‬} از هم جدا شده‌اند.  این شکاف آنقدر باریک است که تنها اجازه عبور یون ها ی  بسیار کوچک را  میدهد. در سیناپس الکتریکی یون‌ها از یک نورون به طور مستقیم  و بدون واسطه از کانال شکاف پیوندگاه وارد نورون بعدی می‌شوند و با قرار گرفتن بر  روی غشای نورون باعث واقطبیدگی آن می‌شوند. اگر واقطبیدگی سلول از میزان آستانه فعالیت نورون تجاوز کند کانال‌های یونی وابسته به ولتاژِ\LTRfootnote{voltage-gated ion chanel} نورون پس‌سیناپسی باز می‌شوند و یک پتانسیل عمل تولید می‌شود.  این کانال‌ها دقیقا روبروی هم قرار دارند.  به همین خاطر یون‌ها و دیگر مولکول‌ها به راحتی از یک نورون به نورون دیگر وارد می‌شوند. 
یکی از  ویژگی سیناپس الکتریکی این است که به دلیل سرعت بالای انتقال پالس‌های الکتریکی به طور ناگهانی تعداد بسیار زیادی از نورون‌ها با همدیگر به طور هم‌زمان فعال می‌شوند و این یعنی هم‌زمانی فعالیت الکتریکی نورون‌ها.

در سیناپس شیمیایی، نورون‌های پیش و پس‌سیناپسی به طور کامل توسط  شکاف سیناپسی از هم جدا شده‌اند. 
گفتیم که تاخیر زمانی در سیناپس الکتریکی بسیار ناچیز است. اما این تاخیر  اندک در مورد سیناپس شیمیایی امکان‌پذیر نمی‌باشد. چرا که انتقال پیام از طریق سیناپس شیمیایی نیازمند عبور از چند مرحله است: 
\begin{itemize}
\item آزاد شدن پیام‌رسان‌های عصبی از نورون پیش‌سیناپسی
\item پخش مولکول‌های پیام‌رسان در فضای سیناپسی
\item چسبیدن پیام‌رسان‌ها به گیرنده‌های نورون پس‌سیناپسی
\item باز شدن کانال‌های یونی برای شارش یون‌ها و تولید پتانسیل عمل در نورون پس‌سیناپسی
\end{itemize}

با انجام مرحله به مرحله این چهار پروسه پیام عصبی به شکل شیمیایی منتقل می‌شود (شکل \ref{fig:synapse}).
 \begin{figure} [htbp]
\centering
\includegraphics[width=10cm , height=5cm]{synapse.png} 
\caption[عملکرد سیناپس الکتریکی و شیمیایی] {\footnotesize عملکرد سیناپس الکتریکی (سمت چپ) و سیناپس شیمیایی (سمت راست) \cite{kandel}.}
\label{fig:synapse}
\end{figure}

و اما سیناپس شیمیایی بسته به نوع نوروترنسمیتر‌ها به دو نوع تحریکی\LTRfootnote{excitatory} و مهاری\LTRfootnote{inhibitory} تقسیم‌بندی می‌شوند \cite{lodish}. نوروترنسمیتر‌ها با مکانیزم‌های قابل توجهی می‌توانند بر عملکرد یک نورون تاثیر بگذارند. از جمله تاثیر‌های مستقیم نوروترنسمیتر‌ها بر روی سلول عصبی تحریک‌پذیری الکتریکی نورون است. نوروترنسمیتر‌ها بر جریان یونی غشا نورون پیش‌سیناپسی تأثیر می‌گذارند و با تحریک یا مهار سلول، امکان تولید پتانسیل عمل را در نورونی که با آن در تماس است (نورون پس‌سیناپسی) به وجود می‌آورد \cite{lodish}.
 \begin{figure} [htbp]
\centering
\includegraphics[width=8cm , height=8.5cm]{excite.png} 
\caption[واکنش سیناپس تحریکی] {\footnotesize واکنش سیناپس تحریکی ($a$) و مهاری ($b$) \cite{lodish}.}
\label{fig:excite}
\end{figure}

\subsection{ جریان سیناپسی}
یون‌هایی که از غشا عبور می‌کنند جریان سیناپسی به وجود می‌آورند که ناشی از جریان کل یون‌های گذرنده از کانال‌های غشا است.
در یک رابطه‌ی ریاضی برای هر سیناپس که بین دو نورون وجود دارد جریان سیناپسی متناظر با آن را طبق رابطه زیر خواهیم داشت:
\begin{equation}
\ I_{\textsc{syn}} = g_{\textsc{pre}}S(t)(V_{\textsc{post}} - V_{\textsc{rev}})
\end{equation}
در این رابطه $g$  و $V_{\textsc{rev}}$ به ترتیب قدرت سیناپسی و پتانسیل بازگشتی را نشان می‌دهند. برای هر یون پتانسیل بازگشتی یا تعادلی، پتانسیل غشا است که در آن شارش جریان خالص از طریق کانال‌های باز صفر است. به عبارت دیگر، در پتانسیل بازگشتی نیرو‌های الکتریکی و شیمیایی با یکدیگر در تعادل هستند. این پتانسیل را می‌توان از طریق معادله نرنست\LTRfootnote{Nernst equation} محاسبه نمود. $S(t)$ نیز ضریب وابسته به زمان برای کانال‌‌‌هایی است که باز هستند. مقدار $S(t)$ همیشه غیر منفی و برای نورون‌هایی که شلیک\LTRfootnote{fire} نکرده‌اند برابر صفر است. برای سیناپس تحریکی، پتانسیل بازگشتی بزرگ‌تر از پتانسیل استراحت است، به طوری که یک جریان به سمت داخل را ایجاد می‌کند. در مقابل پتانسیل بازگشتی برای سیناپس مهاری نزدیک به پتانسیل بازگشتی  یون پتاسیم\RTLfootnote{پتانسیل بازگشتی برای یون پتاسیم $-88.7 \textsc{mV} $ می‌باشد.} می‌باشد \cite{ermen}.

با تعاریف اولیه و آشنا شدن با کارکرد دستگاه عصبی مرکزی، اکنون می‌توانیم توضیحی از شبکه نورونی را بیان داریم. شبکه نورونی از قرار گرفتن تعداد بسیار نورون در کنار هم و اتصال بین نورون‌ها با یکدیگر به وجود می‌آید که  مطالعه رفتار و فعالیت هر قسمت از شبکه منجر به درک درستی از تمام شبکه خواهد شد. شبکه نورونی با اتصال‌های بسیار زیاد خود تداعی‌گر شبکه‌های پیچیده‌ هستند.  در این شبکه، نورونی که به آستانه فعالیت خود برسد و آتش کند نورون‌های دیگر را نیز  با تحت تاثیر قراردادن  خود فعال می‌کند و این عمل به نورون‌های بعد که قابلیت دریافت پالس عصبی  را داشته باشند  سرایت می‌کند. به همین ترتیب فعالیت دسته جمعی در شبکه‌ نورونی به وجود می‌آید. بررسی  رفتار این گروه از نورون‌ها و شناخت و درک از نحوه فعالیت در شبکه نورونی مستلزم مطالعه مختصری از خصوصیات و مفاهیم پایه‌ای از شبکه پیچیده  است که در فصل آتی بیانگر آن‌ها خواهیم بود.
\newpage 
\textbf{خلاصه‌ی فصل اول}    \begin{itemize}
\item نورون واحد پردازنده اطلاعات در سیستم عصبی مرکزی و دارای سه قسمت اصلی دندریت، جسم سلولی و آکسون است.
\item برقراری ارتباط بین نورون‌ها از طریق سیناپس شکل می‌گیرد و بسته به نوع پیام‌رسانی‌شان به دو دسته الکتریکی و شیمیایی تقسیم می‌شوند. 
\item در سیناپس الکتریکی پیام‌رسانی بدون واسطه و از طریق کانال‌های شکاف پیوندگاه انجام می‌شود. در صورتی‌که در سیناپس شیمیایی با آزادشدن پیام‌رسان‌های شیمیایی بین شکاف سیناپسی این عمل صورت می‌گیرد.
\item سیناپس شیمیایی با توجه به نوع پیام‌رسان‌هایش به سیناپس تحریکی و مهاری دسته‌بندی می‌شود. 
\item در سیناپس تحریکی، پتانسیل عمل نورون‌ پیش‌سیناپسی امکان وقوع پتانسیل عمل را در نورون پس‌سیناپسی افزایش می‌دهد. اما در سیناپس مهاری، احتمال رخ‌داد پتانسیل عمل در نورون پس‌سیناپسی توسط پتانسیل عمل نورون پیش‌سیناپسی بسیار اندک است.
\item جریان سیناپسی برای هر نورون متناظر با کل جریان‌های گذرنده از کانال‌های نورون است. این جریان به عواملی مانند میزان رسانندگی سیناپس‌ها، اختلاف پتانسیل غشای نورون پس‌سیناپسی و پتانسیل بازگشتی و نیز کسری از کانال‌‌های باز وابسته است. 
\end{itemize}

\chapter{شبکه‌های پیچیده }
\section{مقدمه}
  شبکه‌های بزرگ مقیاس به عنوان یک سامانه پیچیده\LTRfootnote{complex network} رفتارهای پیچیده‌ای از خود نشان می‌دهند \cite{boccara}. شبکه‌های کامپیوتر، جامعه انسان‌ها  و شبکه‌های نورونی نمونه‌هایی از شبکه پیچیده هستند که از تعداد زیادی عناصر متصل به هم ساخته شده‌اند \cite{newman}. شبکه پیچیده، شبکه‌ای است  که رفتار کلی آن با بررسی رفتار تک تک عناصر ممکن نمی‌باشد بلکه باید رفتار دسته‌جمعی شبکه مطالعه شود. برای بررسی این شبکه‌ها از نظریه گراف استفاده می‌شود \cite{costa}. در این نظریه‌ی ریاضی، شبکه مانند گرافی در نظر گرفته می‌شود  که راس‌های\LTRfootnote{vertex} آن عناصر شبکه و اتصال‌ بین راس‌ها، یال‌های\LTRfootnote{edge} آن را تشکیل می‌دهد \cite{boc,albert}. مطالعه شبکه سابقه‌ی طولانی در ریاضیات گسسته، جامعه‌شناسی و نظریه گراف داشته است و اخیرا فیزیک و زیست‌شناسی را نیز تحت تاثیر خود قرار داده است. 

مطالعه شبکه و تئوری گراف به اوایل سده ۱۸ برمی‌گردد، زمانی‌ که اویلر\LTRfootnote{Leonhard\,Paul\,Eule} در پی جوابی برای حل مساله‌ی هفت پل کونیگسبرگ\RTLfootnote{نام امروزیش کالینینگراد است و در شهر لیتوانی در کشور روسیه قرار دارد.} مشغول بود؛ ساکنین شهر به دنبال مسیری بودند که از تمام پل‌ها تنها یک بار عبور کنند. اویلر با استفاده از گراف زیر ثابت می‌کند که این کار غیر ممکن است \cite{network,west}:
\begin{figure}[htbp]
\centering
\includegraphics[width=9cm , height=4cm]{pol.png} 
\caption[گراف مربوط به پل گونیگسبرگ] {\footnotesize گراف مربوط به پل گونیگسبرگ (راست) و طرحی از شهر کونیگسبرگ (چپ) \cite{network}.}
\label{fig:pol}
\end{figure}\\

تا قبل از دهه  $1950$ مطالعات بر روی گراف بیشتر در زمینه گراف‌های منظم\LTRfootnote{regular} بوده است. اما بعد از آن با مطالعه شبکه‌های واقعی، دیدگاه بشر به سمت گراف‌های نامنظم سوق پیدا کرده است. اردوش و رنی\LTRfootnote{Erdös–Rényi}  اولین بار در دهه $1950$ مدل گراف تصادفی\LTRfootnote{random} را مطالعه کردند. در گراف تصادفی، هر دو راس  به طور تصادفی  با احتمالی یکسان با یک یال به یکدیگر متصل می‌شوند.  مدت‌ها بعد از تئوری اردوش و رنی، واتس و استروگتز\LTRfootnote{Watts and Strogatz}مدل خود را در قالب شبکه‌ی جهان کوچک\LTRfootnote{small World} \cite{watts} و   باراباسی و آلبرت\LTRfootnote{Barabási and Albert}  در قالب شبکه‌ی بی‌مقیاس \LTRfootnote{scale free} \cite{bara} رفتار شبکه‌ واقعی را بررسی کردند.

همانطور که گفتیم، شبکه‌ها از دیرباز شاخه‌ای از علم گراف محسوب می‌شدند و همچنین به این دلیل که شبکه‌ها قابل نمایش با گراف هستند، در ابتدا نماد‌ها و تعاریف اولیه در گراف را مورد بررسی قرار می‌دهیم و سپس مدل‌هایی از شبکه‌های پیچیده را معرفی می‌کنیم \cite{new}.
\section{ماتریس مجاورت}
در تعریف گراف، برای نمایش عناصر آن چندین روش وجود دارد\RTLfootnote{از روش‌های دیگر می‌توان به لیست مجاورت و لیست مجاورت معکوس اشاره کرد.}. یکی از این روش‌ها نمایش ماتریس مجاورت\LTRfootnote{adjacency matrix} است. این ماتریس راس‌هایی را که با هم در ارتباط هستند مشخص می‌کند. برای مثال شبکه‌ای را در نظر می‌گیریم که راس‌های آن از $1$ تا $\textsc{n}$ شماره‌گذاری شده است و تعدادی از راس‌ها نیز با‌ هم در ارتباط می‌باشند. ماتریس مجاورت $\textsc{A}$ برای گراف ساده غیرجهتی\LTRfootnote{undirected grapf} با عناصر $\textsc{A}_{\textsc{ij}}$ چنین تعریف می‌شود که اگر بین دو راس یالی وجود داشته باشد $\textsc{A}_{\textsc{ij}} = 1$ و در غیر این‌صورت 
$\textsc{A}_{\textsc{ij}} = 0$ خواهد بود. ماتریس مجاورت برای یک گراف غیرجهتی یک ماتریس متقارن خواهد بود. یعنی اگر از راس $\textsc{i}$ به راس $\textsc{j}$ یالی وجود داشته باشد حتما از $\textsc{j}$ به$\textsc{i}$ نیز وجود خواهد داشت. %همچنین به دلیل نبودِ  حلقه در شبکه‌های غیر جهتی درایه‌های قطر اصلی مقادیر صفر دارند. 
\subsection{شبکه جهتی}
شبکه یا گراف جهتی\LTRfootnote{directed graph} که به اختصار گراف ‌جهت‌دار (digraph) نامیده می‌شود گرافی‌ است که هر یالش با یک مسیر راس‌ها را با یک فلش به یکدیگر متصل می‌کند. مثال‌هایی از این نوع سامانه‌ها شبکه جهانی وب\LTRfootnote{World Wide Web} می‌باشد که در آن لینک‌ها در یک جهت از یک صفحه به صفحه دیگر به اجرا در می‌آیند و یا نمونه دیگر شبکه‌های نورونی هستند که انتقال اطلاعات از یک نورون به نورون دیگر توسط سیناپس‌ها صورت می‌گیرد. ماتریس مجاورت برای گراف جهتی این‌گونه تعریف می‌شود که اگر از $i$ به $j$ مسیری وجود داشته باشد 
$\textsc{A}_{\textsc{ij}} = 1$  و در غیر این‌‌صورت $\textsc{A}_{\textsc{ij}} = 0$ . باید توجه کنیم که تفاوت اساسی میان گراف جهتی و غیر جهتی  آن است که ماتریس مجاورت شبکه غیر جهتی متقارن نیست.  یعنی اگر بین دو راس $\textsc{i}$ و $\textsc{j}$ یالی وجود داشته باشد لزوما معکوس آن برقرار نمی‌باشد. % همچنین درایه‌های روی قطر اصلی در این شبکه‌ها برابر با یک است. 
\begin{figure}[htbp]
\hspace*{0cm}
\centering
%\begin{minipage}[b]{0.4\textwidth}
\includegraphics[width=0.3\linewidth, height=45mm]{directed.png}\centering(الف)
\includegraphics[width=0.3\linewidth, height=45mm]{undirected.png}\centering(ب)
\caption[تقسیم‌بندی گراف جهتی و غیر جهتی] {\footnotesize دو گراف: (الف) گراف غیرجهتی  و (ب) گراف جهتی برای چهار راس \cite{network}.}
\end{figure}
\subsection{شبکه وزن‌دار}
بیشتر شبکه‌هایی که مورد مطالعه قرار می‌گیرند بین رئوسشان دو حالت وجود دارد: یا اتصالی  بین آنها برقرار است  و یا برقرار نیست. اما در برخی موارد  بسته به نوع کاربرد‌هایی که شبکه می‌تواند داشته باشد وزنی را به شبکه نسبت می‌دهیم. شبکه وزن‌دار\LTRfootnote{weighted} یعنی اینکه درجه راس‌ها از توزیع خاصی پیروی می‌کنند. بنابراین اگر بخواهیم به شبکه‌های کامپیوتری وزنی را نسبت دهیم تعداد داده‌هایی که در طول شبکه وجود دارد و پهنای باند شبکه وزن آن محسوب می‌شود. وزن‌ها در شبکه‌های وزنی اصولأ مثبت در نظر گرفته می‌شوند. اما در برخی مواقع می‌توان وزن منفی نیز به شبکه نسبت داد. به عنوان مثال در یک شبکه اجتماعی افرادی که با هم رابطه دوستی دارند مثبت و آنهایی که با هم دشمنی دارند منفی در نظر می‌گیرند. مثالی دیگر از شبکه وزنی، شبکه نورونی است که در آن نورون‌ها نمایشگر گره‌ها و سیناپس‌ها یال‌های شبکه را شکل می‌دهند و وزن شبکه معرف قدرت سیناپسی  شبکه است. ماتریس مجاورت برای  چنین شبکه‌هایی کسری است. 
 
 \subsection{شبکه درختی}
  از نمونه‌های گراف غیرجهتی می‌توان به شبکه‌های درختی\LTRfootnote{tree } اشاره کرد. شبکه درختی، شبکه غیرجهتی متصل به هم هستند که هیچ حلقه‌ای در آن وجود ندارد\normalfootnotes\footnote{در واقع می‌توانیم به یال‌ها در شبکه درختی جهت نیز نسبت دهیم که در این صورت یک گراف جهتی خواهیم داشت. اما همان‌طور که تعریف کرده‌ایم این شبکه، یک شبکه بدون حلقه درنظر گرفته می‌شود و بنابراین از جهتی بودن آن صرف نظر می‌شود.}. اتصال در این شبکه به این معناست که هر راس در شبکه درختی از طریق برخی مسیر‌ها در شبکه به راس‌های دیگر دسترسی پیدا می‌کند. چنین شبکه‌هایی می‌توانند دارای دو یا چند قسمت مجزا باشند. اگر یک بخش منحصر به فرد در شبکه دارای حلقه نباشد ساختار آن درختی خواهد بود. از  جمله ویژگی‌های شبکه درختی این است که برای   $\textsc{n}$ راس در ساختار دقیقا   $\textsc{n - 1}$ یال وجود دارد. عکس این قضیه نیز درست است. یعنی اگر شبکه‌‌ای داشته باشیم که   $\textsc{n}$ راس و $\textsc{n - 1}$ یال داشته باشد شبکه درختی خواهد بود.

\subsection{طول کوتاه‌ترین مسیر و خوشگی}
مسیر\LTRfootnote{path} در گراف زنجیری است که دو راس را به هم وصل می‌کند. طول کوتاه‌ترین مسیر\LTRfootnote{shortest path length}  در گراف، کوتاه‌ترین فاصله میان دو گره است. در گراف فاصله میان دو گره را با متوسط طول کوتا‌ترین مسیر\LTRfootnote{average shortest path length}  پیدا می‌کنند. این کمیت با میانگین گیری بر روی تمام کوتاه‌ترین مسیر بین هر دو جفت گره به دست می‌آید.

 
ضریب خوشگی\LTRfootnote{clustering coefficient} یکی از ویژگی‌های آماری شبکه است. ویژگی خوشه شدن در شبکه به این معناست که راس‌های شبکه تمایل به تشکیل خوشه‌های جدا از هم دارند. مشابه این وضعیت در شبکه اجتماعی همان تشکیل گروه‌های متفاوت در جامعه است که بر اساس آن مشخص می‌شود چقدر احتمال دارد که دوستان یک فرد در شبکه با یکدیگر دوست باشند. ضریب خوشگی یک راس برابر است با تعداد مثلث‌های گراف تقسیم بر تعداد کل مثلث‌هایی که می‌تواند در گراف وجود داشته باشد. منظور از مثلث یعنی اینکه اگر بین دو راس $i$ و $\textsc{j}$ و نیز راس‌های $\textsc{j}$ و $\textsc{k}$ مسیری وجود داشته باشد بین دو راس $\textsc{i}$ و $\textsc{k}$ نیز یالی برقرار باشد.
\subsection{تابع توزیع درجه}
یکی از ویژگی‌های  متمایز کننده شبکه‌ها از یکدیگر توزیع درجه‌\LTRfootnote{degree distribution} راس‌ها می‌باشد که آن را با $P(k)$ نمایش می‌دهند. $P(k)$ نشان دهنده کسری از راس‌های گراف است که درجه آنها $k$ است. تابع توزیع درجه برای شبکه‌های مختلف متفاوت است. برای مثال این توزیع برای شبکه بی‌مقیاس به شکل توانی\LTRfootnote{power low}، برای شبکه‌های تصادفی به شکل پواسونی\LTRfootnote{poisson} و یا گاوسی\LTRfootnote{gaussian}.
لازم به  یادآوری است که در تعریف درجه یک راس تنها تعداد یال‌های خارج شده و یا وارد شده به آن راس شمرده می‌شود. اما اینکه آن یال از کدام راس به آن رسیده است در نظر گرفته نمی‌شود. 
\section{مدل‌های شبکه}  
دستیابی به اطلاعات یک سیستم پیچیده واقعی کاری بس مشکل و چه بسا ناممکن است. چرا که  اطلاعات دقیقی از سیستم واقعی در اختیار نداریم و نیز کامپیوتر‌ها قدرت محاسبه چنین اطلاعات سنگینی را ندارند. یکی از این مشکلات در شبکه‌های عصبی دیده می‌شود که برای بررسی سیستم اطلاعات دقیقی از نحوه کارکرد و نیز تعداد سیناپس‌ها و نورون‌ها در اختیار نداریم. برای رفع این مشکل از مدل‌های شبکه بهره می‌گیرند که به شبکه‌های دنیای واقعی نزدیک است.  ویژگی‌های مشترک در شبکه‌های دنیای واقعی طول مشخصه مسیر و ضریب خوشگی  در شبکه‌هاست. اکثر شبکه‌هایی که در دنیای واقعی دیده می‌شوند دارای طول مسیر کوچک و ضریب خوشه شدن بالا هستند. بر اساس این ویژگی‌های ساختاری، می‌توان شبکه‌ها را دسته‌بندی‌ کنیم. از جمله این شبکه‌ها می‌توان   شبکه منظم\LTRfootnote{regular}، جهان کوچک، شبکه تصادفی و بی‌مقیاس را نام برد که در ادامه به بررسی هر یک از آنها می‌پردازیم.
\subsection{شبکه منظم}
در نظریه گراف، یک گراف منظم شبکه‌ای است که هر راس آن همسایه‌های یکسان داشته باشد. به عبارت دیگر همه راس‌ها دارای توزیع درجه یکسان باشند. گراف منظمی که درجه همه راس‌های آن $k$ باشد، گراف منظم-$k$\LTRfootnote{k-regular graph} نامیده می‌شود. 
مثال‌هایی  گراف منظم رادر زیر آمده است:
\begin{itemize}
\item گراف منظم-$0$ (گرافی که درجه همه راس‌ها صفر است.)
\item گراف منظم-$1$ (گرافی که درجه راس‌ها  $1$ است و هر دو راس تنها با یک یال به هم اتصال دارند. )
\item  گراف منظم-$2$(گرافی که درجه همه راس‌ها $2$ باشد. در این نوع هر $3$ راس یک چرخه جدا از هم را تشکیل می‌دهند.)
\item گراف منظم-$3$(به شکل یک گراف مکعبی در نظر گرفته می‌شود.)
\end{itemize}
\begin{figure} [htbp]
\centering
\includegraphics[width=12cm , height=4.cm]{regular.png}
\caption[نمونه‌هایی از گراف منظم] {\footnotesize نمونه‌هایی از گراف منظم: از چپ به راست(گراف منظم-0، 1، 2 و 3)}\cite{network}
\label{fig:regular}
\end{figure}\\\\
توجه کنیم که در شبکه منظم برای گراف‌هایی که 
$k  \geq 2$ باشد خوشگی داریم.
\subsection{شبکه جهان کوچک}
در اواخر سده $1960$ میلگرام\LTRfootnote{Milgram} طی آزمایشی معروف افرادی را به طور تصادفی انتخاب کرد و برای آنها نامه‌هایی را فرستاد. وی از آن افراد خواست تا نامه‌ها را به یک فرد که از پیش تعیین شده است بفرستند و در ضمن شرط فرستادن نامه نیز این بود که باید آن را به یک دوست صمیمی بفرستند. در نتیجه افراد به دنبال دوستانی می‌گشتند که به فرد مورد نظر نزدیک باشد. این روند تا جایی تکرار می‌شد که نامه به شخص مورد نظر برسد. زمانی که حدود $20$ درصد از نامه‌های میلگرام به مقصد رسید او متوجه شد  مسیری که هر نامه طی کرده است به طور متوسط دارای طول $6$ بود؛ یا به بیان دیگر فقط $6$ نفر بین هر دو نفر قرار داشتند. نتیجه آزمایش این بود هردو شخصی که بر روی زمین زندگی می‌کنند نیز به تعداد این تکرار‌ها با یکدیگر رابطه برقرار می‌کنند \cite{milgram,traver}. براساس آزمایش میلگرام این پدیده جهان کوچک نام گرفت. 
این نظریه  می گوید هر دو انسان ساکن بر روی کره زمین، به طور میانگین در یک رابطه با $6$ واسطه یا کمتر به هم مربوط می‌شوند.
ش %بکه جهان کوچک این نتیجه را می‌دهد که فاصله میانگین بین دو گره در دو شبکه مختلف می‌تواند نزدیک‌تر از فاصله گره‌هایی باشد که همان شبکه را می‌سازند. 

شبکه جهان کوچک دارای دو ویژگی است:
\begin{enumerate}
\item طول مسیر کوتاه
\item ضریب خوشگی بالا
\end{enumerate}
شبکه جهان کوچک از مدل واتس-استروگتز پیروی می‌کند \cite{albert,watts}. مدل پیشنهادی واتس و استروگتز برای جهان کوچک مدلی است که ساختارش بین ساختار گراف منظم و گراف تصادفی است. این مدل بر پایه‌ بازآرایی\LTRfootnote{rewiring} یال‌ها با احتمال $p$ انجام می‌شود. آنها در ابتدا یک گراف منظم که کامل نیست با $N$ راس در نظر می‌گرفتند. سپس هر یال را با احتمال $p$ بازآرایی می‌کردند. یعنی با احتمال $p$ یک یال را از یک راس جدا کرده و به راس دیگر وصل کردند. این عمل برای تک تک راس‌ها تکرار شد \cite{watts,boc,albert}. در این مدل با انتخاب $p = 0$ گراف به یک شبکه منظم و با احتمال $p = 1$ به گراف تصادفی می‌رسیم و برای مقادیر میانی کوچک $p$ شبکه به یک گراف جهان کوچک تبدیل می‌شود.
\begin{figure} [htbp]
\centering
\includegraphics[width=9.5cm , height=3.cm]{small.png}
\caption[شبکه جهان کوچک با مدل پیشنهادی واتس-استروگتز] {\footnotesize شبکه جهان کوچک با مدل پیشنهادی واتس-استروگتز. در این مدل بازآرایی    یال‌ها را مشاهده می‌کنیم \cite{watts}.}
\label{fig:small}
\end{figure}
\subsection{شبکه تصادفی}
اولین مدلی که برای توصیف شبکه‌های واقعی ارائه شد مدل گراف تصادفی  بود که در سال $1959$ توسط اردوش و رنی پایه‌گذاری شد \cite{erdos}. در این مدل تعدادی گره در نظر می‌گیرند و سپس هر یک از گره‌ها را با احتمال ثابتی به گره‌های دیگر وصل می‌کنند. به زبان دیگر، برای ساختن چنین گرافی یک پارامتر احتمال $p$ تعریف می‌کنند. دو راس با احتمال $p$ به هم با یک یال متصل می‌شوند. 
توزیع درجه‌ی $P_{k}$ برای گراف تصادفی  به صورت زیر از توزیع دو جمله‌ای پیروی می‌کند،
\begin{equation}
P_{k} = \binom{N-1}{k}p^{k}(1-p)^{N-1-k}.
\end{equation}
 ‌این توزیع درجه در حد $N$های بزرگ به تابع توزیع پواسونی میل می‌کند،
\begin{equation}
P_{k} = e^{-c}~~\frac{c^{k}}{k!}. 
\end{equation}
که $k$ تعداد راس ها با درجه $k$ و  $c$ میانگین درجه راس های شبکه است \cite{new}.

مدل گراف تصادفی ویژگی گراف جهان کوچک را دارد و اندازه کوتاه‌ترین مسیر بین دو گره با لگاریتم تعداد گره‌های گراف متناسب است. در مقابل این گراف ضریب خوشگی کوچکی دارد.
\subsection{شبکه بی‌مقیاس}
شبکه بی‌مقیاس به شبکه‌ای گفته می‌شود که توزیع درجات راس‌ها از قانون توانی  به صورت زیر پیروی کند \cite{bar}،
\begin{equation}
P_{k} = k^{-\gamma}. 
\end{equation}
این توزیع نشان‌دهنده احتمال اتصال گره جدید به گره $i$ در شبکه مورد نظر است و متناسب با درجه هر راس تعریف می‌شود. 
همانطور که دیدیم مدل‌های قبلی که برای شبکه‌ها در نظر گرفتیم دارای توزیع درجه پواسونی بودند. به همین دلیل در سال $1998$ آلبرت و باراباسی مدل جدیدی را با در نظر گرفتن توزیع توانی شبکه‌های واقعی ارائه دادند\cite{bara,barab}.  در مدل آن‌ها نمای $\gamma$   برابر با $3$  در نظر گرفته شد. مدل باراباسی و آلبرت از دو مرحله رشد\LTRfootnote{growth} و مرحله پیوند ترجیحی\LTRfootnote{preferential attachment} تشکیل شده است. 

طبق این روش، در مرحله رشد شبکه‌ای کاملا به هم پیوسته با $m_{0}$  گره را در نظر گرفتند. سپس در هر مرحله یک گره جدید با $m$ یال$(m < m_{0})$ متصل به گره قدیمی اضافه کردند.
در مرحله پیوند ترجیحی هر یال جدید به یک گره قدیمی با احتمالی متناسب با درجه آن متصل می‌شود. احتمالی که راس جدید به راس قدیم متصل شود به شکل زیر تعریف می‌شود،

\begin{equation}
P(i) = \dfrac{k_{i}}{\sum_{i}{k_{i}}}.
\end{equation}
که در آن  $k_{i}$  درجه راس $i$ام است. با توجه با این رابطه هرچه احتمال برای یک راس بیشتر باشد، احتمال اتصال راس جدید به آن راس بیشتر می‌شود. % از مشخصه‌های مدل آلبرت و باراباسی این است که با افزایش درجه راس‌ها، ضریب خوشگی کاهش می‌یابد. 
مدل بی‌مقیاس نمونه‌ای از "ضرب‌المثل غنی-غنی‌تر میگردد‌\LTRfootnote{rich-get-richer}" است.
\begin{figure} [htbp]
\centering
\includegraphics[width=9.5cm , height=3.cm]{scale.png}
\caption[نمایش مدل پیوند رشد برای یک شبکه بی‌مقیاس] {\footnotesize نمایش مدل رشد برای یک شبکه بی‌مقیاس. در مرحله $t = 0$ سیستم شامل $m_{0} = 3$ راس مجزا است. در هر گام بعدی یک راس جدید(دایره تیره) به سیستم اضافه می‌شود و با $m = 2$ راس قبلی ارتباط برقرار می‌کند  \cite{watts}.}
\label{fig:scale}
\end{figure}\\\\
بعد از آنکه آلبرت و باراباسی مدل خود را ارائه دادند، مدل‌های دیگری برای شبکه‌های واقعی مطرح شد که حالت تعمیم یافته مدل باراباسی بوده است.  با بررسی داده‌های واقعی از بسیاری از شبکه‌های پیچیده یافت شد که نمای $\gamma$ برای این شبکه‌‌ها بین $2$ و $3$ متغیر است. از نمونه‌ شبکه‌های بی‌مقیاس که در شبکه‌های واقعی دیده می‌شود می‌توان  به شبکه جهانی وب، شبکه‌های بیولوژیکی و اجتماعی اشاره کرد.

\subsection{شبکه نورونی}
تا کنون انواع مدل‌های شبکه‌ را مورد بررسی قرار داده‌ایم. در اینجا مثالی از شبکه‌ پیچیده را معرفی می‌کنیم و ویژگی‌های آن را با شبکه‌هایی که تاکنون مطالعه کرده‌ایم بررسی می‌کنیم. مجموعه‌ای از نورون‌ها و اتصال‌های سیناپسی بین ‌آنها شبکه نورونی را به وجود می‌آورند.

شبکه‌های نورونی به دو شکل تعریف و بررسی می‌شوند:
 
نوع اول شبکه ساختاری\LTRfootnote{structural network} است که یال‌های بین گره‌ها (سیناپس‌های بین نورون‌ها) را بر اساس ارتباط فیزیکی در نظر می‌گیرند. این ارتباط می‌تواند در مقیاس کوچک بین تک تک نورون‌ها و یا در مقیاس بزرگ بین مجموعه‌ای از نورون‌ها در قسمت‌های مختلف سیستم باشد. 

نوع دوم شبکه کارکردی\LTRfootnote{functional network} مغز است که برای اتصال بین نورون‌ها باید شبکه نورونی بین دو ناحیه فعالیت هم‌زمان داشته باشد. به عبارت دیگر، مستقل از مکان نورون‌ها در صورتی که بین دو ناحیه که در آن نواحی تعداد نورون‌ها زیاد است  رفتار مشابهی دیده شود می‌توان بین نورون‌های آن نواحی اتصالی در نظر گرفت.

شبکه ساختاری  نورونی دارای ارتباط‌های وسیع و پیچیده‌ای است. با در نظر گرفتن مدل‌های شبکه‌ای، می‌توان به درک درستی از شبکه‌های پیچیده از جمله شبکه مغز پیدا کرد. ساختار شبکه‌های نورونی در مقیاس‌های کوچک و بزرگ متفاوت است و نشان داده شده است که این شبکه‌ها خواص شبکه بی‌مقیاس و جهان کوچک را دارا می‌باشند \cite{soode. از طرف دیگر در این شبکه، طول مسیر کوتاه به چشم می‌خورد. به عبارت دیگر هر نورون به واسطه ارتباط‌های سیناپسی، با نورون‌های همسایه‌اش  با احتمال بیشتری نسبت به نورون‌های دورتر ارتباط برقرار می‌کند\cite{gilani}.
 
نورون‌ها می‌توانند به‌ صورت الکتریکی برانگیخته شوند. همانطور که در فصل اول شرح دادیم، سلول در حال استراحت هیج‌ گونه فعالیتی ندارد. نورون‌‌ها آستانه‌ی مشخصی  برای فعال شدن دارند. زمانی که پتانسیل غشا از این حد آستانه بیشتر شود نورون آتش می‌کند. این عمل به نورون‌‌های بعدی نیز منتقل می‌شود و بنابراین باعث تحریک نورون‌های مجاور  و در نتیجه فعال شدن آنها می‌شود. در فصل آینده خواهیم گفت که فعال شدن دسته جمعی در نورون‌ها چه رویکرد‌هایی را به همراه دارد.

 شبکه نورونی مثالی از یک شبکه واقعی بیولوژیکی  است که ساختار آن  و نیز آستانه‌ نورون‌ها از توزیع درجه تصادفی تبعیت می‌کند.  اتصال بین نورون‌ها یکسان نبوده و هر نورون ورودی و خروجی متفاوتی را دریافت می‌کنند.  اما  در مجموع اتصال  بین نورون‌ها حول مقداری متوسط است. از این رو توزیع درجه‌ مناسب برای این شبکه را گاوسی در نظر گرفتند. از طرفی به دلیل همسان نبودن نورون‌ها برای دریافت آستانه فعالیت برای آتش کردن، توزیع آستانه‌ای  که برای شبکه نورونی در نظر  گرفتند گاوسی است. در فصل بعد فعالیت شبکه نورونی را به طور مفصل مورد بررسی قرار می‌دهیم. 
\newpage 
\textbf{خلاصه‌ی فصل دوم}    
\begin{itemize}
\item  ویژگی اصلی شبکه‌های پیچیده وجود عناصر بیشمار در ساختار آن و نیز برهم‌کنش میان این اجزا می‌باشد. 
\item  برای درک بهتر شبکه‌های پیچیده باید رفتار دسته‌جمعی اجزای آن را مورد بررسی قرار داد. 
\item مطالعه گراف اولین بار توسط اویلر بر روی گراف‌های منظم انجام شده است و در پی آن برای بررسی شبکه‌های واقعی، مدل‌های تصادفی شبکه‌ها نیز مورد توجه قرار‌گرفته است.
\item  از مشخصه‌های شبکه‌های واقعی طول مشخصه و ضریب خوشگی است. طول مشخصه فاصله میانگین میان دو راس و ضریب خوشگی تعداد دور‌های به طول $3$  را در گراف مشخص می‌کند.
\item گراف منظم دارای طول مسیر کوتاه و ضریب خوشگی بالا از مرتبه $N^\frac{1}{d} $ است که $d$ بعد شبکه را نشان می‌دهد.
\item طول مشخصه برای شبکه‌های جهان کوچک و تصادفی از $O(\log N)$ است. در عوض ضریب خوشگی در شبکه جهان کوچک بزرگ‌تر از شبکه تصادفی است.
\item ضریب خوشگی شبکه بی‌مقیاس بزرگ‌تر از شبکه تصادفی و کوچک‌تر از شبکه بی‌مقیاس است و طول شخصه این شبکه نیز از $O(\log(\log N))$ می‌باشد. 
\item شبکه‌های نورونی به علت همسان نبودن در دریافت ورودی و نیز آستانه‌ فعالیت شکل ساختاری و نیز آستانه فعالیتشان به شکل گاوسی در نظر گرفته می‌شود.
\end{itemize}
 





 
 
 




\chapter{تراوش}
\section{مقدمه}
 مساله تراوش\LTRfootnote{percolation} (نفوذپذیری) به مطالعه رفتار  خوشه‌ها  در شبکه‌های با ابعاد بی‌نهایت می‌پردازد. این خوشه‌ها زمانی در شبکه ظاهر می‌شوند که مکان‌های اشغال شده به یکدیگر متصل شده و جزیره‌هایی از مکان‌های اشغال شده را به وجود آورند. در این فصل مفاهیم مربوط به تراوش را روشن‌تر بیان می‌کنیم و سپس مدل استاندارد آن را بر روی شبکه دوبعدی و گراف تصادفی بررسی خوهیم کرد. در انتها نیز تراوش خودراه‌انداز را به شکل ساده برای درک بیشتر آن مختصری چند خواهیم گفت.
\section{تعاریف}
تراوش یکی از مفاهیم مهم در فیزیک به شمار می‌آید و در بسیاری از پدیده‌های طبیعی می‌توان آن را مشاهده کرد. 
در طول پنج دهه گذشته نظریه‌ی تراوش مطالعات گسترده‌ای را برای درک بهتر در زمینه‌های مختلف فیزیک، علم مواد، شبکه‌های پیچیده و زمینه‌های دیگر برای دنیای بشریت  فراهم آورده است.  در فیزیک آماری و ریاضی این مدل رفتار خوشه‌های به هم ‌پیوسته  از مکان‌ّای فعال را در یک گراف توصیف می‌کند. در علم زمین شناسی نیز این مساله به شارش آب از میان خاک و سنگ‌‌های نفوذپذیر اشاره می‌کند. در ادامه بحث، ابتدا خواص و انواع تراوش را برشمرده و سپس این مدل استاندارد آن را برای ساختار مربعی و گراف  مورد بررسی قرار می‌دهیم.
\subsection{ویژگی‌های تراوش}
در بررسی مساله تراوش برخی از کمیت‌ها، ویژگی‌های شبکه را برای ما روشن می‌سازد. از جمله این کمیت‌ها اندازه یا تعداد مکان‌های اشغال شده و نیز تعداد خوشه‌ها در شبکه با ابعاد بزرگ است.   
چگالی خوشه‌ها با اندازه $\textsc{s}$، ($\textsc{n}_{\textsc{s}}(\textsc{f})$) و احتمالی که یک مکان پر متعلق به خوشه‌ی بی‌‌نهایت باشد ($\textsc{P}_{\infty}(\textsc{f})$)، به شکل زیر تعریف می‌شوند، 
\begin{align}
\textsc{n}_{\textsc{s}}(\textsc{f}) &= \dfrac{\text{تعداد خوشه‌ها با ابعاد $\textsc{s}$ }}{\text{تعداد کل مکان‌ها در شبکه }},\\ \textsc{P}_{\infty}(\textsc{f})  &= \dfrac{\text{تعداد مکان‌های اشغال شده در خوشه بی‌نهایت}}{\text{تعداد کل مکان‌های اشغال شده در شبکه }}.
\end{align}
$\textsc{f}$ احتمال اولیه اشغال شدن مکان‌های شبکه است. در یک شبکه دو بعدی $\textsc{P}_{\infty}$ در نقطه گذار $\textsc{f}_{\textsc{c}}$  دو مقدار صفر و یک دارد. می‌توان گفت:
\begin{equation}
\[
\begin{cases}
  \text{if } & \textsc{f} < \textsc{f}_{\textsc{c}}~~~~~~~~\ \textsc{P}_{\infty} = 0 \\
   \text{if} & \textsc{f} > \textsc{f}_{\textsc{c}}~~~~~~~~\textsc{P}_{\infty} = 1             
\end{cases}
\]
\end{equation}
\begin{figure}[htbp]
\hspace*{0cm}
\centering
%\begin{minipage}[b]{0.4\textwidth}
\includegraphics[width=0.4\linewidth, height=50mm]{per.png}\centering(الف)    
\includegraphics[width=0.4\linewidth, height=50mm]{per1.png}\centering(ب)
\caption[مثالی از تراوش بر روی شبکه مربعی] {\footnotesize مثالی از تراوش بر روی شبکه مربعی با اندازه $16\times 16$:  (الف) برای $\textsc{f} = 0.2$ 
خوشه‌هایی با چگالی 
$\textsc{n}_{\textsc{s}}(1) = 20$ 
و 
$\textsc{n}_{\textsc{s}}(2) = 4$
 و 
 $\textsc{n}_{\textsc{s}}(3) = 5$
  و
 $\textsc{n}_{\textsc{s}}(7) = 1$
     دیده می‌شود که به ترتیب تعداد خوشه‌ها با اندازه $1$  و $2$ و $3$ و $7$  را نشان می‌دهد. (ب) 
 $\textsc{f} = 0.59$
     . برای این پیکربندی 
  $\textsc{P}_{\infty}(\textsc{f} = 0.56) = 140/154$ \cite{jan}.}
\label{fig:jan}
\end{figure}\\
 شکل (\ref{fig:jan}) مثالی از روشن شدن مفاهیم 
 $\textsc{P}_{\infty}$
  و  
 $\textsc{n}_{\textsc{s}}(\textsc{p}})$
  است. قسمت (الف) برای مقدار
 $\textsc{f} = 0.2$
    تعداد خوشه‌ها با اندازه $1$، $2$، $3$ و  $7$ را نشان می‌دهد. چگالی این خوشه‌ها به ترتیب 
 $\textsc{n}_{\textsc{s}}(1) = 20$ 
 و 
 $\textsc{n}_{\textsc{s}}(2) = 4$
  و 
 $\textsc{n}_{\textsc{s}}(3) = 5$ 
  و
 $\textsc{n}_{\textsc{s}}(7) = 1$
    است. قسمت ‌(ب) برای 
 $\textsc{f} = 0.59$ 
    مقدار 
 $\textsc{P}_{\infty}(\textsc{f} = 0.56) = 140/154$
      را نشان می‌دهد. 
 



%\begin{align}
%P_{\infty}& = \dfrac{\text{تعداد مکان‌های اشغال شده در خوشه بی‌نهایت}}{\text{تعداد کل مکان‌های اشغال شده در شبکه }}.\\
%&  ~~~~~n_{s}(p) = \dfrac{\text{تعداد خوشه‌ها با ابعاد $s$}}{\text{تعداد کل مکان‌هادر شبکه }}.
%\end{align}


\section{انواع تراوش}
در تعریف تراوش به دو نوع تراوش جایگاهی\LTRfootnote{site-percolation} و پیوندی\LTRfootnote{bond-percolation} می‌توان اشاره کرد\cite{percolate}. 
\subsubsection{تراوش جایگاهی}
فرض می‌کنیم در جعبه‌ای به طور تصادفی گلوله‌های فلزی و شیشه‌ای ریخته‌ایم. می‌خواهیم بدانیم که آیا جعبه به شکل یک رسانا رفتار می‌کند یا عایق. به زبان دیگر، به دنبال مسیری هستیم که از به هم پیوستن گلوله‌های فلزی، بالا و پایین و یا چپ و راست شبکه را به هم متصل کند. تصوّر می‌کنیم یک شبکه دو بعدی داریم که همه مکان‌هایش تهی است. در یک حالت تصادفی، گلوله‌های فلزی  با احتمال 
$\textsc{f}$
 پر می‌شوند و یا با احتمال
  $1 - \textsc{f}$ 
  گلوله‌های شیشه‌ای قرار می‌گیرند. یال ها نیز در این شبکه اتصال گلوله‌هایی است که با یکدیگر در تماس هستند. با این ساختار به دنبال نقطه گذاری هستیم تا رفتار شبکه را بررسی کنیم. می‌بینیم که در در شکل (\ref{fig:jan})  تصویر اول (راست) مسیر پرکولیت وجود ندارد، اما  در تصویر دوم (چپ) مسیر به وضوح پیدا است. 
\begin{figure}[htbp]
\hspace*{0cm}
\centering
%\begin{minipage}[b]{0.4\textwidth}
\includegraphics[width=0.3\linewidth, height=45mm]{path1.png}\centering(الف)    
\includegraphics[width=0.3\linewidth, height=45mm]{path.png}\centering(ب)
\caption[شمایی از تراوش جایگاهی] {\footnotesize شمایی از تراوش جایگاهی با ابعاد $35\times 35$: (الف) در احتمال 
$\textsc{f} = 0.25$ 
تراوش رخ نمی‌دهد و (ب) در احتمال 
$\textsc{f} = 0.65$ 
تراوش و مسیر خوشه مشخص است \cite{bela}.}
\label{fig:site}
\end{figure}
به این ترتیب نتیجه می‌گیریم مادامی که در گراف با راس‌ها سروکار داشته باشیم تراوش از نوع جایگاهی داریم. در این نوع تراوش مکان ها مستقل از هم با احتمال $\textsc{f}$ اشغال می‌شوند. در تراوش جایگاهی به دنبال خوشه به هم پیوسته از مکان‌های اشغال شده هستیم \cite{bela}.
\subsubsection{تراوش پیوندی}
برای درک بهتر از مدل تراوش  پیوندی  می‌توان مثال‌های زیادی را مورد بررسی قرار داد. یکی از مثال‌ها مربوط به شارش آب از میان یک محیط متخلخل است \cite{sahini}.

مساله از این قرار است که فرض می‌کنیم یک محیط متخلخل مانند اسفنج داریم و در این محیط از بالا آب می‌ریزیم. در اینجا سوالی  که برای ما مطرح می‌شود  این است که آیا مسیری وجود خواهد داشت تا مایع خود را از بالا به پایین برساند؟ رفتار این مساله از نوع تراوش پیوندی است. 

محیط متخلخل است. اما این تخلخل چقدر باشد تا آب را از خود نفوذ دهد؟ نکته اصلی در این است که حتما باید مسیری وجود داشته باشد که آب از بالا به پایین برسد.  فرض کنیم هر زمان  که آب به یکی از این مسیر‌ها وارد شود، می‌تواند به یکی از همسایه‌هایش نفوذ کند، به شرط آنکه آن مسیر‌ِ همسایه نیز دارای حفره‌ای باشد که جا برای آب وجود داشته باشد. به همین منوال هر مسیرِ 
همسایه نیز بتواند آب را به حفره‌ی مجاور خود برساند. 

اگر فرض کنیم احتمال اینکه هرکدام از این مسیرها که آب را عبور می‌دهد $\textsc{f}$ باشد (یعنی با احتمال $\textsc{f}$ یکی از مسیرها را روشن می‌کنیم)، و با احتمال
 $1 - \textsc{f} = \textsc{q}$ 
 آب نفوذ نکند؛ با این شرایط می‌توانیم مجموعه‌ای را به صورت آماری و تصادفی بسازیم. به این صورت که  برای هر مسیر شبکه یک عدد تصادفی بین صفر و یک انتخاب می‌کنیم. اگر آن عدد تصادفی از $\textsc{f}$ بزرگ‌تر بود پیوند  بین دو مکان را روشن می‌کنیم. اگر عدد $\textsc{f}$ خیلی کوچک باشد تعداد خیلی کمی از پیوندها روشن خواهد شد و احتمال اینکه تراوش اتفاق بیفتد خیلی کم است.  اما اگر 
 $\textsc{f} = 1$
  باشد، یعنی همه مکان‌ها روشن هستند و حتما تراوش اتفاق افتاده است. بنابراین ترواش پیوندی به مطالعه و پیدا کردن خوشه‌های به هم ویوسته   می‌پردازد که در آن خوشه به جای مکان‌ها، یال‌ها به هم متصل هستند \cite{book}.
 شکل (\ref{fig:bond})  این موقعیت را برایمان مشخص می‌کند.
\begin{figure}[htbp]
\hspace*{0cm}
\centering
%\begin{minipage}[b]{0.4\textwidth}
\includegraphics[width=0.3\linewidth, height=45mm]{pathbond.png}\centering(الف)    
\includegraphics[width=0.3\linewidth, height=45mm]{pathbond1.png}\centering(ب)
\caption[شمایی از تراوش پیوندی] {\footnotesize شمایی از تراوش پیوندی با ابعاد $40\times 40$: (الف) تراوش رخ می‌دهد  (ب) تراوش رخ داده است \cite{bela}.}
\label{fig:bond}
\end{figure}\\
\ 

 

 
%\subsection{بررسی تراوش استاندارد در یک بعد}

%در فضای یک بعدی مساله خیلی بدیهی است و به شکل تحلیلی قابل حل است. در این فرایند $f_{c}$ برابر یک است. برای اثبات این ادعا، شبکه یک بعدی با مجموعه نامحدودی از مکان‌ها را مطابق با شکل (\ref{fig:1D})  در نظر می‌گیریم که در فاصله برابر در طول یک خط چیده شده‌اند. همه مکان‌ها دو حالت دارند: یا با احتمال $\textsc{f}$ اشغال می‌شوند و یا با احتمال $1 - f$ تهی باقی می‌مانند.

%\begin{figure} [htbp]
%\centering
%\includegraphics[width=8cm , height=1cm]{1D.png} 
%\caption{\footnotesize تراوش در یک بعد. سایت‌ها با احتمال $\textsc{f}$ اشغال می‌شوند. علامت ضرب‌در نقاط تهی و دایره‌های مشکی نقاط اشغال شده‌اند. در این شکل خوشه‌هایی با اندازه پنج ، دو و یک را مشاهده می‌کنیم\cite{percolation}.}
%\label{fig:1D}
%\end{figure}

%چیزی که برای ما اهمیت دارد پیدا کردن نقطه گذار برای تشکیل خوشه در هر بعدی برای این نظریه است. در مساله یک بعدی، خوشه پرکولیت شده از $-\infty$ تا $\infty$ را شامل می‌شود و بدیهی است این زمانی امکان‌پذیر است که همه مکان‌ها اشغال شده باشند. زمانی که همه مکان‌ها اشغال شوند به این معنی است که نقطه گذار $f_{c}$ برابر یک است و اگر تنها یک مکان تهی در شبکه وجود داشته باشد مانع از تشکیل شدن خوشه تراوا می‌شود. اما در دو و سه بعد پیدا کردن مقدار $f_{c}$ به نوع شبکه بستگی دارد. برای بعضی از شبکه‌ها برای محاسبه حد بحرانی حل دقیق داریم. اما برای برخی دیگر حل دقیق پاسخ‌گوی نیاز ما نیست و باید از شبیه‌سازی برای تعیین مقدار بحرانی بهره بگیریم.
\section{مدل تراوش استاندارد در دو بعد}


{برای بررسی مساله تراوش استاندارد در شبکه‌های دوبعدی، یک شبکه \textsc{N } $\times$ \textsc{N} از مکان‌ها را در نظر می‌گیریم.
 $\textsc{N}$ 
 تعداد کل مکان‌ها در شبکه است. همان‌طور که در مبحث‌های پیشین گفتیم، این مکان‌ها با احتمال $\textsc{f}$ اشغال می‌شوند. با در نظر گرفتن عدد تصادفی برای هر مکان و  مقایسه احتمال $\textsc{f}$ با عدد تصادفی ، مکان‌ها را اشغال می‌کنیم. مکان‌هایی که اشغال شده‌اند  نیز تا پایان در همین حالت باقی می‌مانند. مکان‌های شبکه با احتمال داده شده پُر می‌شوند و خوشه‌هایی در این شبکه شکل می‌گیرند. هدف  ما تعیین تعداد نقاط اشغال شده و به دست آوردن بزرگ‌ترین خوشه به هم پیوسته از مکان‌های اشغال شده از میان چندین خوشه موجود در شبکه است. خوشه‌ای که شکل می‌گیرد، کسری از نقاط اشغال شده در شبکه است. در تراوش استاندارد  در شبکه‌ مربعی، رفتار خوشه نسبت به پارامتر $\textsc{f}$  پیوسته است و  نقطه گذار با اندازه شبکه جابجا می‌شود.  همچنین زمانی که اندازه شبکه را زیاد می‌کنیم به نقطه گذار واقعی نزدیک‌ می‌شویم.
 
 برای به دست آوردن بزرگ‌ترین خوشه به هم پیوسته  در شبکه مربعی از الگوریتمی که هوشن\LTRfootnote{Hoshen} و کوپلمن\LTRfootnote{Kopelman} در سال $1976$ پیشنهاد دادند  استفاده  کرده‌ایم. در این الگوریتم آرایه‌ها سطر به سطر از چپ به راست و از بالا به پایین پویش می‌شوند و مکان‌هایی که پُر و متصل به مکان‌های قبلی هستند یافت می‌شوند. با تکرار این روند و پیمایش کامل مکان‌ها، بزرگ‌ترین خوشه تعیین می‌شود \cite{baba}.  }

%مزایای این الگوریتم نسبت به باقی روش‌ها، ساده و سریع‌ بودن الگوریتم، استفاده از حافظه کم برای اجرای الگوریتم و ... است.
 الگوریتم هوشن-کوپلمن به این صورت است که همه مکان‌های پر از بالا به پایین و از چپ به راست پوییده می‌شوند و هر مکان پر شده با یک برچسب به عنوان اندازه خوشه معین می‌شود. الگوریتم اینگونه پیش می‌رود که اگر مکان اشغال شده‌ای همسایه پر نداشته باشد با برچسب جدید مشخص می‌شود. به عبارت دیگر مکان‌هایی که همسایه پر ندارند به عنوان یک خوشه جدید و مجزا شناخته می‌شوند. اگر مکانی یک همسایه پر داشته باشد با برچسب همسایه‌اش هم علامت می‌شود و به اندازه خوشه همسایه اضافه می‌شود. اما اگر مکان اشغال شده دو یا بیشتر از دو همسایه پر داشته باشد،  به همسایه‌ای متصل می‌شود که  برچسب آن کوچک‌تر است. در پیمایش بعدی همه مکان‌هایی که اشغال و به هم متصل هستند هم‌برچسب می‌شوند و به این ترتیب بزرگ‌ترین خوشه به هم پیوسته از مکان‌های اشغال را پیدا می‌کنیم \cite{hosh}. . 
 
 شکل (\ref{fig:s1}) اندازه بزرگ‌ترین خوشه به هم پیوسته از مکان‌های اشغال شده را برحسب تعداد مکان‌های اشغال شده اولیه برای شبکه مربعی با  اندازه‌های متفاوت نشان می‌دهد. در این شکل
  $\textsc{S}_{\textsc{gc}}$ 
  اندازه  بزرگ‌ترین خوشه به هم‌ پیوسته از مکان‌های اشغال شده و 
  $\textsc{f}$
   احتمال اولیه اشغال شدن مکان‌ها است. میانگین‌گیری روی $100$ نمونه انجام شده است.
 
\begin{figure} [htbp]
\centering
\includegraphics[width=11cm , height=8cm]{max3.eps} 
\caption[اندازه بزرگترین خوشه به هم پیوسته در یک شبکه دو بعدی] {\vspace{-0.01}\footnotesize اندازه بزرگترین خوشه به هم پیوسته  بر حسب تعداد مکان‌های اوبیه اشغال شده در یک شبکه دو بعدی مربوط به تراوش استاندارد.}
\label{fig:s1}
\end{figure}
 نمودار (\ref{fig:s1}) این گفته را برای ما یادآوری می‌کند که تغییر در اندازه شبکه، نقطه گذار را تغییر داده و با افزایش اندازه به نقطه گذار واقعی  نزدیک می‌شویم. علاوه بر این، در این نمودار‌ها گذار پیوسته را برای تراوش استاندارد مشاهده می‌کنیم. 
 
\section{تراوش استاندارد بر روی گراف تصادفی}
   برای بررسی تراوش استاندارد بر روی شبکه‌های تصادفی از مدل گراف تصادفی اردوش و رنی (\textsc{ER}) استفاده می‌کنیم. در این گراف با 
   $\textsc{N}$ 
   راس، در ابتدا با یک احتمال اولیه 
   $\textsc{p}$ 
   تعدادی از راس‌ها را به طور تصادفی به هم وصل می‌کنیم. میانگین درجه راس‌ها در حد 
   $\textsc{N}$های بزرگ با رابطه زیر به احتمال
    $\textsc{p}$
     مربوط است \cite{newman}، 
   \begin{equation}
   {\left\langle \textsc{k} \right\rangle = \textsc{p}\times \textsc{N}}.
   \end{equation}
   همچنین تعدادی از مکان‌ها را با احتمال اولیه $\textsc{f}$ فعال می‌کنیم. سپس بزرگترین خوشه از مکان‌های فعال را در شبکه به دست می‌آوریم. 
    قبل از آنکه به بررسی نتایج بپردازیم به این نکته نیز اشاره می‌کنیم که برای به دست آوردن اندازه بزرگ‌ترین خوشه به هم پیوسته از مکان‌های فعال از الگوریتم جستجوی سطح اول\LTRfootnote{Breadth First Search} (‌BFS) بهره گرفته‌ایم. این الگوریتم برای پیدا کردن خوشه‌ها در ساختار‌های درختی و گراف استفاده می‌شود. برای پیاده‌سازی این الگوریتم از صف استفاده می‌شود. یک صف برای نگه‌داشتن راس‌های همسایه استفاده می‌شود. 
   
   روش کار مبنتی بر این است که ابتدا یکی از راس‌ها را به طور تصادفی برای شروع الگوریتم انتخاب می‌کنیم و آن را در صف به عنوان ریشه قرار می‌دهیم. سپس همه راس‌هایی که با آن همسایه هستند و نیز تا به حال ملاقات نشده‌اند را پیدا می‌کنیم و آن را در  صف قرار می‌دهیم. در این زمان راسی که به عنوان ریشه انتخاب شده بود را از صف حذف می‌کنیم. در مرحله بعد از اولین راسی که در صف قرار دارد شروع به بازدید راس‌های همسایه و ملاقات نشده می‌کنیم و تمام راس‌هایی که  در این مرحله بازدید و  مشخص شده‌اند را در انتهای صف قرار می‌دهیم. به همین روال از دومین  راس در صف شروع کرده و راس قبلی را از صف حذف و راس‌های جدید را به آن اضافه می‌کنیم. به این ترتیب همه راس‌های همسایه سطح به سطح در گراف پویش می‌شوند. در خاتمه، همه  راس‌هایی که در صف قرار می‌گرفتند را شمارش کرده و تعدادشان را به عنوان اندازه خوشه ذخیره می‌کنیم. فرایند تا جایی پیش می‌رود که همه راس‌ها در گراف پیموده شوند. از میان خوشه‌هایی که به دست می‌آید، خوشه‌ای که اندازه‌اش بزرگ‌تر از باقی خوشه‌ها باشد به عنوان بزرگ‌ترین خوشه فعال به هم پیوسته گزارش شده است  \cite{zu,newman}.
   
   
   در گراف
    $\textsc{ER}$
     نتایجی که از تراوش استاندارد بدست آمده است در شکل (\ref{fig:SP}) نشان داده می‌شود. در این نمودار با تغییر احتمال 
$\textsc{p}$، مکان گذار جابه‌جا می‌شود. از محاسبات تحلیلی نشان داده شده است در گراف 
 $\textsc{ER}$ 
 مکان گذار با معکوس میانگین درجه مشخص می‌شود. به عبارت دیگر گذار دقیقا بر روی معکوس میانگین درجه‌اش اتفاق می‌افتد \cite{newman}. نتایج شبیه‌سازی دو منحنی که در شکل  (\ref{fig:SP}) به دست آمده است تایید کننده نتایج تحلیلی است. به روشنی مشخص است برای شبکه‌ای که میانگین درجه‌اش
  $\left\langle \textsc{k} \right\rangle  = 5$
   است (منحنی $1$)، نقطه گذار در $0.2$ و برای شبکه‌ای که میانگین درجه‌اش
    $\left\langle \textsc{k}\right\rangle  = 2$
     است (منحنی $2$)، نقطه گذار بر روی $0.5$ قرار دارد. منحنی‌های به دست آمده برای شبکه‌ای با تعداد راس‌های 
     $\textsc{N} = 10000$
      رسم شده است. 
      $\textsc{E}$
       در اینجا تعداد میانگین‌گیری است که  بر روی  $100$  نمونه انجام گرفته است. 
   \begin{figure} [htbp]
   \centering
   \includegraphics[width=11cm , height=7cm]{standard.eps} 
   \caption[نمودار مربوط به اندازه بزرگ‌ترین خوشه در گراف  
   $\textsc{ER}$
   ] {\vspace{-0.01}\footnotesize نمودار مربوط به اندازه بزرگ‌ترین خوشه در گراف 
    $\textsc{ER}$ 
    با 
    $\textsc{N}=10000$
     راس مربوط به تراوش استاندارد. منحنی‌ $1$ برای گراف با میانگین درجه
      $\left\langle \textsc{k} \right\rangle = 5$ 
      و منحنی $2$ برای گراف با میانگین درجه 
      $\left\langle \textsc{k} \right\rangle = 2$ 
      رسم شده است. همانطور که می‌بینیم نقطه گذار برای هر نمودار با معکوس میانگین‌ درجه‌اش متناسب است.}
   \label{fig:SP}
   \end{figure}
 
 
 
 \newpage
 \section{معرفی تراوش خود‌راه‌انداز}
 تراوش خودراه‌انداز به عنوان مدلی برای انتشار  فعال‌سازی مکان‌های یک شبکه مطرح شده است \cite{cohen}. از جمله کاربرد‌های این مدل می‌توان به شبکه‌های نورونی \cite{ec}، شبکه‌های مغناطیسی \cite{baxter}، اجتماعی \cite{camp} و غیره اشاره کرد.  
مشابه با  مدل تراوش استاندارد در تئوری تراوش خود‌راه‌انداز در حالت اولیه، مکان‌ها با یک احتمال اولیه $\textsc{f}$ با یک پیکربندی اولیه فعال می‌شوند. مکان‌هایی که فعال شدند تا پایان فرایند در همان حالت باقی می‌مانند. در مدل خود‌راه‌انداز با استفاده از یک قاعده فعال‌سازی نقاطی که غیر فعال باقی ماندند را فعال می‌کنیم؛ به این شکل که هر مکان غیرفعال در صورتی که ‌حداقل
 $\textsc{m}$ 
 تا از نزدیک‌ترین همسایه‌هایش فعال باشند، فعال می‌‌شوند. 
 $\textsc{m}$
   آستانه فعالیت برای اشغال شدن مکان‌های تهی است. این فرایند تا جایی پیش می‌رود که هیچ مکانی قابلیت فعال شدن را نداشته باشد. اگر مکان‌های فعال شبکه هم‌مرتبه با کل نقاط شبکه شوند می‌گوییم که تراوش در شبکه اتفاق افتاده‌ است (شکل \ref{fig:boot}). نتایج این مدل  تابع بعد شبکه 
   $(\textsc{d})$، احتمال اولیه 
   $(\textsc{f})$ 
   و پارامتر آستانه فعالیت 
   $(\textsc{m})$
    است \cite{grav}. 
 
 \begin{figure}[htbp]
 \hspace*{0cm}
 \centering
 %\begin{minipage}[b]{0.4\textwidth}
 \includegraphics[width=0.3\linewidth, height=35mm]{boot1.png}\centering(الف)    
 \includegraphics[width=0.3\linewidth, height=35mm]{boot2.png}\centering(ب)
 \includegraphics[width=0.3\linewidth, height=35mm]{boot3.png}\centering(ج)
 \caption[نمایش تراوش خودراه‌انداز مربوط به شبکه مربعی] {\footnotesize نمایش تراوش خودراه‌انداز  مربوط به شبکه مربعی با آستانه فعالیت 
 $\textsc{m} = 2$
 : (الف) موقعیت اولیه شبکه، مکان‌های فعال اولیه با با مربع‌های سیاه‌رنگ مشخص شده اند; (ب) مکان‌هایی که می‌توانند با $2$ همسایه فعال شوند با علامت ضربدر مشخص شده‌اند; (ج) مکان‌های مشخص شده نیز فعال شده‌اند \cite{kozma}.}
 \label{fig:boot}
 \end{figure}
 
 در یک شبکه 
 $\textsc{d}$
  بعدی، یک نقطه گذار در  
  $\textsc{f}_{\textsc{c}}$
   وجود دارد که تابع بعد شبکه و آستانه فعالیت است 
   $\textsc{f}_{\textsc{c}} = \textsc{f}(\textsc{d,m})$
   . در این مقدار، زمانی که
     $\textsc{f} > \textsc{f}_\textsc{{c}}$
      تراوش اتفاق می‌افتد و برای 
      $\textsc{f} < \textsc{f}_{\textsc{c}}$
       تراوش اتفاق نمی‌افتد \cite{kozma}. نقطه 
       $\textsc{f}_{\textsc{c}}$
       ، نقطه‌ای است که در آن خوشه شروع به شکل گرفتن می‌کند. کمتر از نقطه گذار 
       $\textsc{f}_{\textsc{c}}$
        جزیره‌های کوچکی از مکان‌های فعال وجود دارد که به هم متصل نیستند و به همین دلیل خوشه به هم پیوسته در شبکه دیده نمی‌شود. اما بعد از آن جزیره‌ها به هم وصل شده و تشکیل خوشه بی‌نهایت را می‌دهند.
 
   در شبکه مربعی درجه هر مکان چهار است. یعنی هر مکان به چهار همسایه چپ و راست، بالا و پایین خود متصل است. هر مکان دو حالت می‌پذیرد: فعال و یا غیر فعال. ابتدا تعدادی از مکان‌ها را با احتمال اولیه 
   $\textsc{f}$ فعال می‌کنیم\RTLfootnote{در شبکه مربعی می‌توانیم به جای کلمه فعال کردن از اشغال شدن مکان‌ها نیز استفاده کنیم.}. همانطور که در فصل قبل نیز گفته شد، در این حالت با نسبت دادن هر مکان به یک عدد تصادفی و مقایسه آن عدد با احتمال اولیه مکان‌ها فعال می‌شوند. بنابراین کسری از مکان‌های فعال اولیه ($\textsc{f}$)در شبکه خواهیم داشت. در مرحله بعد با استفاده از قوانین تراوش خودراه‌انداز دینامیک را روی شبکه اثر می‌دهیم. یک مکان غیر فعال را گزینش می‌کنیم. با در نظر گرفتن یک آستانه‌ی فعالیت 
   $\textsc{m}$ 
   برای مکان‌ها، در صورتی که مکان غیر فعال در نزدیک‌ترین همسایه‌های خود به تعداد حداقل 
   $\textsc{m}$
    مکان فعال داشته باشد می‌تواند فعال شود. فرایند تا جایی پیش می‌رود که مکانی شرط فعال شدن برایش وجود نداشته باشد. بعد از اتمام فرایند، دو کار را بررسی می‌کنیم. اول اینکه اندازه  تعداد کل نقاط فعال در شبکه را  بر حسب تغییرات نقاط فعال اولیه  به دست می‌آوریم و سپس از میان کل نقاط فعال، آن نقاطی  که با یکدیگر همسایه هستند و از به هم پیوستن آن‌ها خوشه بزرگ در شبکه تشکیل می‌شود را پیدا می‌کنیم.  
 
 نمودار (\ref{fig:lattice})  منحنی‌ مربوط به اندازه خوشه به هم پیوسته را برای یک شبکه با اندازه‌های متفاوت نشان می‌دهد. پارامتر 
  $\textsc{S}_{\textsc{gc}}$
   و
    $\textsc{E}$  به ترتیب بزرگ‌ترین خوشه به هم پیوسته و تعداد آنسامبل‌ها را مشخص می‌کند. نمودار تعیین شده برای آستانه‌ی فعالیت 
  $\textsc{m} = 3$
   با تعداد آنسامبل $\textsc{E}=100$ 
     رسم شده است. با توجه به این نمودار می‌بینیم که در این مورد نیز مانند حالت تراوش استاندارد با افزایش اندازه شبکه نقطه گذار به نقطه گذار واقعی نزدیک می‌شود و نیز گذار پیوسته را نیز در رفتار خوشه مشاهده می‌کنیم. 
 
 \begin{figure}[htbp]
 \hspace*{0cm}
 \centering
 %\begin{minipage}[b]{0.4\textwidth}
 %\includegraphics[width=0.4\linewidth,height=55mm]{lattice2.eps}\centering(الف)  
 \includegraphics[width=11cm , height=8cm]{lattice3.eps}
 \caption[نمودار بزرگترین خوشه به هم پیوسته مربوط به تراوش  خودراه‌انداز برای شبکه مربعی] {\footnotesize  نمودار بزرگترین خوشه به هم پیوسته مربوط به تراوش  خودراه‌انداز برای شبکه مربعی با اندازه‌های متفات برای آستانه‌ی فعالیت  
 $\textsc{m}=3$.}
 \label{fig:lattice}
 \end{figure}
 


\section{تراوش خودراه‌انداز بر روی شبکه نورونی}
 بعد از بررسی تراوش استاندارد در شبکه تصادفی در این بخش به معرفی و عملکرد تراوش خودراه انداز در شبکه نورونی می‌پردازیم. 
 در شبکه نورونی، با اعمال تحریک خارجی اولیه به شبکه، مجموعه‌ کوچکی از نورون‌ها فعال می‌شوند که این تعداد می‌تواند منجر به فعال شدن تعداد زیادی از نورون‌ها در شبکه شوند. مدل تراوشی که در بررسی این شبکه‌ها مورد بررسی قرار می‌گیرد از نوع خودره‌انداز است. در این مدل همان‌طور که برای شبکه مربعی در نظر گرفته‌ایم فعال شدن نورون‌ها وابسته به همسایه‌های فعالش است. با این تفاوت که در شبکه مربعی یک راس تنها با چهار همسایه خود در ارتباط است و همسایه‌ها نیز مکانشان مشخص است، اما در شبکه‌های نورونی تعداد همسایه‌های فعال نامعلوم و تصادفی جایگزیده‌اند.
 

 شبکه نورونی، شبکه بزرگی شامل اتصالات بسیار زیاد از نورون‌هاست. در این مجموعه بزرگ گروه‌های کوچکی وجود دارند که در آن نورون‌ها به یکدیگر متصل هستند و فعالیت می‌کنند. فعالیت این نورون‌ها نه تنها تاثیرپذیر از همان مجموعه کوچک است بلکه ممکن است ناشی از نورون‌هایی باشد که در خارج از این جمعیت هستند و نیز به طور مستقیم به جمعیت کوچک نورون‌ها متصل نیستند. به موجب این امر می‌توان با بررسی ارتباط‌هایی که بین نورون‌ها وجود دارد کمیت‌هایی از شبکه از جمله میانگین اتصال هر نورون ($\left\langle \textsc{k} \right\rangle$)، توزیع درجه (
 $\textsc{P}_{\textsc{k}}$
 ) و بزرگترین خوشه فعال (
 $\textsc{g}$
 ) در شبکه را بدست آورد \cite{sori}. 
 تراوش خودراه‌انداز در شبکه‌ نورونی زمانی اتفاق می‌افتد که برای آتش کردن یک نورون حتما باید در اطرافش نورون‌های فعال به اندازه‌ای وجود داشته باشد تا بتوانند آن نورون را فعال کنند. در این مدل با یک ولتاژ خارجی تعدادی از نورون‌ها فعال می‌شوند. این تعداد تا زمانی که شبکه به حالت پایای خود برسد فعال باقی می‌مانند. نورون‌های فعال در حالت اولیه به وسیله مکانیزم فعال‌سازی قادر به فعال کردن همسایه‌های غیر فعال هستند. در این حالت نیز وجود نقطه گذار که رفتار شبکه را در آن نقطه مشخص می‌کند، وجود و عدم وجود خوشه فعال به هم پیوسته را نشان می‌دهد.
 دو عامل در شبکه وجود این گذار را برای ما مشخص می‌کنند. عامل اول به اتصال بین نورون‌ها مربوط است و پارامتر کنترل تعریف می‌شود. پارامتر کنترل در واقع تعداد ورودی‌های لازم برای برای فعال کردن یک نورون غیر فعال در نظر گرفته می‌شود. در ادامه از آستانه فعالیت به جای پارامتر کنترل استفاده می‌کنیم. با این پارامتر ارتباط بین اجزا در شبکه کاهش  یا افزایش می‌یابد. زمانی‌ که آستانه فعالیت افزایش می‌یابد، به دلیل آنکه نورون غیر فعال نمی‌تواند  تعداد 
 $\textsc{m}$ 
 نورون فعال در اطراف خود پیدا کند خاموش  می‌ماند. به همین منوال این شرط برای نورون‌های دیگر نیز ارضا نمی‌شود.  با ادامه این روند از آنجایی که نورون‌ها غیر فعال باقی می‌مانند  اتصالی بین آن‌ها برقرار نمی‌شود و ارتباط بین آن‌ها کا‌هش می‌یابد. در حالی که برای 
 $\textsc{m}$های پایین به دلیل آنکه شرط فعال شدن برای نورون‌های غیر فعال راحت‌تر می‌شود، نورون‌های بیشتری فعال می‌شوند و بنابراین ارتباط افزایش می‌یابد. عامل دوم تعداد نورون‌هایی است که به تحریک پاسخ می‌دهند تا بتوانند فعال شوند و آن را به عنوان پارامتر نظم تعریف می‌کنیم. از آنجایی که پارامتر نظم، دو محیط با فاز‌های مختلف را نشان می‌دهد، حضور یک جهش ناگهانی  در این پارامتر به معنای وجود بزرگترین خوشه فعال به هم پیوسته خواهد بود.  
 
 
\subsection{بررسی نتایج تجربی}
چیدمان آزمایشگاهی برای بدست آوردن نتایج مورد نظر از بافت‌های عصبی اولین بار توسط پاپا\LTRfootnote{Papa} در آزمایشگاه راه‌اندازی شد\cite{papa}. برای انجام آزمایش، قسمتی از بافت‌های جوان از نورون‌های هیپوکمپ\LTRfootnote{hippocampal} موش گرفته شده و آن را به مدت $3 - 2$ هفته در یک محفظه شیشه‌ای پرورش می‌دهند تا ارتباط میان نورون‌ها در بافت شکل گیرد. دو الکترود  به دو طرف بافت متصل است و پالس‌های دو قطبی به اندازه 
$20 \textsc{msec}$
 به بافت‌ها از طریق الکترود‌ها وارد می‌شود و به سبب آن نورون‌ها تحریک می‌شوند\RTLfootnote{پالس دو قطبی یک پالس منفی و مثبت است که قسمت منفی باعث مهار و قسمت مثبت باعث تحریک ولتاژ می‌شود. به عبارت دیگر با اعمال پالس مثبت و منفی نورون‌ها به آستانه فعالیت نزدیک و یا از آن دور می‌شوند.} \cite{sorian}.
با اعمال هر پالس، جریان کنترل و ولتاژ به تدریج بین هر دو پالس افزایش می‌یابد. فعالیت نورون‌ها توسط اسیلوسکوپ فلئورسنس اندازه گیری می‌شود. مزیت استفاده از این اسیلوسکوپ دقت بسیار زیاد، بی تاثیر بودن اختلال بر روی آن و آسیب نرساندن به بافت‌های نورونی‌ است. اسیلوسکوپ فلئورسنس با شناساگر کلسیم،‌ شارش جریان و تغییرات ولتاژ در قسمت‌های مختلف بافت را به وضوح نمایش می‌دهد (شکل \ref{fig:setop}). 
\begin{figure}[htbp]
\hspace*{0cm}
\centering
%\begin{minipage}[b]{0.4\textwidth}
\includegraphics[width=0.3\linewidth, height=45mm]{setap.png}
\includegraphics[width=0.3\linewidth, height=45mm]{felo.png}    
\caption[تصویرسازی فلئورسنس از یک بافت نورونی و شمایی از چیدمان آزمایشگاهی] {\footnotesize (a) تصویرسازی فلئورسنس از یک بافت نورونی. قسمت‌های روشن جسم سلولی، اتصال‌های نورونی و شاخه‌های دندریتی را نشان می‌دهد. (b) شمایی از چیدمان آزمایشگاهی \cite{sorian}.}
\label{fig:setop}
\end{figure}

شبکه نورونی متشکل از نورون‌های تحریکی و مهاری است. نورون‌های تحریکی باعث تحریک ولتاژ و نورون‌های مهاری باعث کاهش ولتاژ نورون‌های دیگر می‌شوند. نورون پس‌سیناپسی دارای گیرنده‌هایی است که اطلاعات را از نورون‌های پیش‌سیناپسی دریافت می‌کنند. این گیرنده‌ها همانطور که در گذشته ذکر شده است دارای دو نوع تحریکی و مهاری است.
 $\textsc{NMDA}$
 \LTRfootnote{N-methyl-D-aspartete} و 
 $\textsc{AMPA}$
 \LTRfootnote{alpha-amino-3-hydroxy-5-methyl-4-isoxazolepropionic acid} دو نوع گیرنده تحریکی و 
 $\textsc{GABA}$
 \LTRfootnote{Gamma-aminobutyric acid} گیرنده نوع مهاری شناخته شده است. در طی انجام آزمایش محققان برای تقویت و تضعیف قدرت سیناپسی نورون‌ها از موادی برای مسدود کردن و یا فعال کردن گیرنده‌ها استفاده می‌کنند. از جمله موادی که برای تضعیف و مسدود کردن گیرنده‌های تحریکی 
 $\textsc{AMPA}$
  مورد استفاده قرار می‌گیرد اعمال   $\textsc{CNQX}$ \LTRfootnote{6-cyano-7-nitroquinoxaline-2,3-dione} به بافت‌های نورونی است. از طرفی گیرنده‌های مهاری 
  $\textsc{GABA}$
   نیز توسط $\mu M$40
بایکوکولین\LTRfootnote{biguguline} (که در واقع غیر فعال کننده گیرنده
 $\textsc{GABA}$
  است) مسدود می‌شوند. اعمال   $\textsc{CNQX}$  به شبکه موجب تغییراتی در اتصال و نیز آستانه فعالیت نورون‌ها می‌شود. واکنش شبکه به غلظت   $\textsc{CNQX}$  داده شده به آن به عنوان کسری از نورون‌های $\phi$ تعریف می‌شود که این نورون‌ها با تحریک خارجی فعال می‌شوند. با توجه به شکل (\ref{fig:CNQX}) مشاهده می‌کنیم در حالتی که شبکه کاملا به هم متصل است متناسب با
   $\textsc{CNQX} = 100\textsc{nM}$
    و زمانی که نورون‌ها کاملا جداگانه‌ از هم رفتار می‌کنند
     $\textsc{CNQX} = 700\textsc{nM}$ 
     می‌باشد \cite{sori}.
\begin{figure} [htbp]
\centering
\includegraphics[width=9cm , height=3cm]{CNQX.png} 
\caption[تاثیر   $\textsc{CNQX}$  بر روی بافت‌های نورونی] {\vspace{-0.01}\footnotesize تاثیر   $\textsc{CNQX}$  بر روی بافت‌های نورونی. در غلظت‌های پایین اندازه خوشه هم‌مرتبه با کل شبکه است و به تدریج در غلظت‌های بالا اندازه خوشه کم می‌شود \cite{sori}.}
\label{fig:CNQX}
\end{figure}

 زمانی که غلظت   $\textsc{CNQX}$  پایین است، همه نورون‌ها به هم متصل هستند و بنابراین کمترین ولتاژ تحریکی به شبکه باعث فعال شدن مجموعه‌ی بزرگی از نورون‌ها می‌شود. در این زمان رشد ناگهانی‌ در  تعداد نورون‌های فعال شبکه دیده می‌شود و شبکه را از حالتی که در آن هیچ نورونی فعال نیست و یا تعداد بسیار کمی از آن‌ها فعالند به سمت حالتی با تعداد نورون‌های فعال هم مرتبه با خود شبکه سوق می‌دهد. این پرش بزرگ نمایشگر اندازه بزرگترین خوشه به هم پیوسته از نورون‌های فعال در شبکه است. در مقابل زمانی‌ که غلظت   $\textsc{CNQX}$  به تدریج افزایش می‌یابد به علت تضعیف قدرت سیناپسی بین نورون‌ها، از تعداد نورون‌های متصل به هم کاسته شده و نورون‌ها به شکل جداگانه از هم رفتار می‌کنند. از آنجایی که این نورون‌ها نمی‌توانند از نورون‌های فعال اطراف خود ورودی دریافت کنند، بنابراین برای فعال شدن آن‌ها احتیاج به ولتاژ بالا می‌باشد، به اندازه‌ای که ولتاژ هر نورون به ولتاژ آستانه فعالیت خود برای آتش کردن برسد. به همین خاطر تعدادی از نورون‌ها ممکن است با ولتاژ ورودی کم و تعدادی نیز با ولتاژ ورودی زیاد به آن مقدار آستانه برسند. اما در مجموع همه نورون‌ها در یک مقدار خاص میانی به آستانه می‌رسند. با در نظر گرفتن این موضوع و نتایجی که از داده‌های تجربی بدست آمد (شکل \ref{fig:s(a)})، مشاهده می‌کنیم که در غلظت‌های بالا می‌توان توزیع آستانه فعالیت برای نورون‌ها را گاوسی در نظر گرفت. محققان طی انجام آزمایشات متعدد نشان داده‌اند که داده‌های به دست آمده با تابع خطا\LTRfootnote{error function} به شکل
  $\phi(v) = 0.5 + 0.5erf(\dfrac{v-v_{0}}{\surd2\times\sigma})$ 
  قابل برازش است. این تابع آستانه آتش کردن نورون‌ها را به صورت تابعی از یک توزیع گاوسی با میانگین
   $\textsc{v}_{0}$
    و پهنای
     $2\sigma$ 
     نشان می‌دهد.
      $v_{0}$ 
       آستانه فعالیت میانگین نورون‌ها است. بنابراین توافق شده است که توزیع گاوسی برای آستانه فعالیت نورون های شبکه نورونی توزیعی قابل قبول و مناسبی است.  
       $\phi_{v}$ 
       نیز تعداد نورون‌هایی ا\phiست که به تحریک خارجی 
       $v$ 
       برای فعال شدن پاسخ می‌دهند \cite{sorian}.

\begin{figure}[htbp]
\hspace*{0cm}
\centering
%\begin{minipage}[b]{0.4\textwidth}
\includegraphics[width=0.45\linewidth, height=55mm]{data2.png}\centering(الف)   
\includegraphics[width=0.4\linewidth, height=50mm]{giant.png}\centering(ب)
\caption[واکنش شبکه به اعمال   $\textsc{CNQX}$ ] {\footnotesize
 (الف) واکنش شبکه به اعمال   $\textsc{CNQX}$ . در غلظت‌های پایین تعداد نورون‌هایی که به تحریک پاسخ می‌دهند تا فعال شوند زیاد است و بنابراین شبکه به یکباره فعال می‌شود. اما با افزایش غلظت   $\textsc{CNQX}$  چون نورون‌ها جدا از هم رفتار می‌کنند و اتصال بین آن‌ها ضعیف می‌شود شبکه دیرتر فعال می‌شود. اندازه پرش در  منحنی‌ها اندازه بزرگ‌ترین خوشه به هم پیوسته از نورون‌ها را نشان می‌دهد. (ب) نمودار مربوط به اندازه بزرگترین خوشه به هم پیوسته بر حسب غلظت   $\textsc{CNQX}$ . در غلظت‌های پایین اندازه خوشه برابر با یک است و به تدریج با افزایش آن اندازه خوشه کوچک شده و در نهایت در غلظت‌های بالا به صفر می‌رسد \cite{sori}.}
\label{fig:s(a)}
\end{figure}

آزمایشاتی که بر روی بافت‌های نورونی صورت می‌گیرد در واقع برای مشاهده و اندازه‌گیری بزرگترین خوشه به هم پیوسته از نورون‌های فعال در شبکه است. این خوشه بزرگترین کسر از نورون‌های فعال در شبکه است که با یکدیگر نسبت به تحریک وارد شده به آن‌ها فعال می‌شوند و به یکبارهتمام شبکه را پوشش می‌دهند. با توجه به قسمت (الف) شکل (\ref{fig:s(a)}) تاثیر   $\textsc{CNQX}$  را بر روی بافت‌های نورونی مشاهده می‌کنیم. تزریق   $\textsc{CNQX}$  به بافت‌ها اتصال بین آن‌ها را تضعیف کرده و نورون‌ها به تدریج از هم جدا می‌شوند. کاسته شدن اتصال بین نورون‌ها به معنای کاهش اندازه خوشه به هم پیوسته در شبکه تلقی می‌شود و هنگامی که غلظت   $\textsc{CNQX}$  خیلی زیاد شود نورون‌ها کاملا از هم جدا هستند و بنابراین خوشه‌ای در شبکه وجود نخواهد داشت. در مقابل برای غلظت پایین و نزدیک به صفر   $\textsc{CNQX}$   که شبکه کاملا به هم متصل است، اندازه خوشه بزرگ‌ترین مقدار خود را دارد. در این زمان گذار گسسته‌ای را در رفنار خوشه می‌بینیم. در این هنگام زمانی که نورون‌ها هیچ‌گونه فعالیتی ندارند سایز خوشه هم‌مرتبه با کل شبکه در نظر گرفته می‌شود. در قسمت  (ب) شکل (\ref{fig:s(a)})  مشاهده می‌کنیم که حضور خوشه در شبکه ناشی از غلظت پایین   $\textsc{CNQX}$  است. در این مقدار اندازه خوشه بزرگ‌ترین مقدار خود را دارد. کوچک شدن جهش در منحنی واکنش به منزله کاهش اندازه خوشه می‌باشد. بنابراین می‌توان گفت زمانی که اندازه خوشه به صفر رسیده است به علت اعمال   $\textsc{CNQX}$  در غلظت‌های بالا، اتصال‌ بین نورون‌ها از بین رفته و به موجب آن خوشه‌ای در شبکه دیده نمی‌شود. این امر به وضوح از نمودار‌های فوق قابل مشاهده است.


\subsection{تاثیر \textsc{CNQX} بر آستانه فعالیت}
آستانه فعالیت، تعداد نورون‌های فعال مورد نیاز برای آتش کردن یک نورون غیر فعال می‌باشد. طبق گفته‌های پیشین افزایش   $\textsc{CNQX}$  و نیز کاهش قدرت سیناپسی بین نورون‌ها سبب جدا شدن نورون‌ها از یکدیگر می‌شود. به موجب این امر، تعداد نورون‌های فعال در شبکه به تدریج کم می‌شود. در این صورت از تعداد نورون‌های فعال در همسایگی نورون غیر فعال نیز کم می‌شود. بنابراین از آنجایی که نورون‌ها به سختی قادر خواهند بود تا با شرط آستانه بالا خود را فعال کنند، رفته رفته از تعداد نورون‌هایی که قادر به فعال شدن هستند کم و اندازه خوشه‌ای که تولید می‌شود به تدریج کوچک می‌شود. بنابراین هنگامی که آستانه فعالیت برای نورون‌ها را بالا می‌بریم شبکه سخت‌تر فعال شده و به نسبت آن اندازه خوشه نیز کاهش می‌یابد.
 \begin{figure} [htbp]
\centering
\includegraphics[width=12cm , height=5cm]{ss.png} 
\caption[تاثیر آستانه فعالیت بر اندازه خوشه] {\vspace{-0.01}\footnotesize تاثیر آستانه فعالیت بر اندازه خوشه. دایره‌های روشن نورون‌های فعال و تیره نورون‌های غیرفعال و فلش‌ها ارتباط سیناپسی بین نورون‌ها را نشان می‌دهند. آستانه فعالیت برای این شبکه
 $\textsc{m} = 2$
  در نظر گرفته شده است.  با افزایش
   $\textsc{m}$
    می‌بینیم که تعداد نورون‌های فعال به تدریج کم شده و اندازه خوشه نیز به سمت صفر پیش می‌رود. برای 
    $\textsc{m} = 8$
     می‌بینیم که هیچ نورون فعال و خوشه‌ای در شبکه وجود ندارد \cite{cohen}.}
\label{fig:ss}
\end{figure}


همانطور که  از شکل (\ref{fig:ss}) مشاهده می‌کنیم زمانی که آستانه فعالیت کوچک است و نیز ساختار شبکه به طور کامل شکل گرفته است، به خاطر حضور تعداد نورون‌های فعال فراوان، یک نورون غیر فعال به راحتی می‌تواند آتش کند. آتش کردن یک نورون می‌تواند منجر به فعال شدن کل شبکه شود. در این زمان بزرگترین خوشه فعال در شبکه را از نورون‌های فعال خواهیم دید. اما زمانی که ارتباط سیناپسی بین نورون‌ها کاهش یابد (به خاطر حضور   $\textsc{CNQX}$ )، آستانه فعالیت 
$\textsc{m}$
 افزایش پیدا می‌کند. تا زمانی که آستانه فعالیت به مقدار بحرانی خود نرسد، می‌توانیم خوشه بزرگ را در شبکه اندازه‌گیری کنیم. اما بعد از آن به خاطر افزایش مقدار   $\textsc{CNQX}$   نورون‌ها از یکدیگر جدا می‌شوند و خوشه‌های مجزا با اندازه‌ای کوچک‌تر در شبکه به وجود می‌آید؛ و در نهایت در
 $\textsc{m}$های بزرگ (تزریق مقدار بسیار زیاد   $\textsc{CNQX}$ ) ارتباط سیناپسی بین نورون‌ها کاملا قطع شده و خوشه‌ای در شبکه شکل نمی‌گیرد. 
 
در فصل بعد مدل سازی شبکه نورونی و نیز نتایجی را که از شبیه‌سازی بدست آورده‌ایم بررسی خواهیم کرد.

  
 
 
 

 \newpage 
\textbf{خلاصه‌ی فصل سوم}    \begin{itemize}
\item از خواص اصلی تراوش پیدا کردن تعداد کل نقاط فعال و نیز پیدا کردن بزرگترین خوشه به هم پیوسته از مکان‌‌های فعال در شبکه است.
\item تراوش پیوندی و جایگاهی دو نوع از متعارف‌ترین و معمول‌ترین تعریف تراوش  در حالت استاندارد و در دو بعد به حساب می‌آیند.
\item در هر دو نوع تراوش پیوندی و جایگاهی، شبکه با یک احتمال اولیه $\textsc{f}$ ساخته می‌شود و هدف ساختن چنین ترکیبی پیدا کردن مسیری است که مکان‌های فعال در شبکه را به هم مرتبط کند. $\textsc{f}$ احتمال مکان‌های فعال اولیه در شبکه است.
\item در تراوش پیوندی، یال‌ها و در تراوش جایگاهی، مکان‌ها با یکدیگر در ارتباط هستند. 
\item تراوش  خودراه‌انداز نوع دیگری از تراوش است که در آن مکان‌ها طی قاعده‌ای فعال می‌شوند. در این نوع، مکان‌های غیرفعال با نگاه‌کردن به همسایه‌های فعال خود و با در نظرگرفتن آستانه فعالیت، قادر به فعال شدن هستند.
\item شبکه نورونی از جمله شبکه‌هایی است که تئوری تراوش خودراه‌انداز در آن کاربرد دارد. با اعمال این  مدل بر روی شبکه بزرگترین خوشه فعال از نورون‌ها که با یکدیگر در ارتباط هستند مشخص می‌شوند. 
\item در کار تجربی که تا کنون انجام شده است، با اعمال   $\textsc{CNQX}$  سیناپس‌ها را تضعیف می‌کنند.
این ماده، روی ارتباط‌های نورونی‌ و نیز روی آستانه فعالیت نورون‌ها تاثیرگذار است. 
\end{itemize}








 


 









 
 
\chapter{تراوش خود‌راه‌انداز در  گراف‌های تصادفی}
شبکه نورونی از جمله شبکه‌های پیچیده‌ای است که برای به دست آوردن بزرگترین خوشه به هم پیوسته از نورون‌های فعال  از تئوری تراوش خود‌راه‌انداز استفاده می‌شود.
در فصل گذشته با بررسی کار‌های تجربیِ  مدل تراوش  روی بافت‌های نورونی مشاهده کردیم که مسدود کردن و یا فعال کردن سیناپس‌ها چه تاثیری بر خوشه‌ی فعال در شبکه می‌گذارد. همچنین تاثیر آستانه‌ی فعالیت را روی شبکه و اندازه خوشه مشاهده کردیم. در این فصل  تراوش خودراه‌انداز را ابتدا روی  گراف ER و سپس  ساختار گاوسی  مربوط به شبکه نورونی بررسی خواهیم کرد. همچنین آستانه یکنواخت و نیز آستانه با توزیع گاوسی را بررسی خواهیم کرد. 

\section{شبکه ER با میانگین درجه رئوس یکنواخت}
شبکه تصادفی ER را با N راس در نظر می‌گیریم. با انتخاب یک  احتمال اولیه p به طور تصادفی بین هر دو جفت راس اتصالی برقرار می‌کنیم. برای هر راس میانگین درجه نیز در نظر می‌گیریم که ساختار شبکه با این توزیع توصیف می‌شود. به خاطر داریم که در حد N‌های بزرگ  برای شبکه‌‌ای با میانگین درجه همگن (پوآسونی) احتمال  اولیه با میانگین درجه اتصال میان راس‌ها ارتباط مستقیم دارد.
با این شرایط ساختمان شبکه را می‌سازیم. برای نمایش اتصال بین راس‌ها نیز از ماتریس مجاورت استفاده می‌کنیم.
 هرچه احتمال اولیه بزرگتر باشد، تعداد راس‌هایی که به هم متصل هستند نیز بیشتر خواهد بود.
بعد از ساخت شبکه تصادفی دینامیک را روی شبکه اثر می‌دهیم. این دینامیک مربوط به فعال و یا غیر فعال بودن راس‌ها در شبکه است. نورون‌ها در شبکه دو حالت را می‌پذیرند. برای نمایش حالت‌ها، بردار حالت Sرا به شکل زیر تعریف می‌کنیم،
\begin{equation}
{S_{i} = 0 , 1}
\end{equation}

زمانی که نورون غیرفعال باشد S_{i} = 0  و زمانی که فعال باشد .S_{i} = 1 

می‌دانیم که هر نورون برای آتش کردن نیازمند ولتاژی  است  تا بتواند آن را به آستانه  آتش کردنش برساند.  آستانه آتش کردن هر نورون بسته به نوع نورون‌ها (به عنوان مثال مهاری و یا تحریکی بودن آن‌ها) متفاوت خواهد بود. اگر ولتاژ تحریکی را با v_{i} و آستانه‌ی فعالیت هر نورون را باf نشان دهیم، می‌توان گفت که برای رسیدن به آستانه و در نهایت آتش کردن نورون باید مقدار ولتاژ تحریکیv_{i} بزرگ‌تر از مقدار آستانه‌ی فعالیتfباشد. i نماینده نورون iام است. همچنین در یک تعریف دیگر،fرا احتمال اولیه فعال شدن نورون‌ها در نظر می‌گیریم که متناظر با کسری از نورون‌های فعال اولیه در شبکه است. با این احتمال تعدادی از مکان‌ها فعال می‌شوند و این تعداد تا زمانی‌ که شبکه به حالت پایا برسد فعال باقی ‌می‌مانند. حالت پایا زمانی‌ است که دیگر هیچ نورونی قابلیت فعال شدن را ندارد  و بعد از آن شبکه فعالیتی نمی‌کند و در همان حالت خود باقی می‌ماند. بعد از آنکه با احتمال اولیه مجموعه‌ای از راس‌ها فعال شدند، کسری از راس‌های غیر فعال در شبکه باقی می‌مانند که می‌خواهیم آنها را طی قاعده‌ای فعال کنیم. برای این راس(نورون)‌ها نیز آستانه‌ی فعالیتی(m) جدا از آستانه‌ی فعالیت نورون‌های اولیه تعریف می‌کنیم. بدین معنی که هر راس(نورون)‌ غیر فعال در اطراف خود باید حداقل به اندازه m راس(نورون) فعال داشته باشد تا بتواند ورودی لازم برای آتش کردن را از همسایه‌های فعال خود دریافت کند و فعال شود. این عمل برای همه مکان‌ها در گام‌های مختلف تکرار می‌شود و در هر گام تعداد مکان فعال تولید شده  را به مکان‌ گام‌های قبل اضافه می‌کنیم و آن‌ها را در بردار حالت S با حالت 1 نشان می‌دهیم.

در یک شکل نمادین می‌توانیم بنویسیم: 
\begin{equation}
\text{if}~~~~~v_{i}\geq f \longrightarrow S_{i} = 1
\end{equation}


و می‌توانیم به این شکل بگوییم که اگر احتمال اولیه بزرگتر از آستانه‌ی فعالیت باشد حالت آن راس را در بردار حالت 1 در نظر می‌گیریم.
بعد از این مرحله نوبت به فعال کردن راس‌ها توسط همسایه‌های فعال است. یک راس غیر فعال را انتخاب می‌کنیم. ابتدا بررسی می‌کنیم که آن راس با چه راس‌های دیگر در ارتباط است. با نظر به آنها و شمارش راس‌هایی که از میان آنها فعال هستند، در صورتی که تعدادشان حداقل به اندازه تعداد آستانه‌ی فعالیت شبکه باشند راس غیر فعال نیز فعال می‌شود. بنابراین می‌توان نشان داد:  
\begin{equation}
\text{if}~~~~~m: \sum_{i=1}^{n} A_{ij}S_{j}\geq m \longrightarrow S_{i} = 1
\end{equation}
در اینجا می‌توانیم رابطه‌ای را میان احتکمال مکان‌های فعال اولیه و احتمال‌ فعال شدن هر مکان به شکل زیر بیان کنیم:
\begin{equation} 
\phi(f) = f + (1-f)\psi_{m}(\phi) \label{eq1}
\end{equation}
در این رابطهfاحتمال مکان‌های اولیه فعال،1-f احتمال وجود نورون‌های غیر فعال شبکه هستند. \psi_{m}(\phi) احتمالی است که نورون‌های غیر فعال با داشتن حداقل m همسایه فعال، فعال می‌شوند و \phi نیز احتمال کل فعال شدن یک نورون را نشان می‌دهد. در واقع این رابطه فعال شدن شبکه ابتدا با یک تحریک خارجی اولیهfو سپس پخش شدن با اثر همسایه‌هایش را برایمان روشن می‌سازد. رابطه (\ref{eq1}) گویای این است که\phi(f) در fهای کوچک  به طور خطی باfرشد می‌کند و مقدارش بین  \phi(0) = 0 و \phi(1) = 1 متغیر است.
 به زبان دیگر، زمانی که احتمال اولیهfصفر است هیچ راسی در شبکه فعال نیست و امکان فعال شدن برای راس‌های دیگر نیز وجود ندارد. بنابراین \phi(0) = 0. در مقابل اگر احتمال اولیه بیشترین مقدار خود را داشته باشد (f = 1)، همه راس‌ها با همان احتمال اولیه فعال می‌شوند. از این رو کل مکان‌های شبکه فعال هستند و \phi(1) = 1. از طرفی، تا زمانی که شبکه به طور جداگانه رفتار کند و ارتباطی بین راس‌ها وجود ندشته باشد، با رشدfهمان تعداد راسی که از ابتدا فعال شدند در شبکه به عنوان  راس‌های فعال نهایی باقی می‌مانند. چرا که نمی‌توانند شرط فعال شدن برای دیگر  راس‌ها را مهیا کنند. بنابراین 
 \phi به طور خطی باfزیاد می‌شود (\phi(f) = f). اما زمانی‌‌که به یک مقدار خاص ازfبرسیم که در آن نقطه گذاری در شبکه حاصل شود برایمان قابل قبول خواهد بود که وابستگی بهfاز بین می‌رود. به عبارت دیگر، در این زمان فعال‌سازی شبکه وابسته به همسایه‌های فعال می‌شود و این یعنی این که اتصال بین راس‌ها وابسته به دینامیک است. این نکته قابل ذکر است نقطه‌ای که در آن  از حالت خطی خارج می‌شویم نقطه گذار شبکه است.

با اعمال  آستانه‌ی فعالیت یکنواخت و گاوسی به شبکه ER احتمال تعداد کل مکان‌های فعال و نیز اندازه بزرگ‌ترین خوشه فعال به هم پیوسته را پیدا می‌کنیم.
\subsection{آستانه‌ی فعالیت یکنواخت}
زمانی که نتایج تجربی را بررسی می‌کردیم به این نکته پی بردیم افزایش تعداد مکان‌های اولیه و  نیز آستانه‌ی فعالیت، اندازه خوشه به هم پیوسته از مکان‌های فعال را کاهش می‌دهد. در اینجا می‌خواهیم ببینیم که آیا نتایج شبیه‌سازی با نتایج تجربی هم‌خوانی دارد؟

این‌گونه می‌توانیم ادامه دهیم؛ شبکه‌ای تصادفی از N راس فعال و غیر فعال داریم که با احتمال p به هم وصل شده‌اند. تعداد راس‌های اولیه فعال که با تحریک خارجیfفعال شده‌اند درصدی از تعداد کل راس‌های شبکه است. راس‌های غیر فعال می‌توانند از طریق دینامیک حاکم بر شبکه فعال شوند. دینامیک فعال‌سازی این راس‌ها به این صورت است که راس‌های غیر فعال از طریق فعالیت همسایه‌های فعال خود در صورتی که بتوانند بر آستانه‌ی فعالیت غلبه کنند فعال می‌شوند. این فرایند تا زمانی که راس‌های غیر فعال قابلیت فعال شدن داشته باشند ادامه خواهد داشت. زمانی سیستم به حالت پایا می‌رسد که دیگر دینامیک بیش‌تر از آن جلو نمی‌رود  و دیگر امکان فعال شدن راسی وجود نخواهد داشت. مراحل  فعال‌سازی  تا اتمام فرایند گام به گام تکرار می‌شود. در انتهای فرایند تعداد کل راس‌های فعال شمرده می‌شود. این تعداد  هم شامل راس‌هایی است که با تحریک اولیه فعال شدند و هم آنهایی که در طول اجرای دینامیک توانستند فعال شوند. نمودار (\ref{fig:sites}) اندازه احتمال کل مکان‌های فعال  برای آستانه‌های مختلف  را برحسب احتمال نقاط فعال اولیه نشان می‌دهد. منجنی‌های زیر برای شبکه‌ای با10000 راس و با میانگین درجه 
\lqngle k \rangle = 5  رسم شده  است.  به عبارت دیگر هر راس به طور میانگین با5 راس دیگر اتصال دارد. میانگین‌گیری روی 100
 نمونه انجام شده است. S_{a} در نمودار، زیر تعداد کل مکان‌های فعال را مشخص می‌کند.

\begin{figure} [htbp]
\centering
\includegraphics[width=11cm , height=8cm]{sites.eps} 
\caption [نمودار مربوط به احتمال کل نقاط فعال بر  حسب  تغییر احتمال اولیه برای شبکه ER]{\footnotesize نمودار مربوط به احتمال کل نقاط فعال بر  حسب  تغییر احتمال اولیه برای شبکه ER با 10000 راس و  میانگین درجه اتصال راس‌ها  \lqngle k \rangle = 5 .}
\label{fig:sites}
\end{figure}
نمودار (\ref{fig:sites}) گویای این است که با افزایش احتمال اولیه f  نقاط فعال در شبکه افزایش پیدا می‌کند. با این حال می‌بینیم با افزایش آستانه‌ی فعالیت m تعداد نقاط فعال کاهش می‌یابد. علت نیز این است که با زیاد شدن تعداد نقاط فعال، آستانه‌ی فعالیت نیز برای نورون‌ها افزایش می‌یابد و شرطی که برای فعال کردن نورون غیر فعال در نظر می‌گیریم سخت‌تر می‌شود. به عبارت دیگر یک راس غیر فعال در اطراف خود امکان پیدا کردن نقاط فعال  به اندازه آستانه‌ی فعالیت را نخواهد داشت. به همین دلیل تعداد کل نقاط فعال  شبکه به طور خطی با تعداد نقاط اولیه فعال رشد می‌کند. در نمودار (\ref{fig:sites})  آستانه  5 این رفتار را نشان می‌دهد. همچنین در این نمودار رشد ناگهانی  در احتمال اولیه مکان‌های  فعال شبکه در fهای کوچک مشاهده می‌کنیم. در واقع برای یک مقدار معین f، قبل از آن، راس فعالی در شبکه وجود ندارد یا تعدادشان به اندازه‌ای نیست که شبکه را فعال کند. اما بعد از آن،  شبکه به یک باره فعال می‌شود. همانطور که می‌بینیم با افزایش مکان‌های فعال، اندازه جهش  رفته رفته کاهش می‌یابد و در fهای بالا از بین می‌رود. به روشنی می‌توان دریافت که احتمال  اولیه f باعث تغییر در نقطه گذار می‌شود. آستانه m نیز نقطه احتمال fرا که در آن گذار اتفاق می‌افتد مشخص می‌کند.

از آنجایی که آستانه‌ی فعالیت شرط فعال شدن مکان‌‌های غیر فعال  را آسان یا سخت می‌کند، چیزی که صریحا با آن قابل تغییر است اندازه خوشه ایست که در شبکه به وجود می‌آید. نمودار زیر وجود و عدم وجود خوشه را نشان می‌دهد. این نمودار برای شبکه ER با 10000 راس و با میانگین درجه \lqngle k \rangle = 5 رسم شده است.  میانگین‌گیری روی 100 نمونه انجام شده است.
\begin{figure} [htbp]
\centering
\includegraphics[width=11cm , height=8cm]{max.eps} 
\caption [اندازه خوشه به هم پیوسته از مکان‌های فعال بر حسب احتمال اولیه مکان‌های فعال با آستانه یکنواخت  شبکه ER]{\footnotesize اندازه خوشه به هم پیوسته از مکان‌های فعال بر حسب احتمال  اولیه مکان‌های فعال با آستانه یکنواخت برای شبکه ER با تعداد راس N = 10000 و میانگین درجه \lqngle k \rangle = 5. افزایش آستانه‌ی فعالیت m و  احتمال اولیهfسبب کاهش اندازه خوشه می‌شوند.}
\label{fig:max}
\end{figure}

با توجه به نمودار (\ref{fig:sites})، می‌توان ‌روند تغییرات فعال‌سازی بر حسب آستانه‌های مختلف را استنتاج کرد. با توجه به مباحث قبل در این مورد هم تاثیر احتمال اولیهfرا مشاهده می‌کنیم. در احتمال کوچک به دلیل کم بودن تعداد مکان‌های فعال شکل گرفتن خوشه ممکن نبوده و به همین دلیل اندازه‌اش صفر است (منحنی روی محور افقی  روی صفر حرکت می‌کند). به عبارت دیگر در $f$های کوچک جزیره‌‌های کوچکی از مکان‌های فعال داریم که به هم متصل نیستند. به همین دلیل اندازه خوشه تا مقادیری ازfصفر است. اما با افزایشfو زیاد شدن تعداد مکان‌های فعال در یک نقطه خاص  منحنی رشد ناگهانی را متحمل می‌شود که نشان از فعال شدن سریع شبکه  است. نقطه‌ای که در آن شبکه یکباره فعال می‌شود گذار تراوش را نشان می‌دهد. زمانی که اندازه مکان‌های فعال از یک مقدار معینfعبور کنند  باعث به وجود آمدن جزیره‌های بزرگ‌تر می‌شوند که  اتصالی بین آن‌ها برقرار خواهد شد . همان‌طور که مشخص است قبل از  آنکه گذار تراوش  اتفاق بیفتد  خوشه‌ی  بزرگ نداریم و بعد از آن خوشه شکل می‌گیرد. شایان ذکر است که آستانه‌ی فعالیت نقش موثرتری در تشکیل شدن و نشدن خوشه به هم پیوسته ایفا می‌کند. در مطالب گذشته به این نکته اشاره شده است که افزایش آستانه فعال شدن شبکه را سخت‌تر می‌کند. چرا که با زیاد شدن مقدار آستانه، مکان‌های غیر فعال دسترسی کمتری به مکان‌های فعال اطراف خود دارند و بنابراین نمی‌توانند فعال شوند و شبکه با افزایش آستانه‌ی فعالیت به شکل جداگانه رفتار می‌کند. این عامل سبب کوچک شدن اندازه خوشه می‌شود و در آستانه‌های بالاتر همانطور که از نمودار مشخص است خوشه از بین می‌رود. رفتار خوشه نیز در آستانه‌های بالا گذار پیوسته‌ای را از خود نشان می‌دهد.

\subsection{آستانه‌ی فعالیت گاوسی}
با نقطه نظر به مباحث پیشین، می‌دانیم که در یک شبکه نورونی ‌‌همه نورون‌ها با یک آستانه‌ی فعالیت آتش نمی‌کنند. بلکه تعدادی از آنها آستانه‌ی فعالیتی پایین‌تر از حد میانگین دارند  و تعدادی نیز با آستانه‌ای بالاتر از حد میانگین فعالیت می‌کنند. بنابراین شبکه حول یک مقدار میانگین رفتار می‌کند. با در نظر گرفتن شواهد تجربی می‌توان نتیجه گرفت که رفتار شبکه از یک توزیع گاوسی برخوردار است و نورون‌ها حول میانگین این توزیع رفتار می‌کنند. با تغییر دو پارامتر تابع گاوسی، میانگین و پهنای توزیع گاوسی، رفتار شبکه را بررسی می‌کنیم. ساختار شبکه در این مورد نیز شبکه ER با میانگین درجه اتصال‌های یکنواخت انتخاب شده است. 
در شبکه‌ای که توزیع گاوسی برای آستانه وجود دارد مکان‌هایی که آستانه آنها از مقدار میانگین کمتر باشد به معنای کوچک بودن آستانه‌ی فعالیت برای آن مکان است؛ و نیز ذکر کردیم اگر آستانه‌ی فعالیت برای راسی کوچک باشد شرط فعال شدن برای راس‌های دیگر راحت‌تر خواهد بود.

 
\begin{figure}[htbp]
\hspace*{0cm}
\centering
%\begin{minipage}[b]{0.4\textwidth}
\includegraphics[width=11cm , height=8cm]{setactive.eps}\centering
%\includegraphics[width=0.4\linewidth, height=55mm]{tap.eps}\centering(a)    
\caption [منحنی مربوط‌ به احتمال  نهایی تعداد کل مکان‌های فعال  برای شبکه ER با آستانه گاوسی]{\footnotesize 
احتمال  نهایی تعداد کل مکان‌های فعال برای شبکه ER با تعداد رئوس  $10000$ بر حسب  تغییرات احتمال مکان‌های اولیهfبرای  پهناهای گاوسی  ( $\sigma$های) متفاوت با میانگین آستانه $\mu = 6$.}
\label{fig:ER}
\end{figure}

نمودار (\ref{fig:ER}) تغییرات احتمال  نهایی تعداد کل مکان‌های فعال  در شبکه بر حسب تغییرات احتمال مکان‌های فعال اولیهfرا نشان می‌دهد. $\sigma$ و $\mu$ به ترتیب پهنای گاوسی و میانگین آستانه را مشخص می‌کنند.  در این نمودار، میانگین آستانه فعالیت  ثابت در نظر گرفته شده و منحنی برای مقادیر مختلف پهنای گاوسی رسم شده است. پهنای گاوسی در واقع تعداد راس‌ها با آستانه‌های مختلف را برایمان مشخص می‌کند. با توجه به این گفته، در توجیه رفتار این شکل می‌توان گفت با افزایش پهنای $\sigma$، تعداد راس‌هایی که دارای آستانه کوچک‌تر هستند بیشتر می‌شود. در نتیجه  آستانه‌ی فعالیت مکان‌ها کاهش یافته و بنابراین شبکه در ‌$f$های کوچک‌تر رشد سریع‌تری را از خود نشان می‌دهد.  اما زمانی که $\sigma$ کوچک را برای توزیع آستانه در نظر می‌گیریم با نزدیک شدن به حالت یکنواخت راس‌های کمتری آستانه کوچک دارند و بنابراین راس‌های غیر فعال کمتری می‌توانند فعال شوند. با این استدلال می‌توان دریافت که در  $\sigma$های کوچک به علت کم بودن تعداد مکان‌های فعال  در اطراف یک راس‌ غیرفعال و برآورده نشدن شرط فعال‌سازی برای آن، خوشه به هم پیوسته از مکان‌های فعال نداریم و برای  $\sigma$های بالاتر و پهن‌تر، شاهد به  وجود آمدن خوشه هستیم. از طرفی با تغییر میانگین آستانه نقطه گذار نیز جابه‌جا می‌شود. از آنجایی که آستانه‌های پایین‌تر مسبب رشد سریع شبکه می‌شود می‌توان پی پرد که در میانگین‌های کوچکتر شبکه زودتر فعال می‌شود و نقطه گذارش نسبت به حالتی که میانگین آستانه بزرگتر است، کوچکتر می‌باشد.همچنین برای دو منحنی با مقدار $\sigma = 4$  و  $\sigma = 5$ مقداری از آستانه گاوسی در قسمت منفی قرار گرفته است. می‌توان اینگونه استنتاج کرد که نورون‌ها حتی در آستانه‌های منفی نیز مقداری غیر صفر دارند؛ و یا می‌توان گفت مقدار $f = 0$  برای نورون‌هایی که دارای آستانه منفی هستند به منزله آستانه فعالیت بالا محسوب شده و تعدادی از آنها در این مقدار نیز فعال می‌شوند. به همین دلیل مشاهده می‌کنیم که حتی در  $f = 0$ نیز احتمال کل مکان‌های فعال و نیز اندازه خوشه به هم پیوسته از مکان‌های فعال مقداری غیر صفر را دارند.
%قابل ذکر است، زمانی که میانگین آستانه با میانگین درجه راس‌ها هم مرتبه می‌شود شبکه تنها با مقدار اولیهfرشد می‌کند. به عبارت دیگر، با افزایش آستانه میانگین، فعال شدن راس‌ها در آستانه‌های بالاتر سخت‌تر می‌شود و به همین دلیل شبکه با همان مقدار اولیه نقاط که فعال شدند رشد می‌کند. 

\begin{figure}[htbp]
\hspace*{0cm}
\centering
%\begin{minipage}[b]{0.4\textwidth}
\includegraphics[width=11cm , height=8cm]{setmax.eps}
%\includegraphics[width=0.4\linewidth, height=55mm]{max1.eps}\centering(a)    
\caption [منحنی مربوط به اندازه برگترین خوشه به هم پیوسته در شبکه ER با آستانه گاوسی]{\footnotesize
 منحنی مربوط به اندازه برگترین خوشه به هم پیوسته در شبکه ER با تعداد راس‌های $N = 10000$ و میانگین درجه $\lqngle k \rangle = 10$ برای میانگین آستانه $\mu = 6$. در اینجا نیز مشاهده می‌کنیم با افزایش پهنا تعداد نقاط فعال زیاد شده و باعث رشد سریع شبکه می‌شود.در $\sigma$های بزرگ اندازه خوشه صفر می‌شود و شبکه بااحتمال   اولیه رشد می‌‌کند. }
\label{fig:ER1}
\end{figure}
نمودار (\ref{fig:ER1}) اندازه خوشه به هم پیوسته را بر حسب تغییرات احتمال اولیه فعال نشان می‌دهد. با توجه به نتایج قبلی که برای اندازه خوشه به دست آمده است، مشاهده می‌کنیم اندازه خوشه در مقادیر کوچکfصفر است؛ و این یعنی اینکه شرط تشکیل خوشه زمانی که تعداد مکان‌های اشغال شده در شبکه کم باشد ارضا نمی‌شود. نکته‌ای که در اینجا می‌توان به آن اشاره کرد این است که به خاطر غیر یکنواخت بودن آستانه‌ی فعالیت مکان‌ها و نیز به این خاطر که آستانه‌ی فعالیت تعدادی از مکان‌ها از حد میانگین بالاتر و تعدادی از آنها نیز آستانه‌شان از حد میانگین کمتر است مشاهده می‌کنیم که با افزایش پهنا ابتدا منحنی‌ها سریع رشد کرده و در نهایت دیرتر به حالت پایا می‌رسند. در این حالت نیز می‌توان تاثیر مقدار میانگین را به وضوح مشاهده کرد که در میانگین کوچک‌تر از آنجایی که آستانه کوچک می‌شود احتمال فعال شدن مکان‌ها بیشتر شده و شبکه سریع‌تر رشد خواهد کرد. بنابراین  نمودار (\ref{fig:ER1}) مصداقی است بر این ادعا که همه نورون‌ها با یک آستانه یکسان آتش نمی‌کنند.


%\begin{figure}[htbp]
%\hspace*{0cm}
%\centering
%\begin{minipage}[b]{0.4\textwidth}
%\includegraphics[width=0.4\linewidth, height=55mm]{.eps}\centering(b)
%\includegraphics[width=0.4\linewidth, height=55mm]{sigma1.eps}\centering(a)    
%\caption{\footnotesize منحنی‌های مربوط به اندازه کل نقاط فعال در شبکه: هر دو نمودار برای یک پهنای مشترک به دست امده است. (a) ماتریس اتصال‌ها با احتمال $0.02$ و (b) ماتریس اتصال‌ها با احتمال $0.03$ درست شده است}
%\label{fig:ER2}
%\end{figure}

\subsection{تغییر اندازه بزرگترین خوشه به هم پیوسته بر حسب تغییرات آستانه‌ی فعالیت}
گفتیم از جمله پارامتر‌هایی که در تغییر رفتار خوشه به هم پیوسته نقش اساسی دارد آستانه‌ی فعالیت است. همان‌طور که در نمودار‌های قبل دیدیم، افزایش آستانه‌ی فعالیت باعث کاهش اندازه خوشه می‌شود. این گفته خود را با استفاده از نمودار (\ref{fig:SM})   تایید می‌کنیم. نمودار (\ref{fig:SM})  برای شبکه ER با تعداد راس $N = 1000$، میانگین درجه $\lqngle k \rangle = 15$ برای مقادیر مختلف کثری از مکان‌های اولیه فعال به دست آمده است. 
\begin{figure}[htbp]
\hspace*{0cm}
\centering
\includegraphics[width=11cm , height=8cm]{S_TH.eps}
%\includegraphics[width=0.4\linewidth, height=55mm]{max1.eps}\centering(a)    
\caption [تغییر اندازه خوشه بر حسب آستانه‌ی فعالیت برای شبکه ER با آستانه‌ی فعالیت یکنواخت]{\footnotesize تغییر اندازه خوشه بر حسب آستانه‌ی فعالیت یکنواخت برای شبکه ER با تعداد راس $N = 1000$ و میانگین درجه $\lqngle k \rangle = 15$  می‌بینیم که با افزایش آستانه‌ی فعالیت اندازه خوشه کوچک می‌شود و در نهایت آستانه‌های بزرگ به صفر می‌رسد. }
\label{fig:SM}
\end{figure}\\
در نمودار (\ref{fig:SM}) رفتار کاهشیِ اندازه خوشه‌ی به هم پیوسته را مشاهده می‌کنیم. می‌بینیم که در آستانه‌های کوچک اندازه خوشه بزرگ است و با افزایش آستانه اندازه آن به صفر می‌رسد.  همچنین در بحث‌های پیشین  نشان دادیم که با افزایش مکان‌های اولیه فعال اندازه خوشه کوچک می‌شود و درfهای بزرگ شبکه تنها با مکان‌های اولیه فعال شده پیش می‌رود. به وضوح می‌بینیم که این گفته با  منحنی‌های زیر هم‌خوانی دارد. هرچهfبزرگ‌تر می‌شود اندازه خوشه تا جایی پیش می‌رود که همان مقدار اولیهfرا داریم.  

اکنون موردی را در نظر می‌گیریم که در آن آستانه‌ی فعالیت گاوسی بر شبکه حاکم است و در این مورد نیز اندازه خوشه به هم پیوسته را بر حسب تغییرات میانگین آستانه‌ی فعالیت نشان می‌دهیم. از نمودار 
(\ref{fig:EM1})  به این نتیجه می‌رسیم که با اعمال آستانه گاوسی به شبکه به علت غیر همسان بودن آستانه‌ی فعالیت برای مکان‌ها، اندازه خوشه به هم پیوسته‌ی فعال برای مقادیر کوچک میانگین آستانه بیشترین مقدار خود را دارد و با افزایش آستانه میانگین اندازه خوشه نیز به کوچک می‌شود و در نهایت در میانگین آستانه‌های بزرگ به صفر می‌رسد.  
\begin{figure}[htbp]
\hspace*{0cm}
\centering
\includegraphics[width=11cm , height=8cm]{EM.eps}
%\includegraphics[width=0.4\linewidth, height=55mm]{max1.eps}\centering(a)    
\caption [تغییر اندازه خوشه بر حسب آستانه‌ی فعالیت برای شبکه ER با آستانه گاوسی]{\footnotesize تغییر اندازه خوشه بر حسب آستانه‌ی فعالیت گاوسی برای شبکه ER با تعداد راس $N = 1000$ و میانگین درجه $\lqngle k \rangle = 15$ . می‌بینیم که با افزایش آستانه‌ی فعالیت اندازه خوشه کوچک می‌شود و در نهایت در میانگین آستانه‌های بزرگ به صفر می‌رسد. }
\label{fig:EM1}
\end{figure}

\section{نمودار فاز شبکه اردوش-رنی}
در پایان نمودار فاز  سه بعدی شبکه ER  را بر حسب سه متغیر میانگین درجه $\lqngle k \rangle$، میانگین آستانه فعالیت  $\mu^*$ و احتمال نهایی تعداد کل مکان‌های فعال ($S_{a}$) رسم کردیم. نمودار زیر احتمال  نهایی تعداد کل مکان‌های فعال در گراف تصادفی ER را با $10000$ راس نشان می‌دهد. محور عمودی توزیع  درجه اتصال‌ راس ها ($\lqngle k \rangle$) و محور افقی میانگین آستانه‌ی فعالیت  ($\mu^*$) را نشان می‌دهد. شبکه با تعداد اولیه $f = 0.15$ فعال شده است و پهنای گاوسی نیز $\sigma = 0.5$ در نظر گرفته شده است. برای مقادیر کوچک $\mu$ تعداد زیادی از مکان‌ها به علت کوچک بودن آستانه می‌توانند فعال شوند و بنابراین در میانگین‌های کوچک فعال سازی خود به خودی به شکل دسته جمعی به طور ناگهانی صورت می‌گیرد. با افزایش $\mu$  تعداد مکان‌های فعال نیز کم می‌شود. از طرفی با دقت به این نمودار فاز در می‌یابیم که هرچه میانگین درجه اتصال راس‌ها نیز کوچک باشد تعداد کمتری از راس‌ها قادر خواهند بود فعال شوند و همچنین تعداد محدودی از راس‌های غیر فعال را نیز فعال کنند. بنابراین در $\lqngle k \rangle$های کوچک نیز تعداد راس‌های فعال کمتری نسبت به $\lqngle k \rangle$های بزرگ مشاهده می‌کنیم. هرچه از ناحیه تیره به سمت ناحیه روشن می‌رویم افزایش مکان‌های فعال را می‌بینیم. 
\begin{figure}[htbp]
\hspace*{0cm}
\centering
\includegraphics[width=11cm , height=8cm]{DP.eps}
%\includegraphics[width=0.4\linewidth, height=55mm]{max1.eps}\centering(a)    
\caption [نمودار فاز مربوط به شبکه ER]{\footnotesize 
نمودار فاز مربوط به شبکه ER. این نمودار احتمال نهایی تعداد کل مکان‌های فعال را برای یک مقدار میانگین آستانه و میانگین درجه نشان می‌دهد. محور افقی میانگین آستانه و محور عمودی میانگین درجه راس‌ها را نشان می‌دهد.}
\label{fig:DP}
\end{figure}

\section{شبکه با ساختار گوسی}
نتایجی که از داده‌های قبل به دست آمد، برای شبکه ER با میانگین درجه اتصال یکنواخت و همگن بوده است. در ادامه کار می‌خواهیم ببینیم با تغییر در ساختمان اتصال‌های شبکه و اعمال دینامیک یکسان با شبکه‌ ER در حصول نتایج چه تفاوت‌هایی دیده می‌شود. در فصل پیش  ذکر کردیم، شبکه‌های نورونی را با گراف تصادفی مدل می‌کنند. در این مدل نورون‌ها، راس‌ها و اتصال‌های سیناپسی، یال‌های گراف را تشکیل می‌دهند. همانطور که می‌دانیم در شبکه‌های با  اندازه  بسیار بزرگ  برای  بررسی رفتار شبکه از توزیع گاوسی استفاده می‌کنند. از آنجایی که  اتصال‌های شبکه نورونی  نیز در شبکه تعدادی بیشمار  است، ساختمان شبکه نورونی را به شکل  گاوسی ساختیم. به این صورت که با تعریف تابع گاوسی، میانگین درجه اتصال‌های شبکه را گاوسی  درنظر گرفتیم. به این صورت که با تعریف یک تابع گاوسی برای راس‌های شبکه، به هر راس  مقدار میانگبن و توریع درجه خاصی را نسبت دادیم و راس‌های شبکه با این دو مقدار داده شده به  یکدیگر متصل می‌شوند. سپس  با اعمال  دینامیک قبل شامل بررسی آستانه یکنواخت و گاوسی، ‌نتایج را به صورت زیر به دست آوردیم. در همه نمودار‌های زیر پهنا و  میانگین  توزیع گاوسی به ترتیب $10$ و $35 $ در نظر گرفته شده است. مقدار $\lqngle k \rangle = 35$  میانگین درجه اتصال‌ راس‌ها با در نطر گرفتن پهنای گاوسی و میانگین داده شده است که در نمودار‌ها آورده شده است.

\subsection{آستانه‌ی فعالیت یکنواخت}
 با نظر به نمودار (\ref{fig:gaussian})، زمانی که شبکه را با  میانگین درجه گاوسی  می‌سازیم، نسبت به شبکه با میانگین درجه همگن،‌  حالتی که توزیع گاوسی داریم شبکه سریع‌تر نسبت به توزیع همگن رشد می‌کند. همچنین نقطه گذار نیز در این ساختار در $f$های بزرگ‌تر اتفاق می‌افتد.  
 نمودار (\ref{fig:gaussian})  این نتیجه را می‌دهد که شبکه با ساختار گاوسی نسبت به شبکه با ساختار پواسونی سریع‌تر به حالت پایای خود می‌رسد. در آستانه‌های کوچک شبکه زود‌تر رشد می‌کند و هرچه اندازه آستانه بزرگ‌تر می‌شود شبکه با احتمال اولیه فعال رشد می‌کند.
 دلیل این رفتار نیز به علت بزرگ بودن میانگین درجه ساختار گاوسی نسبت به شبکه ER با میانگین درجه همگن است. به عبارت دیگر، در ساختار گاوسی به دلیل بزرگ بودن میانگین درجه، یک راس با راس‌های زیادی در شبکه اتصال دارد. نتایج برای شبکه با $N=10000$ راس و با میانگین‌گیری روی $E=100$  نمونه به دست آمده است. همچنین میانگین درجه برای این ساختار را $\lqngle k \rangle = 35$  در نظر گرفته‌ایم. از هر دو نمودار به دست آمده در (\ref{fig:gaussian}) به این نتیجه می‌رسیم در شبکه‌هایی که میانگین درجه آنها زیاد است اندازه بزرگ‌ترین خوشه به هم پیوسته از مکان‌های فعال با احتمال نهایی تعداد کل مکان‌های فعالدر شبکه یکسان است. 
 
\begin{figure}[htbp]
\hspace*{0cm}
  %\begin{minipage}[b]{0.4\textwidth}
\includegraphics[width=11cm , height=8cm]{gaussavtive1.eps}\centering(الف)
\includegraphics[width=11cm , height=8cm]{gaussmax1.eps}\centering(ب)
\caption [احتمال نهایی تعداد کل مکان‌های فعالو بزرگ‌ترین خوشه‌ی فعال به هم پیوسته برای شبکه  با ساختار گاوسی]{\footnotesize منحنی مربوط به
 احتمال نهایی تعداد کل مکان‌های فعال(الف) و بزرگ‌ترین خوشه‌ی فعال به هم پیوسته (ب) برای شبکه  با ساختار گاوسی.  در این منحنی میانگین درجه برابر با $\lqngle k \rangle  = 35$
 برای شبکه با $N = 10000$ راس در نظر گرفته شده است.}
\label{fig:gaussian}
\end{figure}
\newpage 
 \subsection{آستانه‌ی فعالیت گاوسی}
 از آنجایی که همه نورون‌ها یکسان نیستند و هر نورون آستانه فعال شدنش با نورون‌های دیگر متفاوت است بنابر این با در نظر گرفتن دینامیک با آستانه گاوسی رفتار شبکه را بررسی می‌کنیم.
 
 با توجه به نمودار (\ref{fig:gauss}) این واقعیت برای ما روشن می‌شود که نورون‌ها با یک آستانه یکسان آتش نمی‌کنند. بلکه تعدادی از آن‌ها با آستانه‌ای بزرگتر وتعدادی نیز با آستانه‌ای کوچک‌تر از آستانه‌ی فعالیت آتش می‌کنند همچنین مشاهده می‌کنیم با افزایش پهنای آستانه گاوسی ($\sigma$)، شبکه زودتر به حالت پایا می‌رسد و اندازه خوشه به هم پیوسته از مکان‌های فعال  نیز با افزایش پهنای  $\sigma$، بزرگتر می‌شود. $\mu$ آستانه میانگین شبکه  است و میانگین‌گیری نیز روی $E=100$ انجام شده است.
\begin{figure}[htbp] 
\hspace*{0cm}
\centering
\includegraphics[width=11cm , height=8cm]{TGA.eps}\centering(الف)    
\includegraphics[width=11cm , height=8cm]{TGM.eps}\centering(ب)
\caption [منحنی مربوط به توزیع آستانه گاوسی]{\footnotesize
 منحنی مربوط به توزیع آستانه گاوسی برای شبکه با $N=10000$ راس برای دو حالت (الف) احتمال کل مکان‌های فعال و (ب) اندازه بزرگترین خوشه‌ی به هم پیوسته فعال. منحنی‌های مدنظر با میانگین درجه $\lqngle k \rangle = 35$   برای آستانه میانگین  $\mu = 15$ و پهنا‌های گاوسی متفاوت رسم شده است.}
\label{fig:gauss}
\end{figure}

\newpage
\subsection{تغییر اندازه خوشه‌ی به هم پیوسته بر حسب آستانه‌ی فعالیت}
نمودار (\ref{fig:TS})  روند تغییرات اندازه خوشه‌ی به هم پیوسته از مکان‌‌های فعال را بر حسب تغییرات آستانه‌ی فعالیت برای شبکه با ساختار گاوسی با $N=1000$ راس و میانگین درجه $\lqngle k \rangle = 35$  نشان می‌دهد. با توجه به این نمودار می‌بینیم که اندازه خوشه در آستانه‌های کوچک بزرگ است و برای آستانه‌‌های بزرگ‌تر رفتار کاهشی دارد.

\begin{figure}[htbp] 
\hspace*{0cm}
\centering
\includegraphics[width=11cm , height=8cm]{TS.eps}
\caption [منحنی مربوط به تغییرات اندازه خوشه بر حسب تغییرات آستانه‌ی فعالیت مربوط به ساختار گاوسی]{\footnotesize منحنی مربوط به تغییرات اندازه خوشه بر حسب تغییرات آستانه‌ی فعالیت برای شبکه با $N=10000$ راس. }
\label{fig:TS}
\end{figure}
 \newpage
مشابه با شبکه ER در این مورد نیز می‌توانیم تغییرات اندزه خوشه به هم پیوسته از مکان‌های فعال را بر حسب تغییرات میانگین آستانه برای آستانه‌های گاوسی مشاهده کنیم.  نمودار (\ref{fig:MM1})  رفتار تغییر اندازه خوشه را برحسب تغییرات آستانه‌ی فعالیت گاوسی نشان می‌دهد. از مقایسه این نمودار با نمودار (\ref{fig:EM1}) درمی‌یابیم که در شبکه با ساختار گاوسی به علت بزرگ بودن میانگین درجه نسبت به ساختار ER، تغییرات اندازه خوشه بر حسب میانگین آستانه‌ی $\mu$ در آستانه‌های کوچک سریع‌تر به حالت پایا می‌رسد. به همین دلیل اندازه خوشه در ساختار گاوسی برای میانگین آستانه‌های کوچک بزرگ‌تر است. $\sigma$ پهنای آستانه فعالیت گاوسی را نشان می‌دهد.
\begin{figure}[htbp] 
\hspace*{0cm}
\centering
\includegraphics[width=11cm , height=8cm]{MM.eps}
\caption [منحنی مربوط به تغییرات اندازه خوشه بر حسب تغییرات آستانه‌ی فعالیت گاوسی]{\footnotesize منحنی مربوط به تغییرات اندازه خوشه بر حسب تغییرات آستانه‌ی فعالیت گاوسی برای شبکه با $N=10000$ راس. با توچه به}
\label{fig:MM1}
\end{figure}\\\
 
\newpage
\section{نتیجه گیری}
در این رساله به منظور بررسی میزان تاثیر آستانه‌ی فعالیت و دیگر عوامل دخیل در اندازه بزرگ‌ترین خوشه‌ی فعال به هم پیوسته از راس‌ها(نورون‌ها)ی فعال دو ساختار ER با میانگین درجه همگن و ساختار گاوسی با میانگین درجه گاوسی مورد استفاده قرار گرفته است.  
ابتدا  برای درک بهتر تراوش استاندارد را روی شبکه مربعی دو بعدی  و گراف انجام دادیم. مشاهده کردیم که در شبکه مربعی  تنها با بزرگ شدن اندازه شبکه نقطه گذار به مقدار واقعی خود می‌رسد. همچنین گذار پیوسته را در این شبکه دیدیم.  در بررسی نتایج حاصل از تراوش معمولی روی گراف نیز مشاهده کردیم که نقطه گذار در این حالت با معکوس میانگین درجه خود ارتباط مستقیم دارد. همچنین در این مورد نیز گذار پیوسته را در رفتار خوشه به ‌هم پیوسته دیدیم. 


و اما در بررسی نتایجی که از تراوش خود‌راه‌انداز روی شبکه ER و ساختار گاوسی  به دست آوردیم  مشاهده کردیم که با افزایش احتمال  اولیه در شبکه، افزایش ناگهانی در مکان‌های فعال نهایی شبکه صورت می‌گیرد و شبکه زودتر به حالت پایا می‌رسد. با توجه به نمودار (\ref{fig:EG}) می‌بینیم در شبکه‌ای که میانگین درجه راس‌ها بزرگ است و راس‌های بیشتری به هم متصل هستند(ساختار گاوسی)، نسبت به شبکه با میانگین درجه پایین(ساختار ER)، زودتر رشد می‌کند و بااحتمال اولیه کمتری به حالت پایا می‌رسند. در نمودار (\ref{fig:EG}) دو ساختار گاوسی و ER نشان داده می‌شود. ساختار گاوسی با میانگین درجه $\lqngle k \rangle = 35$  و ساختار ER با میانگین درجه $\lqngle k \rangle = 5$ است. دینامیک یکسان با آستانه‌ی فعالیت $m = 5$ نیز روی شبکه اثر می‌گذارد.
\begin{figure}[htbp]
\hspace*{0cm}
\centering
\includegraphics[width=11cm , height=8cm]{EG.eps}
%\includegraphics[width=0.4\linewidth, height=55mm]{max1.eps}\centering(a)    
\caption [مقایسه اندازه خوشه به هم پیوسته برای دو ساختار تصادفی ER و گاوسی]{\footnotesize 
مقایسه اندازه خوشه به هم پیوسته برای دو ساختار تصادفی ER و گاوسی. در این نمودار مشاهده می‌کنیم که با اعمال دینامیک یکسان به دو شبکه با ساختار‌ متفاوت، شبکه‌ی گاوسی به خاطر بزرگ بودن میانگین درجه نسبت به ساختار ER سریع‌تر به حالت پایا می‌رسد. میانگین درجه ساختار تصادفی ER،  $\lqngle k \rangle = 5$ و میانگین درجه شبکه با ساختار گاوسی $\lqngle k \rangle = 35$ است. در هر دو شبکه دینامیک یکسان با آستانه‌ی فعالیت $m = 5$ اعمال شده است.}
\label{fig:EG}
\end{figure}\\\
 همچنین نتایج حاصل از هر دو مدل حاکی از آن است اندازه بزرگ‌ترین خوشه به هم پیوسته از مکان‌های فعال با افزایش آستانه‌ی فعالیت m کاهش می‌یابد و در  نقطه گذار $m = m_{c}$ به صفر می‌رسد. زمانی که آستانه‌ی فعالیت بزرگ‌تر از نقطه گذار باشد شبکه به سمتی پیش می‌رود که دیگر خوشه بزرگی در آن تشکیل نمی‌شود. در این زمان گذار فاز پیوسته‌ای را در رفتار خوشه‌ی فعال به هم پیوسته  دیدیم. این گذار با پارامتر نظم $S_{gc}$ (اندازه برزگترین خوشه‌ی فعال به هم پیوسته) سنجیده می‌شود. به عبارت دیگر می‌توان اینگونه تعبیر کرد؛ اگر اندازه خوشه صفر باشد، آستانه‌ی فعالیت بزرگ‌تر از مقدار  آستانه در نقطه گذار و اگر اندازه خوشه، بزرگ و البته هم‌مرتبه با اندازه شبکه مورد نظر باشد آستانه کوچک‌تر از مقدار  آستانه در نقطه گذار است. در یک تعریف ریاضی‌گونه گفته خود را به شکل زیر می‌گوییم:\\\\
 

\begin{aling}
\begin{center}
\begin{cases}
\[
\text{if}~~~~~S_{gc} = 0 \longrightarrow m = m_{c}.\\
\text{if}~~~~~S_{gc} > 0 \longrightarrow m < m_{c}.           
\end{cases}
\]
\end{center}
\end{aling}
از طرفی، در یک حالت حدی، زمانی که آستانه‌ی فعالیت خیلی بیشتر از  مقدار  آستانه در نقطه گذار باشد $(m >> m_{c})$، شبکه کاملا جداگانه رفتار می‌کند و مکان‌های شبکه(نورون‌ها) فقط با تحریک خارجی اولیه برانگیخته می‌شود. در این صورت برای همه مقادیر احتمال اولیهfتنها یک جواب $\phi = f$  خواهیم داشت که نشان از رفتار خطی شبکه باfاست. همچنین در هر دو مورد رفتار اندازه خوشه به هم پیوسته در نقطه گذار، هم گذار فاز پیوسته و هم گذار فاز ناپیوسته را نشان می‌دهد.

این نکته قابل ذکر است در بررسی مطالعات خود روی شبکه نورونی فرض را بر این گذاشتیم که تمام نورون‌ها تحریکی بوده و نیز نورون‌ها وزن یکسانی برابر با یک دارند. همچنین فرض بر غیرجهتی بودن شبکه نیز شده است.

 به طور خلاصه می‌توان گفت با مقایسه رفتار اندازه بزرگ‌ترین خوشه هم‌بند در دو شبکه ER و شبکه با ساختار گاوسی به این نتیجه رسیدیم که شبکه با ساختار گاوسی به دلیل اتصال‌های بیشتر و میانگین درجه راس‌های بالاتر نسبت به شبکه ER با میانگین درجه یکنواخت، برای فعال شدن ناگهانی شبکه و نیز تشکیل خوشه به هم پیوسته به احتمال اولیهfبزرگ‌تری نیاز دارند. در صورتی که شبکه ER با میانگین درجه اتصال‌های خود در $f$های کوچک به یک‌باره فعال می‌شوند. همچنین آستانه‌ی فعالیتی که برای شبکه با ساختار گاوسی در نظر گرفته می‌شود بزرگ‌تر از آستانه‌ی فعالیت شبکه ER است. 
 
  \section{کار‌های پیش‌رو و پیشنهادها}
   در ادامه کار قصد داریم ابتدا شبکه جهتی را جایگزین شبکه غیر جهتی کنیم و نیز  وزن خاصی را به یال‌های شبکه نسبت می‌دهیم که معرف قدرت سیناپس‌ها باشد. در این صورت میانگین درجه اتصال راس‌ها به گونه‌ای متفاوت خواهد بود و هر راس درجه ورودی و خروجی خود را دارد. همچنین می‌توانیم در مطالعات آتی تاثیر نورون‌های مهاری را نیز روی  انتشار فعال‌سازی نورون‌های شبکه بررسی کنیم. 
   \newpage
 \section{خلاصه فصل چهارم}
 \begin{itemize}
 \item شبکه ER دارای ساختاری با میانگین درجه یکنواخت و میانین درجه کوچک از مرتبه $\lqngle k \rangle = 5$ تا $\lqngle k \rangle = 10 $ است. 
 \item در شبکه ER، برای بررسی رفتار بزرگ‌ترین خوشه به هم پیوسته از مکان‌های فعال آستانه‌های کوچک‌تری نیاز است تا شبکه به یکباره فعال شود. این فعال شدن ناگهانی گذار فازی را در شبکه ایجاد می‌کند که در نقطه گذار  $f$، قبل آن خوشه‌ای نداریم و بعد از آن خوشه فعال به هم پیوسته دیده می‌شود.
 \item شبکه با ساختار گاوسی با میانگین درجه اتصال  بزرگ در $f$های بزرگ‌تر این گذار را از خود نشان می‌دهد. به این خاطر که در 
 $f$های کوچک به دلیل  گسسته بودن و جدا بودن نورون‌ها از هم تشکیل بزرگ‌ترین خوشه  فعال به هم پیوسته در شبکه ممکن نبوده و شبکه با مقدار اولیه نورون‌ها(مکان‌ها) رشد می‌کند.
 \item در هر دو شبکه در نقطه گذار دو نوع گذار فاز پیوسته و گسسته را در رفتار اندازه خوشه و کل مکان‌های فعال  مشاهده کردیم. دیدیم که در آن نقطه رفتارشان پیوسته  است اما همانند گذار فاز گسسته جهش ناگهانی را از خود نشان می‌دهند.
 \item اندازه بزرگ‌ترین خوشه‌ی فعال در شبکه با افزایش آستانه‌ی فعالیت کاهش می‌یابد و در آستانه‌های بالاتر اندازه خوشه به صفر می‌رسد.

 \end{itemize} 








%--------------------------------------------------------------------------------------------------------
% -------------------- DON'T EDIT ----------------------------------------
% the following lines are needed for making the appendixes name correct in the index.
\makeatletter
\def\@makechapterhead#1{%
  \vspace*{50\p@}%
  {\parindent \z@ \centering\normalfont
    \ifnum \c@secnumdepth >\m@ne
      \if@mainmatter
        \huge\bfseries \@chapapp\space \thechapter
        \par\nobreak
        \vskip 20\p@
      \fi
    \fi
    \interlinepenalty\@M
    \Huge \bfseries #1\par\nobreak
    \vskip 40\p@  }}
%--------------------------------------------------------------------------------------------------------
% -------------------- Please EDIT ----------------------------------------

\appendix
%\chapter{عنوان فصل یا پیوست}
متن فصل را در اینجا بنویسید.

\section{عنوان بخش }
متن بخش را می توانید در این ناحیه بنویسید.

\subsection{عنوان زیر بخش }
متن زیر بخش را می توانید در این قسمت بنویسید 
%\chapter{عنوان فصل یا پیوست}
متن فصل را در اینجا بنویسید.

\section{عنوان بخش }
متن بخش را می توانید در این ناحیه بنویسید.

\subsection{عنوان زیر بخش }
متن زیر بخش را می توانید در این قسمت بنویسید
%\include{appendix_3}

\bibliographystyle{unsrt-fa}
% اگر فایل bibtex با پسوند bib حاوی اطلاعات مربوط به مراجع خود با فرمت صحیح bibtex را دارید از خط زیر استفاده کنید و به جای MyReferences نامه فایل خود را بنویسید.

% در این نمونه پایان‌نامه فرض شده است که شما فایل bib حاوی اطلاعات مربوط به مراجع خود با نام MyReferences.bib را دارید.
\bibliography{MyReferences}

% اگر به صورت عادی می‌خواهید ارجاع دهید خط بالا را غیر فعال کرده و قسمت زیر را فعال کنید و طبق مثال عمل کنید (البته این روش حرفه‌ای نیست و توصیه نمی‌شود).
%\begin{thebibliography}{99}

   %
   % در صورتی که می‌خواهید عنوان «واژه‌نامه فارسی به انگلیسی» در فهرست مطالب 
   % وارد شود، علامت «%» را از ابتدای خط زیر حذف کنید.
\addcontentsline{toc}{section}{واژه‌نامه فارسی به انگلیسی }

\begin{center}
\vspace{1.5cm}
\Huge{واژه‌نامه فارسی به انگلیسی}
\vspace{1.5cm}
\end{center}
%\begin{center}
%الف
%\end{center}
اتصال											\dotfill               \lr{Connectivity}                       \\
انحراف معیار                              \dotfill               \lr{Standard Deviation}                       \\
انشعابات درخت‌گونه                         	  \dotfill               \lr{‌Branching Tree}                       \\
آزاد شدن                         					 \dotfill               \lr{Relaese}                       \\




%\begin{center}
%الف
%\end{center}
برازش											\dotfill               \lr{Fitting}                       \\
بردار حالت                             \dotfill               \lr{State-Vector}                       \\





پارامتر نظم                           			\dotfill               \lr{Order Parameter}                         \\
پتانسیل بازگشتی								\dotfill               \lr{Reversal Potential}                      \\
پخش													\dotfill               \lr{Diffusion}                        \\




تاخیر زمانی												\dotfill               \lr{Delay }                         \\
تحریک خارجی                          	 \dotfill               \lr{Externaly Excitation}                         \\
توزیع دوجمله‌ای							\dotfill               \lr{Binomial Distribution}                         \\





حالت پایا                           			\dotfill               \lr{Steady State}                         \\
حلقه                           							\dotfill               \lr{Loop}                         \\




دستگاه عصبی خودکار                        \dotfill               \lr{Outonomic Nervous System}                         \\
دستگاه عصبی مرکزی                           \dotfill               \lr{Center Nervous System}                         \\
دستگاه عصبی محیطی                         \do             \lr{Primeter Nervous System}                         \\





رشد ناگهانی                           						\dotfill               \lr{Jump}                         \\





سیستم عصبی مهره‌داران                    \dotfill               \lr{Vertebrate Nervous System}                         \\





شبکه اجتماعی                           		  \dotfill               \lr{Social Network}                         \\
شبکه بینهایت                       			\dotfill               \lr{Infinite Network}                         \\
شبکه جنگلی                           		  \dotfill               \lr{Forest Network}                         \\
شبکه واقعی                         				\dotfill               \lr{Real Network}                         \\





قدرت سیناپسی                         		\dotfill               \lr{Sinaptic Strength}                         \\

	


	
کانال                         							\dotfill               \lr{Channel}                         \\





گذار فاز                         				\dotfill               \lr{Phase Transition}                         \\






متقارن                         							 \dotfill                \lr{Symetric}                         \\
مغز                         								\dotfill               \lr{Brain}                         \\
مواد متخلخل                        				\dotfill               \lr{Porous Matterials}                         \\
میانگین درجه                         				  \dotfill               \lr{Mean Degree}                         \\





نورون هیپوکمپ                         		    	\dotfill               \lr{Hipocamtal Neuron}                         \\
هم‌زمان                         							   \dotfill               \lr{Syncronous}                         \\



% و به همین ترتیب می‌توانید ادامه دهید.
 % include persian to english dictionary

% --------------------------------------   INFORMATION IN LATIN  ----------------------------------------------
\begin{latin}
\latintitle{‌Bootstrap Percolation on Neuronal Network}
\latinauthor{Soodabeh Azhdar}
% --------------------------------------
% choose and activate one of the following lines
\latindegree{Master's Thesis}
%\latindegree{Ph.D. Thesis}
% --------------------------------------
\latinthesisdate{July 2016}
\latinsupervisor{\begin{center}
Dr. Nahid Azimi\\ Dr. Alireza ValiZadeh
\end{center}}
% If you have advisor, write its name in the following line, otherwise inactive (comment) the line.
%\latinadvisor{Jafar Mostafavi-Amjad} \advisorexisttrue
\latindepartment{Physics}
\latinuniversity{Institute for Advanced Studies in Basic Sciences}
\latincity{Zanjan}
\begin{latinabstract}
\noindent 
Standard percolation theory is the study of clusters behavior in a network. One of the important issues in  percolation theory is finding the size and the emergence point of infinitively large cluster of occupied locations. A generalized model of standard percolation is a model called bootstrap percolation. This model has been introduced to describe development of an activation process on the network in which activation emerges collectively through successive activation of location in them. Activation process in neuronal network is an illustrative example of bootstrap percolation. In this thesis we investigate neurons activation process in a neural network. In such a network, when a neuron reaches its activation threshold and fire, affects its neighbouring neuron. In the bootstrap percolation, a neuron will be activated on the condition that at least $m$ neighbouring neuron of that active. Collective firing of neuron, leads to a connected cluster of active neuron which is a noticeable fraction of the whole neuronal network. The site of the active cluster obtained using bootstrap percolation models and simulations. In this study, for simplicity, this model has been implemented on a random network whit  
 uniform distribution of degree, and then generalized to a network whit Gaussian distribution of degree, which is appropriate for neuronal networks. We see that increasing the number of formerly active neurons causes the number of active neuron to process at the end, and in a point grows dramatically. More over the size of giant component cluster of active neurons, will show a phase transition which is a combination first and second order phase transition. These transition are studied for uniform and Gaussian distribution fraction and for various threshold.
 
\latinkeywords{Activation, Bootsrap Percolation, Giant Connected Component, Phase Transition}

\end{latinabstract}
\makelatintitle
\end{latin}


\end{document}
