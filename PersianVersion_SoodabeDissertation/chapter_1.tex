\chapter{دستگاه عصبی مرکزی}
\section{مقدمه}
هماهنگی بین اعمال و اندام‌های بدن توسط دستگاه‌های ارتباطی که در بدن موجودات سلولی وجود دارد انجام می‌شود. سیستم عصبی با ساز و کار ویژه خود وظیفه این هماهنگی را بر عهده دارد. سلول‌های عصبی از مهم‌ترین و پیچیده‌ترین واحد پردازنده سیستم عصبی مرکزی هستند. اجزا و سازوکار این سلول از موضوعات اساسی مطالعه دستگاه عصبی به شمار می‌آید.
در این فصل دستگاه عصبی مرکزی را مورد بررسی قرار می‌دهیم.
\section{انواع سلول‌ها در بافت‌های عصبی}
دستگاه عصبی مرکزی به دو دسته تقسیم می‌شود:
\begin{itemize}\item سلول عصبی به نام نورون\LTRfootnote{neuron}  که  انتقال دهنده‌ پیام‌‌های عصبی به شمار می‌آید. این سلول‌ها به عنوان سلول‌های تحریک‌پذیر شناخته می‌شوند. نورون‌ها پیام‌های عصبی را به بافت‌ها، اندام‌ها و دیگر نورون‌ها می‌فرستند و از این طریق با آن‌ها ارتباط برقرار می‌کنند.
\item سلول‌های غیرعصبی به نام نوروگلیا\LTRfootnote{neuroglia}
 یا گلوسیت\LTRfootnote{gliocyte} که سلول‌های پشتیبان محسوب می‌شوند و وظیفه محافظت از نورون‌ها را بر عهده دارند. این سلول‌ها در انتقال پیام عصبی نقشی ندارند. سلول گلیا به صورت الکتریکی تحریک نمی‌شوند. نوروگلیا را سلول‌های تحریک‌ناپذیر نیز می‌گویند. 
\end{itemize}تعداد نورون‌ها در مغز انسان در حدود $100$
 میلیارد است و هر نورون به طور متوسط می‌تواند با $10$ هزار نورون دیگر ارتباط برقرار کند. در مقابل تعداد نوروگلیا چندین برابر نورون‌هاست ($5$ یا $10$ برابر تعداد نورون‌ها).
\section{ساختمان اصلی نورون}
هر نورون شامل سه بخش اصلی است: جسم سلولی\LTRfootnote{cell body}، دندریت\LTRfootnote{dendrite}، آکسون\LTRfootnote{axon} . شکل (\ref{fig:neuron}) این سه بخش را به طور واضح نشان می‌دهد.
\begin{figure} [htbp]
\centering
\includegraphics[width=9cm , height=5cm]{neuron.png} 
\caption[نورون و بخش‌های مختلف آن] {\footnotesize نورون و بخش‌های مختلف آن \cite{bear}.}
\label{fig:neuron}
\end{figure}
جسم سلولی یا سوما\LTRfootnote{soma} مرکز اصلی سلول عصبی می‌باشد. دندریت‌ها انشعابات درخت‌گونه هستند و در واقع قسمت اصلی دریافت اطلاعات و سیگنال‌هایی هستند که از دیگر نورون‌ها به نورون نوعی می‌رسد. آکسون نیز سیگنال‌های دریافتی را به سلول‌های دیگر می‌فرستد. طول آکسون در برخی موارد به دو متر نیز می‌رسد \cite{kandel}.
‌ %\subsection{تقسیم بندی نورون}
%نورون‌ها از نظر طرز خارج شدن تارهای عصبی از جسم سلولی به سه گروه تک قطبی\LTRfootnote{Unipolar}، دوقطبی\LTRfootnote{Bipolar} و چندقطبی\LTRfootnote{Moltipolar} تقسیم می‌شوند. این تقسیم بندی اولین بار توسط رامون کاخال\LTRfootnote{Ramon Cajal} انجام شده است \cite{kandel}.
%نورون‌‌های تک قطبی ساده‌ترین نوع نورون‌ها محسوب می‌شوند که از یک جسم سلولی و شاخه‌های متعدد با اندازه‌های یکسان تشکیل شده است. یکی از شاخه‌ها به عنوان آکسون و باقی آن‌ها دندریت نورون به شمار می‌آیند. چنین نورون‌هایی در ساختار موجودات بی‌مهره وجود دارند. در ساختار مهره‌داران در سیستم عصبی خودکار دیده می‌شوند. دستگاه عصبی خودکار، دسته‌ای از نورون‌های حرکتی هستند که فعالیت ماهیچه‌های صاف، تراوش غدد، تپش قلب و به طور کلی اندام‌های درونی را کنترل می‌‌کنند.

%نورون‌های دوقطبی دارای جسم سلولی بیضی‌گون هستند که دو شاخه از آن خارج می‌شود. یکی از آن‌ها مربوط به دندریت است که سیگنال‌ها را دریافت می‌کند و دیگری در نقش آکسون و حاوی اطلاعاتی است که آن‌ها را به طرف سیستم عصبی مرکزی می‌فرستد. بیشتر سلول‌های حساس از جمله سلول‌های چشم و بویایی در این دسته جای می‌گیرند. 

%نورون‌های چندقطبی در سیستم عصبی مهره‌داران دیده می‌شوند. آن‌ها نوعا دارای تک آکسون و دندریت‌های فراوان در اطراف جسم سلولی هستند. نورون‌های بدن انسان در این گروه قرار دارند.

%در بعضی قسمت‌های دستگاه عصبی، نورون‌هایی که فاقد آکسون هستند شناسایی شده‌اند که این نورون‌ها فقط قادرند تحریک عصبی را به نورون‌های مجاور خود منتقل کنند.

%نورون‌ها را بر مبنای عملکردشان می‌توان به سه دسته‌ی نورون‌های حسی\LTRfootnote{Sensory Neuron }، نورون‌های حرکتی\LTRfootnote{Motor Neuron} و نورون‌های رابط\LTRfootnote{Interneuron} تقسیم کرد.نورون‌های حسی پیام‌های عصبی را به طرف دستگاه عصبی هدایت می‌کنند. نورون‌های حرکتی پیام‌ها را از دستگاه عصبی مرکزی دریافت می‌کنند و آن‌ها را به سمت عضلات صاف و اسکلتی و قلبی می‌فرستند و در نهایت نورون های رابط یا نورون‌های واسطه، نورون‌هایی هستند که رابط بین نورون‌ها با یکدیگر در سیستم عصبی است. از جمله این نورون‌ها می‌توان به نورون‌های واسطه در قشر مغز اشاره کرد \cite{kandel}. 

%نمونه‌ نورون‌هایی را که در بالا ذکر شده است در شکل مشاهده می‌کنیم. \begin{figure} [htbp]	\centering	\includegraphics[width=9cm , height=6cm]{types.png} 	\caption{\footnotsize طبقه‌بندی نورون‌ها از نظر شکل و کارکرد \cite{kandel}.} \label{fig:types} \end{figure}

\section{فعالیت نورون}
سلول‌های زنده اختلاف پتانسیلی در دو طرف غشا\LTRfootnote{membrane} دارند.  قشر داخلی و خارجی  غشا توسط لایه چربی از هم جدا می‌شوند.  این لایه نسبت به عبور یون‌های موثری که در تولید پتانسیل عمل نقش دارند انتخابی\LTRfootnote{selective} است. در طول غشا کانال‌هایی وجود دارد که سبب شارش یون‌ها و در نتیجه باعث ایجاد اختلاف پتانسیل در دو سوی سلول می‌شوند. این پتانسیل، پتانسیل غشا نامیده می‌شود و از رابطه زیر بدست می‌آید \cite{ermen}:
\begin{equation}
V_{\textsc{M}} = V_{\textsc{in}} - V_{\textsc{out}}
\end{equation}
در این رابطه  $V_{\textsc{M}}$  اختلاف پتانسیل غشا، $V_{\textsc{in}}$ و 
$V_{\textsc{out}}$ به ترتیب پتانسیل داخل و خارج غشا هستند (شکل \ref{fig:potential}).
\begin{figure} [htbp]
\centering
\includegraphics[width=10cm , height=7cm]{potential.png} 
\caption[پتانسیل غشای سلول] {\footnotesize پتانسیل غشای سلول ناشی از جدایی یون‌های مثبت و منفی در دو طرف غشا \cite{ermen}.}
\label{fig:potential}
\end{figure} 
شارش یون‌ها در دو طرف غشا مسبب ایجاد اختلاف پتانسیل است. هنگامی که نورون هیچ فعالیتی نداشته باشد به اصطلاح در حال استراحت است. اختلاف پتانسیل ناشی از این حالت پتانسیل استراحت\LTRfootnote{resting potential} نامیده می‌شود. در این زمان پتانسیل داخل سلول نسبت به خارج آن منفی‌تر است. علت آن است که غلظت یون پتاسیم مثبت در داخل بیشتر از خارج و غلظت یون سدیم مثبت در خارج بیشتر از داخل است. بنابراین سدیم بر اساس شیب غلظت تمایل به ورود و پتاسیم تمایل به خروج دارد. اما از آنجایی که هیدراته\RTLfootnote{چون عدد اتمی سدیم $(11)$ کوچکتر از پتاسیم $(19)$ است تعداد مولکول‌های آبی که یون سدیم جذب می‌کند بیشتر از یون پتاسیم می‌شود. بنابراین هیدراته سدیم بزرگتر از پتاسیم است.} یون سدیم نسبت به پتاسیم بزرگ‌تر است، تعداد یون‌های پتاسیمی که از سلول خارج می‌شوند بیشتر از تعداد سدیمی است که وارد می‌شوند. به همین دلیل یون‌های منفی داخل سلول خود را بیشتر نشان می‌دهند.

% \begin{figure} [htbp]
%\centering
%\includegraphics[width=7cm , height=5cm]{hydrolize.png} 
%\caption{\footnotesize شمایی از هیدراته یون سدیم. این تصویر مفهوم هیدراته را برای یون سدیم نشان می‌دهد. \cite{bear}.}
%\label{fig:sude}
%\end{figure}

\section{ پتانسیل عمل }
پتانسیل عمل تغییر ناگهانی اختلاف پتانسیل در دو طرف غشای سلول است. این اختلاف پتانسیل از تفاوت غلظت یون‌های مثبت و منفی در دو طرف غشای سلول  حاصل می‌شود. پتانسیل عمل یک پالس الکتریکی به مدت $1$ تا $2$ میلی‌ثانیه و  با دامنه‌ای در حدود صد‌ میلی‌ولت می‌باشد \cite{ermen}. 

با توجه به شکل (\ref{fig:action}) مراحل ایجاد پتانسیل عمل به این صورت است که با اعمال اندک تحریک به سلول، ابتدا کانال‌های سدیم باز شده و یون‌ها وارد سلول می‌شوند. شارش یون‌های سدیم به داخل، محیط آن ‌را مثبت‌تر از بیرون می‌کند و سبب افزایش پتانسیل غشا می‌شود. سپس کانال‌های سدیمی بسته و کانال‌های پتاسیم باز می‌شود. شارش پتاسیم به خارج، محیط بیرون را منفی‌تر از داخل می‌کند و در نتیجه پتانسیل کاهش می‌یابد\RTLfootnote{افزایش پتانسیل ناشی از یون‌های سدیم واقطبیدگی و کاهش پتانسیل ناشی از یون‌های پتاسیم بازقطبیدگی (\LRTfootnote{Repolarization}) نامیده می‌شود.}. در مرحله آخر نیز با بسته شدن کانال پتاسیم پتانسیل غشا به حالت استراحت بر می‌گردد. با تکرار این فرایند پتانسیل عمل در یک غشای نورونی تولید می‌شود. هر پتانسیل عمل، بار‌های مثبت سدیم را وارد غشا می‌کند. یون‌های سدیم به ناحیه مجاور سلول که در حال استراحت است جابجا می‌شوند و آن ناحیه را واقطبیده\LTRfootnote{depolarization} می‌کنند. در واقع جابجایی سدیم مثبت به ناحیه مجاور موجب تغییر ولتاژ و باز شدن کانال‌های سدیمی شده و در نهایت پتانسیل عمل تولید می‌شود. بنابراین پتانسیل عمل به عنوان محرکی برای تولید پتانسیل عمل در ناحیه دیگری می‌شود.
\begin{figure} [htbp]
\centering
\includegraphics[width=10cm , height=7cm]{action.png} 
\caption[مراحل ایجاد پتانسیل عمل] {\footnotesize مراحل ایجاد پتانسیل عمل \cite{ermen}.}
\label{fig:action}
\end{figure}
\section{ سیناپس }
محل اتصال دو نورون به یکدیگر سیناپس\LTRfootnote{synapse} نامیده می‌شود که در آن تبادل اطلاعات از آکسون یک نورون به دندریت نورون دیگر صورت می‌گیرد. نورون  تولید کننده پیام عصبی را نورون پیش‌سیناپسی\LTRfootnote{presynaptic} و نورون  دریافت کننده پیام را نورون پس‌سیناپسی\LTRfootnote{postsynaptic }  می‌گویند.  نورون‌های پیش‌سیناپسی پیام‌ها را از طریق آکسون به پایانه آکسونی\LTRfootnote{axon terminal} می‌فرستند. این اطلاعات به فضای سیناپسی\LTRfootnote{synaptic Cleft} فرستاده و از طریق نورون‌های پس‌سیناپسی دریافت می‌شوند. ارتباط بین نورون‌های پیش‌ و پس‌سیناپسی از طریق مواد شیمیایی با نام انتقال دهنده‌های عصبی\LTRfootnote{neurotransmiter} صورت می‌گیرد. پیام‌های عصبی به دو طریق الکتریکی و شیمیایی منتقل می‌شودن. به همین منظور سیناپس‌ها را در دو دسته  الکتریکی\LTRfootnote{electrical synapse} و شیمیایی\LTRfootnote{chemical synapse} قرار می‌دهند \cite{kandel} .
 
در سیناپس الکتریکی دو نورون توسط کانال‌هایی به نام شکاف پیوندگاه\LTRfootnote{‫gap‬‬ ‫junction‬‬} از هم جدا شده‌اند.  این شکاف آنقدر باریک است که تنها اجازه عبور یون ها ی  بسیار کوچک را  میدهد. در سیناپس الکتریکی یون‌ها از یک نورون به طور مستقیم  و بدون واسطه از کانال شکاف پیوندگاه وارد نورون بعدی می‌شوند و با قرار گرفتن بر  روی غشای نورون باعث واقطبیدگی آن می‌شوند. اگر واقطبیدگی سلول از میزان آستانه فعالیت نورون تجاوز کند کانال‌های یونی وابسته به ولتاژِ\LTRfootnote{voltage-gated ion chanel} نورون پس‌سیناپسی باز می‌شوند و یک پتانسیل عمل تولید می‌شود.  این کانال‌ها دقیقا روبروی هم قرار دارند.  به همین خاطر یون‌ها و دیگر مولکول‌ها به راحتی از یک نورون به نورون دیگر وارد می‌شوند. 
یکی از  ویژگی سیناپس الکتریکی این است که به دلیل سرعت بالای انتقال پالس‌های الکتریکی به طور ناگهانی تعداد بسیار زیادی از نورون‌ها با همدیگر به طور هم‌زمان فعال می‌شوند و این یعنی هم‌زمانی فعالیت الکتریکی نورون‌ها.

در سیناپس شیمیایی، نورون‌های پیش و پس‌سیناپسی به طور کامل توسط  شکاف سیناپسی از هم جدا شده‌اند. 
گفتیم که تاخیر زمانی در سیناپس الکتریکی بسیار ناچیز است. اما این تاخیر  اندک در مورد سیناپس شیمیایی امکان‌پذیر نمی‌باشد. چرا که انتقال پیام از طریق سیناپس شیمیایی نیازمند عبور از چند مرحله است: 
\begin{itemize}
\item آزاد شدن پیام‌رسان‌های عصبی از نورون پیش‌سیناپسی
\item پخش مولکول‌های پیام‌رسان در فضای سیناپسی
\item چسبیدن پیام‌رسان‌ها به گیرنده‌های نورون پس‌سیناپسی
\item باز شدن کانال‌های یونی برای شارش یون‌ها و تولید پتانسیل عمل در نورون پس‌سیناپسی
\end{itemize}

با انجام مرحله به مرحله این چهار پروسه پیام عصبی به شکل شیمیایی منتقل می‌شود (شکل \ref{fig:synapse}).
 \begin{figure} [htbp]
\centering
\includegraphics[width=10cm , height=5cm]{synapse.png} 
\caption[عملکرد سیناپس الکتریکی و شیمیایی] {\footnotesize عملکرد سیناپس الکتریکی (سمت چپ) و سیناپس شیمیایی (سمت راست) \cite{kandel}.}
\label{fig:synapse}
\end{figure}

و اما سیناپس شیمیایی بسته به نوع نوروترنسمیتر‌ها به دو نوع تحریکی\LTRfootnote{excitatory} و مهاری\LTRfootnote{inhibitory} تقسیم‌بندی می‌شوند \cite{lodish}. نوروترنسمیتر‌ها با مکانیزم‌های قابل توجهی می‌توانند بر عملکرد یک نورون تاثیر بگذارند. از جمله تاثیر‌های مستقیم نوروترنسمیتر‌ها بر روی سلول عصبی تحریک‌پذیری الکتریکی نورون است. نوروترنسمیتر‌ها بر جریان یونی غشا نورون پیش‌سیناپسی تأثیر می‌گذارند و با تحریک یا مهار سلول، امکان تولید پتانسیل عمل را در نورونی که با آن در تماس است (نورون پس‌سیناپسی) به وجود می‌آورد \cite{lodish}.
 \begin{figure} [htbp]
\centering
\includegraphics[width=8cm , height=8.5cm]{excite.png} 
\caption[واکنش سیناپس تحریکی] {\footnotesize واکنش سیناپس تحریکی ($a$) و مهاری ($b$) \cite{lodish}.}
\label{fig:excite}
\end{figure}

\subsection{ جریان سیناپسی}
یون‌هایی که از غشا عبور می‌کنند جریان سیناپسی به وجود می‌آورند که ناشی از جریان کل یون‌های گذرنده از کانال‌های غشا است.
در یک رابطه‌ی ریاضی برای هر سیناپس که بین دو نورون وجود دارد جریان سیناپسی متناظر با آن را طبق رابطه زیر خواهیم داشت:
\begin{equation}
\ I_{\textsc{syn}} = g_{\textsc{pre}}S(t)(V_{\textsc{post}} - V_{\textsc{rev}})
\end{equation}
در این رابطه $g$  و $V_{\textsc{rev}}$ به ترتیب قدرت سیناپسی و پتانسیل بازگشتی را نشان می‌دهند. برای هر یون پتانسیل بازگشتی یا تعادلی، پتانسیل غشا است که در آن شارش جریان خالص از طریق کانال‌های باز صفر است. به عبارت دیگر، در پتانسیل بازگشتی نیرو‌های الکتریکی و شیمیایی با یکدیگر در تعادل هستند. این پتانسیل را می‌توان از طریق معادله نرنست\LTRfootnote{Nernst equation} محاسبه نمود. $S(t)$ نیز ضریب وابسته به زمان برای کانال‌‌‌هایی است که باز هستند. مقدار $S(t)$ همیشه غیر منفی و برای نورون‌هایی که شلیک\LTRfootnote{fire} نکرده‌اند برابر صفر است. برای سیناپس تحریکی، پتانسیل بازگشتی بزرگ‌تر از پتانسیل استراحت است، به طوری که یک جریان به سمت داخل را ایجاد می‌کند. در مقابل پتانسیل بازگشتی برای سیناپس مهاری نزدیک به پتانسیل بازگشتی  یون پتاسیم\RTLfootnote{پتانسیل بازگشتی برای یون پتاسیم $-88.7 \textsc{mV} $ می‌باشد.} می‌باشد \cite{ermen}.

با تعاریف اولیه و آشنا شدن با کارکرد دستگاه عصبی مرکزی، اکنون می‌توانیم توضیحی از شبکه نورونی را بیان داریم. شبکه نورونی از قرار گرفتن تعداد بسیار نورون در کنار هم و اتصال بین نورون‌ها با یکدیگر به وجود می‌آید که  مطالعه رفتار و فعالیت هر قسمت از شبکه منجر به درک درستی از تمام شبکه خواهد شد. شبکه نورونی با اتصال‌های بسیار زیاد خود تداعی‌گر شبکه‌های پیچیده‌ هستند.  در این شبکه، نورونی که به آستانه فعالیت خود برسد و آتش کند نورون‌های دیگر را نیز  با تحت تاثیر قراردادن  خود فعال می‌کند و این عمل به نورون‌های بعد که قابلیت دریافت پالس عصبی  را داشته باشند  سرایت می‌کند. به همین ترتیب فعالیت دسته جمعی در شبکه‌ نورونی به وجود می‌آید. بررسی  رفتار این گروه از نورون‌ها و شناخت و درک از نحوه فعالیت در شبکه نورونی مستلزم مطالعه مختصری از خصوصیات و مفاهیم پایه‌ای از شبکه پیچیده  است که در فصل آتی بیانگر آن‌ها خواهیم بود.
\newpage 
\textbf{خلاصه‌ی فصل اول}    \begin{itemize}
\item نورون واحد پردازنده اطلاعات در سیستم عصبی مرکزی و دارای سه قسمت اصلی دندریت، جسم سلولی و آکسون است.
\item برقراری ارتباط بین نورون‌ها از طریق سیناپس شکل می‌گیرد و بسته به نوع پیام‌رسانی‌شان به دو دسته الکتریکی و شیمیایی تقسیم می‌شوند. 
\item در سیناپس الکتریکی پیام‌رسانی بدون واسطه و از طریق کانال‌های شکاف پیوندگاه انجام می‌شود. در صورتی‌که در سیناپس شیمیایی با آزادشدن پیام‌رسان‌های شیمیایی بین شکاف سیناپسی این عمل صورت می‌گیرد.
\item سیناپس شیمیایی با توجه به نوع پیام‌رسان‌هایش به سیناپس تحریکی و مهاری دسته‌بندی می‌شود. 
\item در سیناپس تحریکی، پتانسیل عمل نورون‌ پیش‌سیناپسی امکان وقوع پتانسیل عمل را در نورون پس‌سیناپسی افزایش می‌دهد. اما در سیناپس مهاری، احتمال رخ‌داد پتانسیل عمل در نورون پس‌سیناپسی توسط پتانسیل عمل نورون پیش‌سیناپسی بسیار اندک است.
\item جریان سیناپسی برای هر نورون متناظر با کل جریان‌های گذرنده از کانال‌های نورون است. این جریان به عواملی مانند میزان رسانندگی سیناپس‌ها، اختلاف پتانسیل غشای نورون پس‌سیناپسی و پتانسیل بازگشتی و نیز کسری از کانال‌‌های باز وابسته است. 
\end{itemize}
