%------------------ call packages -----------------
\documentclass{article} % Defines the document type (e.g., article, report, book) [12pt] default size of all the document
% \documentclass[twocolumn]{article} % for a two column article

\usepackage{geometry} % for page settings like size and margins

\usepackage{amsmath} % for math functions

% clickable table of content
\usepackage[utf8]{inputenc}
\usepackage[colorlinks=true, linkcolor=blue, urlcolor=red, citecolor=green, bookmarks=true, bookmarksopen=true]{hyperref}
\usepackage{tocloft} % adjust table of contents

\usepackage{graphicx} % For figures (images/ charts/ flowcharts...)
\usepackage{chngcntr} % For modifying numbering of tables or figures

\usepackage{booktabs} % For better-looking tables
\usepackage{multirow} %complex tables
\usepackage[table]{xcolor} % color of table rows odd and even

\usepackage[backend=biber, style=numeric]{biblatex} % for referencing (bibilography) %ieee, authoryear for styles

\usepackage{titlesec} % for change setting like size font_family color of headings, sections

% ---------------- customization/ user settings -------------------------
\geometry{
    a4paper, % Paper size
    portrait,
    left=2cm, % Left margin
    right=2cm, % Right margin
    top=2cm, % Top margin
    bottom=2cm % Bottom margin
}

\setcounter{secnumdepth}{5} % Number up to subsubparagraph
\setcounter{tocdepth}{5}    % Include up to subsubparagraph in TOC:table of contents

% Customize section font size/ family
\titleformat{\section}
  {\normalfont\fontsize{16}{19}\bfseries\color{blue}}{\thesection}{1em}{} %16 fontsize 19linespace | bfseries: bold | instead of normal font we can write our font name
\titleformat{\subsection}
  {\normalfont\fontsize{14}{17}\bfseries}{\thesubsection}{1em}{}

\graphicspath{{"//wsl.localhost/Ubuntu/home/mrrahimy/figures/"}} %define folder of figures

%set color of odd and even rows in general
%\rowcolors{1}{gray!20}{white} % Odd rows: light gray, Even rows: white  general for all tables

\counterwithin{figure}{section} % Add section number as a prefix to figure numbers
\counterwithin{table}{section} %add section number as prefix of tables

\addbibresource{references.bib} % Load your .bib file for citation and referencing
% most of articles and books have bibtex style in cite section
% or doi link could used for referencing by doi2bib website

% Customize the TOC table of contents
\setlength{\cftsecindent}{0pt} % No indentation for sections
\setlength{\cftsubsecindent}{0.25cm} % Indentation for subsections
\setlength{\cftsubsubsecindent}{0.5cm} % Indentation for subsubsections
\setlength{\cftparaindent}{0.75cm}
\setlength{\cftsubparaindent}{1cm}
\renewcommand{\cftdotsep}{1} % Spacing between dots

% ----------------------- start of document-----------------------
\title{Mohammad First LaTeX Document}
\author{Your Name}
\date{\today}

\begin{document}
\maketitle % Generates the title
\thispagestyle{empty} %remove page number
\newpage

% table of contents
\pagenumbering{roman} %roman for table of contents
\tableofcontents % Add the table of contents here
\newpage % Force content to start on a new page
\listoftables    % List of tables %shows captions
\newpage % Force content to start on a new page
\listoffigures %charts, diagrams, and other figures
\newpage % Force content to start on a new page

\pagenumbering{arabic} %reset to page 1
\section{Introduction}
Hello, world! This is my first LaTeX document.

% insnert figure
\begin{figure}[h] %h:here
  \centering
  \includegraphics[scale=0.5]{"Screenshot 2024-06-03 130722.png"} % or can define width=4cm, height=3cm
  \caption{An example figure.}
  \label{fig:test} %for referencing in text
\end{figure}
As shown in Figure \ref{fig:test}, this is an example image.
% end of figure

\subsection{First subsection}
random notes generation
\subsubsection{First Subsubsection}
Here is a mathematical equation:
\[
E = mc^2
\]
\paragraph{First Paragraph} 
\par %this go to next paragraph
This is the first paragraph.
\footnote{This is the footnote text.} % insert footnote

% for table we can use df to latex by python ---> latex_code = df.to_latex(index=False, multicolumn=True, multirow=True)
% then using \input{table.tex} that is the export of python
% there are more tools to convert a table for latex in easy and short way

% here we add a manual table
\begin{table}[h]
  \rowcolors{1}{gray!20}{white} %color od rows odd and even
    \centering
    \caption{An example table.}      
    \begin{tabular}{cccc}
      \toprule
      \rowcolor{gray!50} % Header row color
      Column 1 & Column 2 & Column 3 & Column 3 \\
      \midrule
      1 & 2 & 3 & 3 \\
      4 & 5 & 6 & 3 \\
      4 & 5 & 6 & 3 \\
      \bottomrule
    \end{tabular}
    \label{tab:test} %for referencing in text  tab is not keyword, but beneficial to manage lable names!
  \end{table}
%end of table
As you see in table \ref{tab:test} there is no way
\subparagraph{First Subparagraph}
This is the first subparagraph.
According to Einstein \cite{einstein1905}, the theory of relativity is groundbreaking. %reference from .bib file 
Knuth's work on TeX \cite{knuth1984} is also widely recognized.

%complex table
\begin{table}[h]
    \centering
    \caption{A mixed table.} 
    \begin{tabular}{|l|c|r|c|} %left/ center/ right
        \hline
        \multirow{2}{*}{2000} & \multicolumn{2}{c|}{2010} & 2020 \\
        \cline{2-4} % similar to hline both are horizontal
        & Column 2 & Column 3 & Column 4 \\
        \hline
        1 & 2 & 3 & 4 \\
        \hline
        5 & \multicolumn{2}{c|}{6} & 7 \\
        \hline
    \end{tabular}
\end{table}
%end of complex table  

\section{conclusion}
\par
The quadratic formula is:
\[
x = \frac{-b \pm \sqrt{b^2 - 4ac}}{2a}
\]

\par
\[
\nabla \times \mathbf{B} = \mu_0^2 \mathbf{J} + \mu_0 \varepsilon_0 \frac{\partial \mathbf{E}}{\partial t}
\]

\[
  A matrix= 
\begin{pmatrix}
a & b \\
c & d
\end{pmatrix}
\]

\[
\vec{v} = \begin{pmatrix}
  v_x \\
  v_y \\
  v_z
  \end{pmatrix}
\]

\newpage
\printbibliography
\end {document}
