\usepackage[top=3.5cm,right=3cm,bottom=3.5cm,left=3cm]{geometry}  
\usepackage{amsmath}
 \usepackage[marginal,stable,bottom]{footmisc}     % for footnotes: marginal --> the same margins as text, 
                                                                                  %                       stable--> ?
                                                                                  %                       bottom --> starting the footnotes at a fixed place at the bottom of the page.
 
%\usepackage{perpage}                                             % for footnotes: starting from 1 perpage
%\MakePerPage{footnote}

%\usepackage{zref-perpage}
%\zmakeperpage{footnote}

\usepackage{cite}                                                     % for collecting citations: [1,2,3,4] --> [1-4]
\usepackage{setspace}                                         % for switching between double/single space in document
\allowdisplaybreaks                                                    % breaking the lines&pages when needed for the style.
\usepackage{parskip}                                               % ? 
\usepackage{perpage} %the perpage package
\MakePerPage{footnote} %the perpage package command

\def\acknowledgementname{\nastaliq{قدردانی}}
\def\by{توسط}
\def\writtenfor{پایان‌نامه‌ی}
\def\undersupervision{اساتید راهنما:}
\def\underadvising{استاد مشاور:}
\def\departmentof{دانشکدهٔ}
\def\universityof{دانشگاه}
\def\latinby{by}
%\newcommand{\@supervisor}{}
%\newcommand{\supervisor}[1]{\renewcommand{\@supervisor}{#1}}

%\def\latinsupervisor#1{\gdef\@latinsupervisor{#1}}












%\relpenalty=9999                                            % show the neccessity of breaks for lines, changing the number to 10000 cause no break in lines.
%\binoppenalty=9999
\usepackage{float}
\usepackage{xecolour}
\usepackage{amsmath}
\usepackage{commath}
\usepackage{makeidx}
\usepackage{verbatim}
\usepackage[colorlinks,linkcolor=blue,citecolor=blue]{hyperref}
\usepackage{graphicx}
\usepackage{ifthen}
\usepackage{afterpage}
 \parindent0pt


\makeatletter
\pdfstringdefDisableCommands{%
\let\lr\@firstofone
}
\makeatother
% ----------------------------------------------------------------------------
\usepackage{pdfpages}
\usepackage[Lenny]{fncychap}
\usepackage[extrafootnotefeatures]{xepersian}
%\fourcolumnfootnotes
\usepackage{perpage}
\usepackage{zref-abspage}
\settextfont[Scale=1.05]{XB Niloofar}
\setlatintextfont[Scale=0.95]{Times New Roman} 
\setdigitfont[Scale=1.05]{Yas} 
% قلم برای اعداد به صورت فارسی با صفر توخالی، در صورتی که بخواهیم اعداد انگلیسی نوشته شوند این خط را غیر فعال می‌کنیم.

\defpersianfont\nastaliq[Scale=2.0]{IranNastaliq}
% قلم برای نوشتن تقدیم
\defpersianfont\nastaliqone[Scale=1.0]{IranNastaliq}
% قلم نستعلیق با سایز متناسب با متن در صورت نیاز
\defpersianfont\anotherfont[Scale=1.2]{XP Ziba}
% قلم برای نوشتن تشکر (قلم فانتزی)
\newenvironment{fantezi}
{\anotherfont}


\makeindex


\def\beginto{
\newpage
\begin{RTL}
\begin{Huge}
\nastaliq

%\begin{center}
%\vspace*{0.15cm}
%تقدیم به 
}

\def\endto{
~
\end{center}
\end{Huge}
\end{RTL}
}

%\def\beginthanks{
%\newpage

%{\centering\Huge{\nastaliqone{
%تشکر وقدردانی ~\\
%~\\}}}
%}

%\def\endthanks{
%~
%}

%\def\thanks{
%\beginthanks
%شکر شایان نثار ایزد منان که توفيق را رفیق راهم ساخت تا اين پايان نامه را به پايان برسانم.\\\
این پایان‌نامه را ضمن تشکر و سپاس بیکران و در کمال افتخار تقدیم می‌دارم به:\\
استاد عزیز و گران‌‌قدرم، سرکار خانم دکتر ناهید عظیمی\\
استاد فرهیخته‌ام جناب آقای دکتر علیرضا ولیزاده\\
پشتوانه‌های زندگی‌ام: پدر دلسوز، مادر فداکار، خواهران عزیزتر از جان و همراه صبور، همسرم،\\
و همراهان همیشگی‌‌ام در این مدت تحصیل: بهار عزیزم، اکرم دوست‌داشتنی و محمد انصاری‌آرای عزیز،\\
به پاس را‌هنمایی‌ها و حمایت‌های بی‌دریغشان در به تحقق رسیدن اهداف و اتمام این پروژه،\\
به پاس حضور گرم و محبت‌های بی‌چشم‌داشتشان که هرگز فراموش نمی‌شود،\\
به پاس قلب‌های بزرگ‌شان که همواره امید‌بخش من در این دوران سخت بوده‌اند،\\
 و به پاس صبر و حوصله بی‌حدشان که همیشه و در همه جا یاری‌گر من بوده‌اند.\\
\begin{center}
{\large  بوسه بر دستان پرمهرتان.}
\end{center}





%\endthanks
%}

\def\startpage{
\newpage
\vspace*{3cm}
\begin{center}
\includegraphics[width=12cm]{besm}
\end{center}
}
\makeatletter
\def\@makechapterhead#1{%
  \vspace*{50\p@}%
  {\parindent \z@ \centering\normalfont
    \ifnum \c@secnumdepth >\m@ne
      \if@mainmatter
        \huge\bfseries \@chapapp\space \tartibi{chapter} 
        \par\nobreak
        \vskip 20\p@
      \fi
    \fi
    \interlinepenalty\@M
    \Huge \bfseries #1\par\nobreak
    \vskip 40\p@
  }}
\def\@makeschapterhead#1{%
  \vspace*{50\p@}%
  {\parindent \z@ \centering
    \normalfont
    \interlinepenalty\@M
    \Huge \bfseries  #1\par\nobreak
    \vskip 40\p@
  }}
\makeatother



\renewcommand\bibname{مراجع}
\def\contentsname{فهرست}

% some extra diffinitions %%%%%%%%%%%%%%%%%
%%%%%%%%%%%%%%%%%%%%%%%%%%%%%

\def\bea{\begin{eqnarray}}
\def\eea{\end{eqnarray}}

\def\ba{\begin{array}}
\def\ea{\end{array}}

\def\beq{\begin{equation}}
\def\eeq{\end{equation}}

\def\ni{\noindent}
\def\nn{\nonumber}

\def\bc{\begin{center}}
\def\ec{\end{center}}
