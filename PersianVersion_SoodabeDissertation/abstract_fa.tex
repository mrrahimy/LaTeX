تئوری تراوش استاندارد،  به مطالعه رفتار  اندازه خوشه‌ها در  شبکه  می‌پردازد. یکی از مسائل مهم در تئوری تراوش یافتن اندازه و نقطه ظهورِ خوشه‌ی بی‌نهایت از مکان‌های اشغال شده است. یک مدل تعمیم‌یافته از تراوش استاندارد، مدلی است که به تراوش خودراه‌انداز معروف است. این مدل برای توصیف گسترش  فرایند فعال‌سازی روی شبکه‌ها معرفی شده است که در آن فعال‌سازی به شکل دسته‌جمعی و با فعال شدن متوالی مکان‌ها در شبکه ظاهر می‌شود.  فرایند فعال‌سازی در شبکه‌های نورونی مثالی بارز از تراوش خودراه‌انداز است. در این پایان‌نامه، فرایند فعال‌سازی نورون‌ها در شبکه نورونی را مورد بررسی قرار می‌دهیم. در یک شبکه عصبی زمانی که نورون به آستانه فعالیت می‌رسد و آتش می‌کند،‌ نورون‌های همسایه را تحت تاثیر خود قرار می‌دهد.  در مدل تراوش خودراه‌انداز، یک نورون در صورتی فعال خواهد شد که حداقل $m$ تا از نورون‌های همسایه آن فعال باشند. با آتش کردن نورون‌ها به طور دسته‌جمعی خوشه‌ای به هم پیوسته از نورون‌های فعال به دست می‌آید که کسر قابل توجهی از کل شبکه نورونی را به خود اختصاص می‌دهد. اندازه بزرگ‌ترین خوشه به هم پیوسته از مکان‌های فعال را با بهره‌گیری از روش‌های تراوش خودراه‌انداز و با استفاده از شبیه‌سازی به دست می‌آوریم. در این پایان‌نامه برای سادگی ابتدا مدل را بر روی شبکه تصادفی با توزیع درجه یکنواخت پیاده می‌کنیم و سپس آن را به یک شبکه با تابع توزیع درجه گاوسی که مناسب برای شبکه‌ی نورونی است تعمیم خواهیم داد. مشاهده خواهیم کرد که با افزایش تعداد نورون‌های فعال اولیه، تعداد نورون‌های فعال نهایی شبکه افزایش می‌یابد و در یک نقطه به طور ناگهانی رشد می‌کند. همچنین اندازه خوشه به هم پیوسته از نورون‌های فعال، گذار فازی را نشان خواهد داد که ترکیبی از گذار فازهای مرتبه اول و دوم است. این گذار فازها به ازای توابع توزیع یکنواخت و گاوسی برای آستانه‌های متفاوت بررسی می‌شود.

\keywords{ فعال‌سازی ، تراوش خودراه‌انداز ، بزرگ‌ترین خوشه فعال به‌ هم‌پیوسته ، گذار فاز}